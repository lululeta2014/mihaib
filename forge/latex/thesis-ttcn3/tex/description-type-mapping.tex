\section{Type Mapping}
\label{sec:type-mapping}


\subsection{Primitive Types}
\label{sec:type-mapping-primitive-types}

The Java programming language defines 8 built-in types
(as detailed in \citep{website:learning-java}%
\footnote{\url{http://java.sun.com/docs/books/tutorial/%
	java/nutsandbolts/datatypes.html}}):
\verb=byte=, \verb=short=, \verb=int=, \verb=long=,
\verb=float=, \verb=double=, \verb=boolean= and \verb=char=.

Table \ref{tab:ttcn3-prim-types} lists
the primitive data types defined by the \ac{TTCN-3} language,
which are of interest for mapping Java types.
Other built in data types reflect the language's design and use
for protocol testing:\\
\verb=verdicttype=, \verb=bitstring=, \verb=hexstring=, \verb=octetstring=.\\
In addition, the language allows the user to define custom
structured types
(e.g.\ \verb=enumerated=, \verb=record=, \verb=set=, \verb=union=)
and list types (\verb=array=, \verb=set of=, \verb=record of=).
This allows the test developer to express data structures as close as possible
to the real data transmitted when running test scenarios
in a variety of test domains.

\begin{table}
\centering
\begin{tabular}{| r || l |}
	\hline
	\ac{TTCN-3} built-in type	& Description \\
	\hline
	\verb=boolean=	& May take the values \verb=true= and \verb=false= \\
	\verb=integer=	& Arbitrarily large integer values \\
	\verb=float=	& Arbitrarily large real numbers \\
	\verb=(universal) charstring=	& String of (unicode) characters \\
	\hline
\end{tabular}
\caption{TTCN-3 primitive types of interest\label{tab:ttcn3-prim-types}}
\end{table}

\ac{TTCN-3} also allows the definition of new types
by restricting the value range of an existing type.
The basic syntax for numeric type ranges is presented in
Listing \ref{prog:type-range-restrict}.
The built-in \verb=integer= and \verb=float= types
permit arbitrarily large values.
Java integer types (\verb=byte=, \verb=short=, \verb=int=, \verb=long=)
can be mapped to specific value ranges of the unbounded \verb=integer= type.
The Java \verb=float= and \verb=double= types are mapped to
\ac{TTCN-3}'s \verb=float= type.
The \verb=char= type in Java can be mapped to a
\ac{TTCN-3} \texttt{universal charstring} of length \verb=1=.
The \verb=boolean= type can be directly mapped between the two languages
since it has the same semantic.
The Java \verb=void= type may only be used as a method return type.
Such Java methods will be mapped to
\ac{TTCN-3} signatures with no return value.
The mapping of primitive Java types to \ac{TTCN-3} is shown in
Listing \ref{prog:java-primitive-type-mapping}.

\begin{program}
\verbatimtabinput{./ttcn3/type-range-restrict.ttcn3}
\caption{New type definition by restricting the value range %
	of an existing type\label{prog:type-range-restrict}}
\end{program}

\begin{program}
\verbatimtabinput{./ttcn3/java-primitive-type-mapping.ttcn3}
\caption{Mapping of primitive Java types to TTCN-3%
	\label{prog:java-primitive-type-mapping}}
\end{program}


\subsection{Using Java Objects from TTCN-3 code}
\label{sec:using-java-objects-from-ttcn3}

In order to use Java objects from \ac{TTCN-3} code,
test developers will use \emph{handles}.
Upon creation, objects are stored in a map (called the \emph{object registry}).
Each \verb=Object= gets a unique \emph{key} associated to it in this registry.
The handles used in \ac{TTCN-3} code encapsulate information about the key
and the programmer is not concerned
about the mechanism used to translate the handle into an object reference.
Future changes in the information contained in the handle
should have little or no impact on client code.

The object registry uses \verb=String=s as keys.
\ac{TTCN-3} provides a reserved type, \verb=address=,
which must be defined before being used.
It will be used as the type for object handles,
and the contents will be hidden from the user.
The \verb=address= type has been defined as a charstring
in Listing \ref{prog:address-definition}.
The current registry implementation uses \verb=String=s
representing increasing integer numbers for the keys
(e.g.\ \verb="1"=, \verb="2"=).

\begin{program}
\verbatimtabinput{./ttcn3/address-definition.ttcn3}
\caption{Definition of \texttt{address} as handle type%
	\label{prog:address-definition}}
\end{program}


\subsubsection{Mapping signatures to instance methods}

The procedure oriented syntax of the \ac{TTCN-3} language
allows the programmer to define and call signatures
which may take parameters, may return values and may throw exceptions.
This syntax covers most aspects of methods in an object oriented language.
But methods in an object oriented language
may be \emph{invoked} for different \emph{object instances}.

A description of how this can be achieved by a compiler is given in
\citep[pg.~116]{tij}:

\begin{quotation}
If there’s only one method being called, how can that method know
whether it’s being called for one object or another?
To allow you to write the code in a convenient object-oriented syntax
in which you ``send a message to an object,''
the compiler does some undercover work for you.
There’s a secret first argument passed to the method,
and that argument is the reference to the object that’s being manipulated.
\end{quotation}

This mechanism is used at runtime to map \ac{TTCN-3} signature calls
to Java instance method invocations.
When instance methods are mapped to \ac{TTCN-3} signatures,
an extra first parameter is introduced,
as shown in Listing \ref{prog:signature-not-type-safe}.
Static methods need no additional parameter for the object handle.

\begin{program}
\verbatimtabinput{./ttcn3/signature-not-type-safe.ttcn3}
\caption{A signature definition (with no type safety)%
	\label{prog:signature-not-type-safe}}
\end{program}

After solving the issue of specifying the instance handle
when calling a signature,
another (not critical but important) issue comes to light: \emph{Type Safety}.
If handles for all Java object types are \ac{TTCN-3} \verb=address=es,
then no type checking is done when compiling the \ac{TTCN-3} source.
Whenever signatures require an object handle as parameter,
any variable of type \verb=address= may be used.
If the test developer supplies handles for objects of the wrong type
the \ac{TTCN-3} compiler will be unable to detect this.
When the compiled test cases are executed,
runtime errors will show the problem which can then be corrected.
This method for correcting errors in the source code
has two important disadvantages:
\begin{itemize}
\item The errors get reported one by one
after repeatedly compiling and executing the code
(instead of all of them being reported after a single compilation)
\item Type errors will not be reported on execution branches
which have never been taken (even after compiling and running the test cases).
\end{itemize}


\subsubsection{Type Safe Object Handles}
\label{sec:type-safe-object-handles}

As previously shown, having the same type (\verb=address=) for all object types
provides no type safety and type errors will remain undetected at compile time.
To overcome this issue, information about the particular type
must be added to the handle.

\ac{TTCN-3}'s powerful type definition mechanism
can be used to solve this issue.
The language uses \emph{type structure}
to assert the equivalence of data types.
More precisely the two types (\verb=rec1= and \verb=rec2=) defined
in Listing \ref{prog:type-structure-equivalence} are equivalent
in the sense that \emph{a variable of one type may successfully be used
where the other type is required}.
This example hints at the final solution:
a structured (\verb=record=) type with fields for:
\begin{description}
\item[The type name] A \verb=charstring= representation of the Java type name
\item[The handle] The previously mentioned key (of type \verb=address=)
for the \emph{object registry} managed by the codec
\end{description}

\begin{program}
\verbatimtabinput{./ttcn3/type-structure-equivalence.ttcn3}
\caption{Type equivalence is based on type structure%
	\label{prog:type-structure-equivalence}}
\end{program}

The final solution is shown in Listing \ref{prog:type-safe-record-type}.
\ac{TTCN-3}'s type equivalence takes into account type structure
including restrictions placed on \verb=field=s within a \verb=record=.
By defining each handle type as a record with a different structure
(more precisely with a different restriction on the values allowed
for the \verb=typeName= field)
a signature which expects an object handle of a particular type
will determine a compile time error if a handle of the wrong type is supplied
by the test developer --- and thus type safety is achieved.

\begin{program}
\verbatimtabinput{./ttcn3/type-safe-record-type.ttcn3}
\caption{A type-safe handle type\label{prog:type-safe-record-type}}
\end{program}


\subsubsection{The type encoding variant}
\label{sec:type-encoding-variant}

The careful reader may have noticed the \verb=variant=
following the handle type definition
in Listing \ref{prog:type-safe-record-type}.
This is an \emph{attribute} used to specify the \emph{encoding variant}.
Such an attribute may be specified for a \ac{TTCN-3} module
(and inherited by types defined within that module)
or it may be specified for each type individually.
The \verb=variant= attribute is needed because of the
architecture of a \ac{TTCN-3} test system.

Being designed for message exchanging between several parallel components,
a \ac{TTCN-3} test system is designed
to decode messages in the following manner:
\begin{itemize}
\item Messages arrive (encoded) as a string of bytes
\item The codec is asked to decode the message knowing:
	\begin{itemize}
	\item The \emph{decoding hypothesis} --- the structure of the template
	specified in a \verb=receive= operation
	\item The \emph{encoding variant} of the received data
	\end{itemize}
\item The data (if successfully decoded using the given hypothesis)
is matched against the expected \emph{receive template}.
If it matches (by conforming to all restrictions) the message is dequeued,
otherwise the next alternative is considered.
\end{itemize}

Because the \emph{decoding} operation is separate from the
\emph{pattern matching} between the received data and the expected template,
the codec only knows about the \emph{decoding hypothesis}
(i.e.\ expected structure of the received message ---
without knowing of any restrictions for field values)
and the \emph{encoding variant}.
The codec is only responsible for checking if the received message
corresponds to the particular decoding hypothesis
and (if it does) for decoding it (possibly according to the encoding variant).
A \emph{pattern match} is then performed between the decoded data
and the expected receive template
(which may stand for a single return value or for a set of return values).
If the restrictions imposed by the receive template hold for the received data
this is considered a \emph{match}
and the message is removed from the top of the receive queue.
Otherwise a \emph{mismatch} occurs and the next alternative is considered.

As will be shown in section \ref{sec:storage-codec}
the codec's responsibility is to
encode \ac{TTCN-3} values (both primitive values and handles)
to Java (for selecting objects and passing arguments to method invocations)
and to decode returned Java values to expected \ac{TTCN-3} types.
Since a Java object may be assigned a handle for any of its super-types
(classes extended or interfaces implemented by the object's type)
the codec needs to know the target type of the decoding hypothesis
(not just the type's structure) ---
otherwise an incorrect pattern mismatch will occur immediately after decoding.
This is especially useful when catching exceptions
(as the programmer may use a handle of type \verb=Throwable=
in order to catch any thrown Exceptions).
For these purposes, the encoding variant is used
to supply the codec with information about the target decoded type.
This also allows the system to detect type errors as soon as possible
(i.e.\ when assigning a handle of the wrong type
to an object returned by a method call
and not when that object is (incorrectly) used later).


\subsubsection{Type casting of object Handles}
\label{sec:type-casting}

This solution for adding type safety introduces an inconvenient:
\emph{Upcasting}.
Java types form a (singly rooted) hierarchy.
These types may be automatically \emph{upcast}
i.e.\ an object of a derived type may be used
whenever an object of a base type is required
(as in other object oriented languages).
If class \verb=Sparrow= is derived from class \verb=Bird=,
Java allows the use of a \verb=Sparrow= object
whenever a \verb=Bird= is expected.

In Java, a \verb=Sparrow= \emph{is} a \verb=Bird=.
In \ac{TTCN-3} this is not true, since the types have different structures
and the type system is not object oriented
(a Java class may also implement several \verb=interface=s).
This inconvenient is overcome by generating a \texttt{casting template}
as illustrated in Listing \ref{prog:casting-template}.
As with explicit type casting in other programming languages,
incorrect usage of the casting operator can only be detected at runtime.
Upon usage of objects after casting, at runtime,
all objects are checked to be of the correct type.

\begin{program}
\verbatimtabinput{./ttcn3/casting-template.ttcn3}
\caption{The casting template for type \texttt{java.util.Map}%
	\label{prog:casting-template}}
\end{program}

In order to allow casting from \emph{any} type to a particular type
(in the example in Listing \ref{prog:casting-template} casting from any type
to \texttt{java.util.Map} is allowed)
the \texttt{casting template} must be allowed to take
\emph{any} object handle as parameter.
The type safety just discussed must be circumvented in this particular case.
This has been accomplished with the use of a special, predefined handle type
with the same structure as all other handles but without any restriction
on the values assigned to the \verb=typeName= field.
This special type has been called \verb=ObjectType=
and its definition is shown in Listing \ref{prog:generic-object-type}.
Since no restrictions are placed on the fields
in the \verb=ObjectType= record type,
a handle for any other type may be passed
where an \verb=ObjectType= is required (e.g.\ to the casting operator).

\begin{program}
\verbatimtabinput{./ttcn3/generic-object-type.ttcn3}
\caption{The generic (unrestricted) ObjectType\label{prog:generic-object-type}}
\end{program}


\subsection{Mapping Java Object Types}
\label{sec:mapping-java-object-types}

After having examined the task of mapping types between the two languages,
a more detailed presentation follows.
The previously mentioned elements are put together
to show the structure of the code resulted after mapping a Java type.

The \emph{mapping} operation (as defined and used in this project)
treats a Java type as a translation unit.
In other words the mapping operation receives a Java type as parameter.
If several Java types need to be mapped
(as might usually be the case for a test scenario)
then the mapping is performed once for each type.

Each Java type is mapped to a \ac{TTCN-3} module.
Each generated module contains the following elements:

\begin{description}

\item[a Handle Type definition]
for \ac{TTCN-3} references to objects of the mapped Java type.
The name of this handle type can be based on the module name
or customized by the test developer.
\item[Method signatures]
for all public methods of the Java type.
These signatures are complete with return values and exception handling.
Primitive Java types (for parameters and return values)
are mapped to their \ac{TTCN-3} counterparts as detailed in
section \ref{sec:type-mapping-primitive-types}.
Object types (for parameters, return values and thrown exception types)
are mapped to \ac{TTCN-3} record types which act as handle types,
as shown in section \ref{sec:mapping-java-object-types}.
For this reason, all types used in signatures need to be defined in the module
(apart from the actual Java type for which the mapping is generated).
In order to allow the user to handle \emph{unchecked exceptions},
each method signature gets \texttt{java.lang.Throwable}
added to its list of thrown exceptions.

Signatures generated for instance methods require an extra first parameter:
the handle of the object for which the method is to be invoked.
Static methods (also called class methods) are mapped to signatures
which require no such additional parameter.

\item[Constructor signatures]
for all public constructors present in the class.
Constructor parameters and thrown exception types
are treated in the same way as for method signatures.
Constructor signatures always return the type
for which the mapping (the whole module) has been generated.
Constructors have no additional first parameter for an object handle.

\item[Accessor signatures]
are generated for all public fields found in the type
(e.g.\ class, interface, enumeration).
Since any code using this Java type is able to access public fields,
the same functionality must be available to test developers.
As a note, this functionality is needed
in order to access Java enumeration constants
which are actually \texttt{public static final} fields of their enclosing type.

A \emph{getter} signature is generated for all public fields in the class.
This signature's return type is the field type, and it throws no exceptions.

A \emph{setter} signature is generated for public fields
which are not \verb=final=.
This signature requires a parameter
(a primitive value or a handle for an object type)
representing the new value for the field.
It has no return value and throws no exceptions.

Just like method signatures, accessor signatures have an extra parameter
(the object handle) when they correspond to an instance field.
This additional first parameter is not present
for accessor signatures manipulating static fields.

\item[a Casting Template]
needed for casting to the mapped Java type.
The Java programming language allows
both explicit casting and automatic upcasting.
These features may be needed in \ac{TTCN-3} test code
and are provided by the generated template.
This issue is explained in more detail in section \ref{sec:type-casting}.

\item[a Port Type]
for procedure oriented communication
having all the above generated signatures added.
This is a convenience for the user:
instead of defining a port type and listing the needed signatures,
the auto-generated one can be used directly.
For test scenarios where many Java types are used,
it may be convenient to have all signature calls through a single port.
In such a case the test developer must define a port type
and list the signatures of interest.

\item[Handle types for all object types used]
As previously explained, \ac{TTCN-3} signatures need specific handle types
to assure type safety.
For this reason all types used in the module need to be defined at the end.
Test cases using multiple modules will not encounter errors
due to redefinitions of the same type
(e.g.\ \texttt{java.lang.Throwable}) in several modules.
\ac{TTCN-3} allows importing types with the same name from several modules
and requires fully qualified names for them (to avoid ambiguities).
The user will mostly need the types that the mapping has been generated for
(the handle for such types is at the start of the generated module)
and these have custom names
(which, being chosen by the test developer, should not cause clashes).
In the case when a type imported from several modules needs to be used,
the user may generate a short-named template
or (s)he may use the fully qualified name (\texttt{moduleName.typeName}).

\item[a module \texttt{variant} attribute]
as described in section \ref{sec:type-encoding-variant}.
Its role is to specify the Java type the mapping has been generated for.
It is inherited by all module elements (e.g.\ record types, signatures)
which don't have their own \texttt{variant} attribute defined.
It is used by the codec for finding out the type that these elements belong to.

\end{description}

\subsubsection{An example mapping}

As an example, the mapping for the Java class in Listing
\ref{prog:example-mapping-java-class.java} will be presented.
This is a simple class simulating a pocket calculator.
It has two public \verb=field=s of type \verb=double=
(for the result shown on its display and for the contents of its memory)
and several methods for performing computations,
storing the result in memory
and some convenience methods returning an integer representation
(of type \verb=long=)
of the floating point values accesible through the public fields.

\begin{program}
\verbatimtabinput{./java/example-mapping-java-class.java}
\caption{A Java class to be mapped%
	\label{prog:example-mapping-java-class.java}}
\end{program}

Listing \ref{prog:example-mapping-java-class.java}
shows only an overview of the Java class
(code for the shorter methods has been included;
comments for the longer methods
should be sufficient to make their meaning clear)
As stated in section \ref{sec:mapping-java-object-types}
a Java type is a translation unit for the language mapping operation.
An overview of the resulting mapping is shown in
Listing \ref{prog:example-mapping-java-class.ttcn3}.
The previously discussed \ac{TTCN-3} elements
(e.g.\ signatures, record types, a casting template)
are generated for all methods accessible in the Java class
and for all types used (as arguments, return values or thrown exceptions).
The listing is larger and
only some significant elements have been chosen for inclusion here.

\begin{program}
\scriptsize
\verbatimtabinput[4]{./ttcn3/example-mapping-java-class.ttcn3}
\caption{The resulted mapping%
	\label{prog:example-mapping-java-class.ttcn3}}
\end{program}

The mapping operation generates a \verb=module= for each Java type mapped.
The first statement always \verb=import=s the \verb=JAVAAUX= module.
This is a utility module needed by all generated code
(please consult Appendix \ref{sec:javaaux-appendix}
for the complete module listing).
A (type safe) handle record type for the imported Java class is then defined.
The next part consists of \verb=signature= definitions.
These include all signatures available in the type
(i.e.\ those defined by the type,
by all its superclasses and by all implemented interfaces).
Signatures for all public constructors are then defined
followed by accessor signatures for all public fields.

The casting template is also generated.
As previously stated, \ac{TTCN-3} is not an object oriented language
and type equivalence is asserted based on type structure.
When testing Java code, it may be the case that an object of a different type
may need to be supplied where a \verb=Calculator= is expected.
If that type \emph{is} a \verb=Calculator= (i.e.\ it is a subclass)
then the substition is valid in Java.
In order to make this possible in \ac{TTCN-3} the handle must
be manipulated (and the \verb=typeName= \verb=charstring= field changed).
This is accomplished by using the provided \emph{casting template}.

In order to call signatures or send messages in \ac{TTCN-3},
component ports must be used.
To call a \verb=signature=, a \verb=port= of type \verb=procedure=
must be used, and the signature must be listed in the port definition.
Test developers may define port types and list all signatures of interest
(possibly from several generated or user defined modules).
However, as a convenience,
each generated module contains a port type definition
for use with all the signatures from that module.

The last part of the generated module contains
\ac{TTCN-3} record type definitions for all Java types used in the module
(as parameter or return value types).
As an example, the inner class \verb=Operations= has been included in
Listing \ref{prog:example-mapping-java-class.ttcn3}.
The \verb=typeName= field is restricted to the binary name of the Java type,
and each record has an \emph{encoding variant} with the binary name,
needed by the codec.
