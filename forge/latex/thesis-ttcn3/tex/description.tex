\chapter{Solution Description}
\label{chap:description}

This chapter will present the design and implementation of a solution
for the problem stated in Chapter \ref{chap:motivation}.
The solution has been designed according to the previously stated requirements.


\section{Overview}
From an architectural point of view
the solution consists of the following components:
\begin{description}
\item[Name mangling]
	This component is responsible for generating the \ac{TTCN-3} types
	(e.g.\ signatures, object types, port types)
	used in writing testcases.
	At runtime it will perform a reverse mangling
	between the types used in the abstract test suite
	and the original Java types.
\item[Codec]
	This component implements the standard functionality of
	encoding (and decoding) between
	\ac{TTCN-3} types and the corresponding Java types.
	In addition the Codec is responsible for storing the Java objects
	(since they are stateful and must survive
	between method invocations).
	The Codec encodes/decodes all parameters for signature calls
	(the parameters may be primitive values or object references)
	as well as thrown exceptions.
\item[Port Plugin]
	A \ac{TTCN-3} test system consists of several parallel test components.
	Each component may have several ports, and each port can be used
	to send or receive certain types of messages
	(or to make or receive calls for certain signatures).
	The abstract communication model
	requires ports to be connected to one another before being used.
	In this project,
	whenever a (generated) signature is called through a port,
	the Port Plugin is asked to handle this call.
	It receives the respective signature and the encoded parameters
	and is responsible for invoking the actual Java method requested.
	It also enqueues returned values or thrown exceptions
	for processing by the testcase code.
\end{description}

The implementation language is Java.
By using the \verb=reflection= package
an examination of compiled code is made possible.
This permits the automatic generation of corresponding \ac{TTCN-3} types
for any compiled Java types (e.g.\ classes, interfaces, enumerations).
The codec and port plugin implement standard test system interfaces.
The whole project is compiled as a plugin for
Testing Technologies' TTworkbench \citep{website:ttworkbench}
(a test development and execution environment
based on the Eclipse \citep{website:eclipse} \ac{IDE}).

\section{Type Mapping}
\label{sec:type-mapping}


\subsection{Primitive Types}
\label{sec:type-mapping-primitive-types}

The Java programming language defines 8 built-in types
(as detailed in \citep{website:learning-java}%
\footnote{\url{http://java.sun.com/docs/books/tutorial/%
	java/nutsandbolts/datatypes.html}}):
\verb=byte=, \verb=short=, \verb=int=, \verb=long=,
\verb=float=, \verb=double=, \verb=boolean= and \verb=char=.

Table \ref{tab:ttcn3-prim-types} lists
the primitive data types defined by the \ac{TTCN-3} language,
which are of interest for mapping Java types.
Other built in data types reflect the language's design and use
for protocol testing:\\
\verb=verdicttype=, \verb=bitstring=, \verb=hexstring=, \verb=octetstring=.\\
In addition, the language allows the user to define custom
structured types
(e.g.\ \verb=enumerated=, \verb=record=, \verb=set=, \verb=union=)
and list types (\verb=array=, \verb=set of=, \verb=record of=).
This allows the test developer to express data structures as close as possible
to the real data transmitted when running test scenarios
in a variety of test domains.

\begin{table}
\centering
\begin{tabular}{| r || l |}
	\hline
	\ac{TTCN-3} built-in type	& Description \\
	\hline
	\verb=boolean=	& May take the values \verb=true= and \verb=false= \\
	\verb=integer=	& Arbitrarily large integer values \\
	\verb=float=	& Arbitrarily large real numbers \\
	\verb=(universal) charstring=	& String of (unicode) characters \\
	\hline
\end{tabular}
\caption{TTCN-3 primitive types of interest\label{tab:ttcn3-prim-types}}
\end{table}

\ac{TTCN-3} also allows the definition of new types
by restricting the value range of an existing type.
The basic syntax for numeric type ranges is presented in
Listing \ref{prog:type-range-restrict}.
The built-in \verb=integer= and \verb=float= types
permit arbitrarily large values.
Java integer types (\verb=byte=, \verb=short=, \verb=int=, \verb=long=)
can be mapped to specific value ranges of the unbounded \verb=integer= type.
The Java \verb=float= and \verb=double= types are mapped to
\ac{TTCN-3}'s \verb=float= type.
The \verb=char= type in Java can be mapped to a
\ac{TTCN-3} \texttt{universal charstring} of length \verb=1=.
The \verb=boolean= type can be directly mapped between the two languages
since it has the same semantic.
The Java \verb=void= type may only be used as a method return type.
Such Java methods will be mapped to
\ac{TTCN-3} signatures with no return value.
The mapping of primitive Java types to \ac{TTCN-3} is shown in
Listing \ref{prog:java-primitive-type-mapping}.

\begin{program}
\verbatimtabinput{./ttcn3/type-range-restrict.ttcn3}
\caption{New type definition by restricting the value range %
	of an existing type\label{prog:type-range-restrict}}
\end{program}

\begin{program}
\verbatimtabinput{./ttcn3/java-primitive-type-mapping.ttcn3}
\caption{Mapping of primitive Java types to TTCN-3%
	\label{prog:java-primitive-type-mapping}}
\end{program}


\subsection{Using Java Objects from TTCN-3 code}
\label{sec:using-java-objects-from-ttcn3}

In order to use Java objects from \ac{TTCN-3} code,
test developers will use \emph{handles}.
Upon creation, objects are stored in a map (called the \emph{object registry}).
Each \verb=Object= gets a unique \emph{key} associated to it in this registry.
The handles used in \ac{TTCN-3} code encapsulate information about the key
and the programmer is not concerned
about the mechanism used to translate the handle into an object reference.
Future changes in the information contained in the handle
should have little or no impact on client code.

The object registry uses \verb=String=s as keys.
\ac{TTCN-3} provides a reserved type, \verb=address=,
which must be defined before being used.
It will be used as the type for object handles,
and the contents will be hidden from the user.
The \verb=address= type has been defined as a charstring
in Listing \ref{prog:address-definition}.
The current registry implementation uses \verb=String=s
representing increasing integer numbers for the keys
(e.g.\ \verb="1"=, \verb="2"=).

\begin{program}
\verbatimtabinput{./ttcn3/address-definition.ttcn3}
\caption{Definition of \texttt{address} as handle type%
	\label{prog:address-definition}}
\end{program}


\subsubsection{Mapping signatures to instance methods}

The procedure oriented syntax of the \ac{TTCN-3} language
allows the programmer to define and call signatures
which may take parameters, may return values and may throw exceptions.
This syntax covers most aspects of methods in an object oriented language.
But methods in an object oriented language
may be \emph{invoked} for different \emph{object instances}.

A description of how this can be achieved by a compiler is given in
\citep[pg.~116]{tij}:

\begin{quotation}
If there’s only one method being called, how can that method know
whether it’s being called for one object or another?
To allow you to write the code in a convenient object-oriented syntax
in which you ``send a message to an object,''
the compiler does some undercover work for you.
There’s a secret first argument passed to the method,
and that argument is the reference to the object that’s being manipulated.
\end{quotation}

This mechanism is used at runtime to map \ac{TTCN-3} signature calls
to Java instance method invocations.
When instance methods are mapped to \ac{TTCN-3} signatures,
an extra first parameter is introduced,
as shown in Listing \ref{prog:signature-not-type-safe}.
Static methods need no additional parameter for the object handle.

\begin{program}
\verbatimtabinput{./ttcn3/signature-not-type-safe.ttcn3}
\caption{A signature definition (with no type safety)%
	\label{prog:signature-not-type-safe}}
\end{program}

After solving the issue of specifying the instance handle
when calling a signature,
another (not critical but important) issue comes to light: \emph{Type Safety}.
If handles for all Java object types are \ac{TTCN-3} \verb=address=es,
then no type checking is done when compiling the \ac{TTCN-3} source.
Whenever signatures require an object handle as parameter,
any variable of type \verb=address= may be used.
If the test developer supplies handles for objects of the wrong type
the \ac{TTCN-3} compiler will be unable to detect this.
When the compiled test cases are executed,
runtime errors will show the problem which can then be corrected.
This method for correcting errors in the source code
has two important disadvantages:
\begin{itemize}
\item The errors get reported one by one
after repeatedly compiling and executing the code
(instead of all of them being reported after a single compilation)
\item Type errors will not be reported on execution branches
which have never been taken (even after compiling and running the test cases).
\end{itemize}


\subsubsection{Type Safe Object Handles}
\label{sec:type-safe-object-handles}

As previously shown, having the same type (\verb=address=) for all object types
provides no type safety and type errors will remain undetected at compile time.
To overcome this issue, information about the particular type
must be added to the handle.

\ac{TTCN-3}'s powerful type definition mechanism
can be used to solve this issue.
The language uses \emph{type structure}
to assert the equivalence of data types.
More precisely the two types (\verb=rec1= and \verb=rec2=) defined
in Listing \ref{prog:type-structure-equivalence} are equivalent
in the sense that \emph{a variable of one type may successfully be used
where the other type is required}.
This example hints at the final solution:
a structured (\verb=record=) type with fields for:
\begin{description}
\item[The type name] A \verb=charstring= representation of the Java type name
\item[The handle] The previously mentioned key (of type \verb=address=)
for the \emph{object registry} managed by the codec
\end{description}

\begin{program}
\verbatimtabinput{./ttcn3/type-structure-equivalence.ttcn3}
\caption{Type equivalence is based on type structure%
	\label{prog:type-structure-equivalence}}
\end{program}

The final solution is shown in Listing \ref{prog:type-safe-record-type}.
\ac{TTCN-3}'s type equivalence takes into account type structure
including restrictions placed on \verb=field=s within a \verb=record=.
By defining each handle type as a record with a different structure
(more precisely with a different restriction on the values allowed
for the \verb=typeName= field)
a signature which expects an object handle of a particular type
will determine a compile time error if a handle of the wrong type is supplied
by the test developer --- and thus type safety is achieved.

\begin{program}
\verbatimtabinput{./ttcn3/type-safe-record-type.ttcn3}
\caption{A type-safe handle type\label{prog:type-safe-record-type}}
\end{program}


\subsubsection{The type encoding variant}
\label{sec:type-encoding-variant}

The careful reader may have noticed the \verb=variant=
following the handle type definition
in Listing \ref{prog:type-safe-record-type}.
This is an \emph{attribute} used to specify the \emph{encoding variant}.
Such an attribute may be specified for a \ac{TTCN-3} module
(and inherited by types defined within that module)
or it may be specified for each type individually.
The \verb=variant= attribute is needed because of the
architecture of a \ac{TTCN-3} test system.

Being designed for message exchanging between several parallel components,
a \ac{TTCN-3} test system is designed
to decode messages in the following manner:
\begin{itemize}
\item Messages arrive (encoded) as a string of bytes
\item The codec is asked to decode the message knowing:
	\begin{itemize}
	\item The \emph{decoding hypothesis} --- the structure of the template
	specified in a \verb=receive= operation
	\item The \emph{encoding variant} of the received data
	\end{itemize}
\item The data (if successfully decoded using the given hypothesis)
is matched against the expected \emph{receive template}.
If it matches (by conforming to all restrictions) the message is dequeued,
otherwise the next alternative is considered.
\end{itemize}

Because the \emph{decoding} operation is separate from the
\emph{pattern matching} between the received data and the expected template,
the codec only knows about the \emph{decoding hypothesis}
(i.e.\ expected structure of the received message ---
without knowing of any restrictions for field values)
and the \emph{encoding variant}.
The codec is only responsible for checking if the received message
corresponds to the particular decoding hypothesis
and (if it does) for decoding it (possibly according to the encoding variant).
A \emph{pattern match} is then performed between the decoded data
and the expected receive template
(which may stand for a single return value or for a set of return values).
If the restrictions imposed by the receive template hold for the received data
this is considered a \emph{match}
and the message is removed from the top of the receive queue.
Otherwise a \emph{mismatch} occurs and the next alternative is considered.

As will be shown in section \ref{sec:storage-codec}
the codec's responsibility is to
encode \ac{TTCN-3} values (both primitive values and handles)
to Java (for selecting objects and passing arguments to method invocations)
and to decode returned Java values to expected \ac{TTCN-3} types.
Since a Java object may be assigned a handle for any of its super-types
(classes extended or interfaces implemented by the object's type)
the codec needs to know the target type of the decoding hypothesis
(not just the type's structure) ---
otherwise an incorrect pattern mismatch will occur immediately after decoding.
This is especially useful when catching exceptions
(as the programmer may use a handle of type \verb=Throwable=
in order to catch any thrown Exceptions).
For these purposes, the encoding variant is used
to supply the codec with information about the target decoded type.
This also allows the system to detect type errors as soon as possible
(i.e.\ when assigning a handle of the wrong type
to an object returned by a method call
and not when that object is (incorrectly) used later).


\subsubsection{Type casting of object Handles}
\label{sec:type-casting}

This solution for adding type safety introduces an inconvenient:
\emph{Upcasting}.
Java types form a (singly rooted) hierarchy.
These types may be automatically \emph{upcast}
i.e.\ an object of a derived type may be used
whenever an object of a base type is required
(as in other object oriented languages).
If class \verb=Sparrow= is derived from class \verb=Bird=,
Java allows the use of a \verb=Sparrow= object
whenever a \verb=Bird= is expected.

In Java, a \verb=Sparrow= \emph{is} a \verb=Bird=.
In \ac{TTCN-3} this is not true, since the types have different structures
and the type system is not object oriented
(a Java class may also implement several \verb=interface=s).
This inconvenient is overcome by generating a \texttt{casting template}
as illustrated in Listing \ref{prog:casting-template}.
As with explicit type casting in other programming languages,
incorrect usage of the casting operator can only be detected at runtime.
Upon usage of objects after casting, at runtime,
all objects are checked to be of the correct type.

\begin{program}
\verbatimtabinput{./ttcn3/casting-template.ttcn3}
\caption{The casting template for type \texttt{java.util.Map}%
	\label{prog:casting-template}}
\end{program}

In order to allow casting from \emph{any} type to a particular type
(in the example in Listing \ref{prog:casting-template} casting from any type
to \texttt{java.util.Map} is allowed)
the \texttt{casting template} must be allowed to take
\emph{any} object handle as parameter.
The type safety just discussed must be circumvented in this particular case.
This has been accomplished with the use of a special, predefined handle type
with the same structure as all other handles but without any restriction
on the values assigned to the \verb=typeName= field.
This special type has been called \verb=ObjectType=
and its definition is shown in Listing \ref{prog:generic-object-type}.
Since no restrictions are placed on the fields
in the \verb=ObjectType= record type,
a handle for any other type may be passed
where an \verb=ObjectType= is required (e.g.\ to the casting operator).

\begin{program}
\verbatimtabinput{./ttcn3/generic-object-type.ttcn3}
\caption{The generic (unrestricted) ObjectType\label{prog:generic-object-type}}
\end{program}


\subsection{Mapping Java Object Types}
\label{sec:mapping-java-object-types}

After having examined the task of mapping types between the two languages,
a more detailed presentation follows.
The previously mentioned elements are put together
to show the structure of the code resulted after mapping a Java type.

The \emph{mapping} operation (as defined and used in this project)
treats a Java type as a translation unit.
In other words the mapping operation receives a Java type as parameter.
If several Java types need to be mapped
(as might usually be the case for a test scenario)
then the mapping is performed once for each type.

Each Java type is mapped to a \ac{TTCN-3} module.
Each generated module contains the following elements:

\begin{description}

\item[a Handle Type definition]
for \ac{TTCN-3} references to objects of the mapped Java type.
The name of this handle type can be based on the module name
or customized by the test developer.
\item[Method signatures]
for all public methods of the Java type.
These signatures are complete with return values and exception handling.
Primitive Java types (for parameters and return values)
are mapped to their \ac{TTCN-3} counterparts as detailed in
section \ref{sec:type-mapping-primitive-types}.
Object types (for parameters, return values and thrown exception types)
are mapped to \ac{TTCN-3} record types which act as handle types,
as shown in section \ref{sec:mapping-java-object-types}.
For this reason, all types used in signatures need to be defined in the module
(apart from the actual Java type for which the mapping is generated).
In order to allow the user to handle \emph{unchecked exceptions},
each method signature gets \texttt{java.lang.Throwable}
added to its list of thrown exceptions.

Signatures generated for instance methods require an extra first parameter:
the handle of the object for which the method is to be invoked.
Static methods (also called class methods) are mapped to signatures
which require no such additional parameter.

\item[Constructor signatures]
for all public constructors present in the class.
Constructor parameters and thrown exception types
are treated in the same way as for method signatures.
Constructor signatures always return the type
for which the mapping (the whole module) has been generated.
Constructors have no additional first parameter for an object handle.

\item[Accessor signatures]
are generated for all public fields found in the type
(e.g.\ class, interface, enumeration).
Since any code using this Java type is able to access public fields,
the same functionality must be available to test developers.
As a note, this functionality is needed
in order to access Java enumeration constants
which are actually \texttt{public static final} fields of their enclosing type.

A \emph{getter} signature is generated for all public fields in the class.
This signature's return type is the field type, and it throws no exceptions.

A \emph{setter} signature is generated for public fields
which are not \verb=final=.
This signature requires a parameter
(a primitive value or a handle for an object type)
representing the new value for the field.
It has no return value and throws no exceptions.

Just like method signatures, accessor signatures have an extra parameter
(the object handle) when they correspond to an instance field.
This additional first parameter is not present
for accessor signatures manipulating static fields.

\item[a Casting Template]
needed for casting to the mapped Java type.
The Java programming language allows
both explicit casting and automatic upcasting.
These features may be needed in \ac{TTCN-3} test code
and are provided by the generated template.
This issue is explained in more detail in section \ref{sec:type-casting}.

\item[a Port Type]
for procedure oriented communication
having all the above generated signatures added.
This is a convenience for the user:
instead of defining a port type and listing the needed signatures,
the auto-generated one can be used directly.
For test scenarios where many Java types are used,
it may be convenient to have all signature calls through a single port.
In such a case the test developer must define a port type
and list the signatures of interest.

\item[Handle types for all object types used]
As previously explained, \ac{TTCN-3} signatures need specific handle types
to assure type safety.
For this reason all types used in the module need to be defined at the end.
Test cases using multiple modules will not encounter errors
due to redefinitions of the same type
(e.g.\ \texttt{java.lang.Throwable}) in several modules.
\ac{TTCN-3} allows importing types with the same name from several modules
and requires fully qualified names for them (to avoid ambiguities).
The user will mostly need the types that the mapping has been generated for
(the handle for such types is at the start of the generated module)
and these have custom names
(which, being chosen by the test developer, should not cause clashes).
In the case when a type imported from several modules needs to be used,
the user may generate a short-named template
or (s)he may use the fully qualified name (\texttt{moduleName.typeName}).

\item[a module \texttt{variant} attribute]
as described in section \ref{sec:type-encoding-variant}.
Its role is to specify the Java type the mapping has been generated for.
It is inherited by all module elements (e.g.\ record types, signatures)
which don't have their own \texttt{variant} attribute defined.
It is used by the codec for finding out the type that these elements belong to.

\end{description}

\subsubsection{An example mapping}

As an example, the mapping for the Java class in Listing
\ref{prog:example-mapping-java-class.java} will be presented.
This is a simple class simulating a pocket calculator.
It has two public \verb=field=s of type \verb=double=
(for the result shown on its display and for the contents of its memory)
and several methods for performing computations,
storing the result in memory
and some convenience methods returning an integer representation
(of type \verb=long=)
of the floating point values accesible through the public fields.

\begin{program}
\verbatimtabinput{./java/example-mapping-java-class.java}
\caption{A Java class to be mapped%
	\label{prog:example-mapping-java-class.java}}
\end{program}

Listing \ref{prog:example-mapping-java-class.java}
shows only an overview of the Java class
(code for the shorter methods has been included;
comments for the longer methods
should be sufficient to make their meaning clear)
As stated in section \ref{sec:mapping-java-object-types}
a Java type is a translation unit for the language mapping operation.
An overview of the resulting mapping is shown in
Listing \ref{prog:example-mapping-java-class.ttcn3}.
The previously discussed \ac{TTCN-3} elements
(e.g.\ signatures, record types, a casting template)
are generated for all methods accessible in the Java class
and for all types used (as arguments, return values or thrown exceptions).
The listing is larger and
only some significant elements have been chosen for inclusion here.

\begin{program}
\scriptsize
\verbatimtabinput[4]{./ttcn3/example-mapping-java-class.ttcn3}
\caption{The resulted mapping%
	\label{prog:example-mapping-java-class.ttcn3}}
\end{program}

The mapping operation generates a \verb=module= for each Java type mapped.
The first statement always \verb=import=s the \verb=JAVAAUX= module.
This is a utility module needed by all generated code
(please consult Appendix \ref{sec:javaaux-appendix}
for the complete module listing).
A (type safe) handle record type for the imported Java class is then defined.
The next part consists of \verb=signature= definitions.
These include all signatures available in the type
(i.e.\ those defined by the type,
by all its superclasses and by all implemented interfaces).
Signatures for all public constructors are then defined
followed by accessor signatures for all public fields.

The casting template is also generated.
As previously stated, \ac{TTCN-3} is not an object oriented language
and type equivalence is asserted based on type structure.
When testing Java code, it may be the case that an object of a different type
may need to be supplied where a \verb=Calculator= is expected.
If that type \emph{is} a \verb=Calculator= (i.e.\ it is a subclass)
then the substition is valid in Java.
In order to make this possible in \ac{TTCN-3} the handle must
be manipulated (and the \verb=typeName= \verb=charstring= field changed).
This is accomplished by using the provided \emph{casting template}.

In order to call signatures or send messages in \ac{TTCN-3},
component ports must be used.
To call a \verb=signature=, a \verb=port= of type \verb=procedure=
must be used, and the signature must be listed in the port definition.
Test developers may define port types and list all signatures of interest
(possibly from several generated or user defined modules).
However, as a convenience,
each generated module contains a port type definition
for use with all the signatures from that module.

The last part of the generated module contains
\ac{TTCN-3} record type definitions for all Java types used in the module
(as parameter or return value types).
As an example, the inner class \verb=Operations= has been included in
Listing \ref{prog:example-mapping-java-class.ttcn3}.
The \verb=typeName= field is restricted to the binary name of the Java type,
and each record has an \emph{encoding variant} with the binary name,
needed by the codec.


\section{Name mangling}
\label{sec:name-mangling}

A \ac{TTCN-3} module must be generated for each Java type used.
Afterwards, the generated elements need to be mapped back to
the Java methods and (exception) types.
In order for this task to be successfully performed,
a name translation scheme needs to be defined.
Java names have several properties:

\begin{itemize}
\item User defined names are formed by letters, numbers
and the underscore character (\verb|_|)
\item Type names should be contained within a package
and may be defined inside other types
so that a fully qualified type name may look like:\\
\texttt{package.name.OuterType.InnerType}\\
in the source code, with the binary name:\\
\texttt{package.name.OuterType\$InnerType}
\item Types may be anonymous
(in which case they receive a compiler-generated name ---
which may look like \verb=package.OuterType$1=)
\item They are case sensitive
\item Binary names for \verb=Array= types consist of
one or more `\verb=[=' characters specifying the depth of the array nesting
followed by the encoding of the element type name which is defined in
Table \ref{tab:array-binary-names}
according to \citep{website:java-api}.\footnote{%
\url{http://java.sun.com/javase/6/docs/api/java/lang/Class.html\#getName()}}
\end{itemize}

\begin{table}
\centering
\begin{tabular}{|l || c|}
	\hline
	\emph{Element Type}	& \emph{Encoding} \\
	\hline
	boolean			& Z \\
	byte			& B \\
	char			& C \\
	class or interface	& L\emph{classname}; \\
	double			& D \\
	float			& F \\
	int			& I \\
	long			& J \\
	short			& S \\
	\hline
\end{tabular}
\caption{Java binary names for Array types\label{tab:array-binary-names}}
\end{table}

But crucial for the design of the naming scheme is the fact that:

\begin{itemize}
\item Java methods may be overloaded
(i.e.\ several methods inside the same Java type may have the same name
and different signatures)
\end{itemize}

Names used in \ac{TTCN-3} are also formed using
letters, numbers and the underscore character (\verb=_=)
and they are case sensitive.
\ac{TTCN-3} does not allow overloaded signatures.
Thus the challenges for the naming scheme are:

\begin{itemize}
\item To account for the special characters in Java type names
(\verb=.= and \verb=$=)
\item To guarantee that overloaded Java methods
map to different \ac{TTCN-3} signatures
and can be used without causing any conflicts
\end{itemize}

As previously mentioned in Section \ref{sec:project-objectives}
another goal is to have a \emph{stable} naming scheme,
i.e.\ one does not change name mappings when the Java types change
(e.g.\ by adding an overloaded method to the type and mapping it again).


\subsection{Treating special characters}
\label{sec:special-characters}

The special characters which need to be changed
are the ones found in Java binary names
but not allowed in \ac{TTCN-3} identifiers.
These are the dot (\verb=.=), the dollar sign (\verb=$=)
and the array nesting indicator (\verb=[=).

These characters cannot simply be discarded because
at runtime the Codec needs to perform a reverse name mapping
and find out the Java type name
(another reason is that several names may clash
by simply discarding the special characters).
This leads to the conclusion that information about the special characters
needs to be kept in the mapped \ac{TTCN-3} name
and the translation scheme needs to represent this information
using legal characters.

The dot (\verb=.=) and the dollar (\verb=$=) characters
will be discussed first.
Since the array depth character (\verb=[=)
may only appear at the start of a binary type name
a different (and simpler) translation scheme will be used for it.


\subsubsection{Replacing with a valid string}

The information about the special characters will be kept in the mapped names
by replacing special characters with strings
which may be part of \ac{TTCN-3} identifiers.
If the replacement strings are called \emph{<rDot>} and \emph{<rDollar>}
then the Java name\\
\verb=package.Outer$Inner=\\
will be mapped to\\
\verb=package=\emph{<rDot>}\verb=Outer=\emph{<rDollar>}\verb=Inner=

The strings chosen for replacement are shown in
Table \ref{tab:replacement-strings}.
They have been chosen because
they are expected to rarely be part of user defined names
(and the one most commonly used, ``\verb=_0_='' for replacing ``\verb=.='',
bears a graphical resemblance to a dot,
so that human examination of automatically generated types is intuitive).

\begin{table}[htb]
\centering
\begin{tabular}{|r || l|}
	\hline
	\emph{Special character} & \emph{Replacement String} \\
	\hline
	\verb=.= & \verb=_0_= \\
	\verb=$= & \verb=_1_= \\
	\hline
\end{tabular}
\caption{Replacement strings for special characters%
	\label{tab:replacement-strings}}
\end{table}


\subsubsection{Escaping the replacement string}

Replacement strings for special characters have been chosen
in an attempt to avoid any strings that may be part of user defined type names.
While this assumption is expected to hold for the majority of use cases,
it is of course legal to define Java type names which include
these particular character strings (``\verb=_0_='' and ``\verb=_1_='').

For example the type name \verb=com.My_0_Type=
will be mapped to \verb=com_0_My_0_Type= in the automatically generated
\ac{TTCN-3} module.
At runtime, the codec will encounter this mapped name
and will perform a reverse mapping to obtain the original Java name.
The reverse mapping will provide \verb=com.My.Type= as result,
which is incorrect.

The previously presented mapping scheme may easily be altered
so that it will be able to also handle this additional case.
The solution is to \emph{escape} the special strings
if they are found in the user defined names:
Every string is \emph{doubled} (written twice)
in the resulting mapped \ac{TTCN-3} name.

This improved algorithm requires a different reverse mapping approach.
The replacement strings introduced earlier (e.g.\ ``\verb=_0_='')
must not be reverse mapped to the original character (e.g.\ ``\verb=.='')
in a \emph{greedy} fashion.
Instead, the reverse mapping operation needs to \emph{look ahead}
when \emph{parsing} the generated \ac{TTCN-3} name.
When a replacement string is found the algorithm will look ahead
to check if this occurrence if immediately followed by another one.
If so, both occurrences are \emph{consumed}
and mapped to a single occurrence of the respective string.
Otherwise the replacement string is reverse mapped to
the original special character.
The pseudocode is shown in Listing \ref{prog:unmangle-escaped-strings}.

\begin{program}
\verbatimtabinput{./pseudocode/unmangle-escaped-strings}
\caption{Pseudocode for unmangling special characters %
	and checking for escaped replacement strings%
	\label{prog:unmangle-escaped-strings}}
\end{program}

In order to avoid any corner-case problems while reverse-mapping names,
the direct mapping escapes first the \emph{dot replacement string} (\verb=_0_=)
then the \emph{dollar replacement string} (\verb=_1_=)
and the reverse mapping un-escapes them in reverse order.


\subsubsection{Requirements for avoiding ambiguity}
\label{sec:avoiding-ambiguity}

The previously presented escaping mechanism introduces a subtle issue
which will be presented by using an example.
Table \ref{tab:mapping-ambiguity} shows two Java names which correspond
to the same \ac{TTCN-3} generated name.
We can easily check that both will determine the same mapped name
to be automatically generated ---
which is not a problem during the direct mapping,
but becomes an important issue when reverse mapping
(in order to find out the original Java name).

\begin{table}[htb]
\centering
\begin{tabular}{|c || c|}
	\hline
	\emph{\ac{TTCN-3} generated name} & \emph{Java name} \\
	\hline
	\multirow{2}{*}{\texttt{package\_0\_\_0\_\_0\_Type}} &
		\verb=package_0_.Type= \\
	& \verb=package._0_Type= \\
	\hline
\end{tabular}
\caption{Reverse mapping ambiguity\label{tab:mapping-ambiguity}}
\end{table}

In order to overcome this issue a restriction must be imposed
on user defined names to be mapped.
Either names aren't allowed to start with a replacement string,
or they are not allowed to end with one.
When reverse mapping a name with an uneven number of
consecutive replacement strings
the algorithm will replace either the last or the first occurrence
with the escaped character.
Since conventions used in the Java Programming language state that:
\begin{itemize}
\item Package names should be reverse domain names owned by the programmer
and should be in lowercase
\item Class names should start with a capital letter
\end{itemize}
the restriction imposed for mapped names is that:
\emph{User defined Java names must not start with one of the strings
replacing special characters (``\texttt{\_0\_}'' and ``\texttt{\_1\_}'')}.
However user defined names may end with these strings.


\subsubsection{Mapping the array depth character}

The array depth specifier (\verb=[=) may only appear at the start
of a binary Java name.
All names in Java and \ac{TTCN-3} must not start with a digit
(this is also true for most other programming languages).
This allows for the following approach to mapping the name of an Array type:
\begin{itemize}
\item Start with a prefix
\item Continue with a number specifying the array depth
\item Use the mapped name of the element type
(not the binary name of the array type
which starts with ``\verb=[='' and ends with
the encoding of the element type name from Table \ref{tab:array-binary-names})
\end{itemize}

Listing \ref{prog:mapping-array-depth} shows an example of how
array types are mapped to \ac{TTCN-3} types.
By starting with a number to specify the array depth
and using the element type name
(not the encoding from Table \ref{tab:array-binary-names})
the generated names are more user friendly.
The double underscore separator string (\verb=__=) ---
appended at the end of the generated type name for a Java array ---
will be explained in section \ref{sec:description-naming-conventions}.

\begin{program}
\verbatimtabinput{./ttcn3/mapping-array-depth.ttcn3}
\caption{Mapping of the \texttt{String[][]} Java Array type%
	\label{prog:mapping-array-depth}}
\end{program}


\subsection{Mangling (overloaded) method names}

The task of mapping Java types to \ac{TTCN-3} types
consists mainly of generating appropriate data structures
and naming them similar to the original elements from Java.
For the most part, the chosen names are the original ones
with slight changes in order to avoid special characters
not allowed in \ac{TTCN-3}
(as detailed in section \ref{sec:special-characters}).

However mapping \emph{method} names from Java to
\ac{TTCN-3} \emph{signature}s is not as trivial.
The \ac{TTCN-3} language does not allow \emph{overloaded signatures},
but Java does.
Overloaded methods belong to the same type (e.g.\ class, interface)
and have the same name but different argument lists.
For these reasons the mangling scheme
needs to generate different names for such methods
and must also be able to perform the reverse operation.

Several options come to mind,
each with advantages and disadvantages of its own.
An overview of such methods is presented below
and the chosen mangling scheme is detailed.


\subsubsection{Avoiding unnecessary manglings}
One option which comes to mind is to simply ignore the problem
for non-overloaded functions, and when overloaded functions are encountered
to use one of the approaches described below.
A solution for overloaded methods would still be needed of course,
but the goal would be to use friendlier (non-mangled) names
for methods which are not overloaded.
An important point to make is that this is not an option,
since the generated mangling would not be \emph{stable}
as required in section \ref{sec:project-objectives}.

The mangling scheme will therefore be needed even for non-overloaded methods,
as this is the only way to guarantee that a mangled name for a method
will not change even if the type is extended
(e.g.\ by overloading that method).


\subsubsection{Counting the overloads}
A first mangling attempt is to simply \emph{count} the signatures
in the following manner:
\begin{itemize}
\item Sort the methods using some criteria
(e.g.\ lexicographically according to the argument list,
by parameter count and then lexicographically by argument types)
\item Name the signatures by appending `\verb=_1=', `\verb=_2=' etc.
at the end of the name.
\end{itemize}
The problem with this approach is that, no matter the sorting criteria chosen,
a Java type may be extended with another overloaded signature
which will end up `in the middle' of the previous sequence,
thus breaking the stability.
What is more, this method does not allow a user to know
the mangled name for a method
without knowing the final version of the type's interface.


\subsubsection{Appending parameter type names to the signature name}
This idea is also used in the Java to \ac{IDL} mapping for \ac{CORBA}
as described in \citep[pg~10]{website:java2idl-mapping}.
That mapping keeps the original names for methods without overloads
and switches to this mangling scheme only for overloaded methods.
As just shown above, a stable mangling scheme cannot leave
any method names unmangled.
With the small change of mangling everything,
this is the naming scheme chosen for this project.

The mangling scheme will concatenate the following elements
to form a signature name:
\begin{itemize}
\item The Java method name
\item The parameter count (optional)
\item The (mapped) type names of the method parameters
(as defined in section \ref{sec:type-mapping}).
\end{itemize}

The above mentioned elements are concatenated using a \emph{separator string}
to form the mangled signature name.
At runtime, when the signature name needs to be unmangled
in order to obtain the original Java name,
the name is `broken' into its elements
(by splitting it around occurrences of the separator string).
The chosen separator string in this case is ``\verb+__+''.

The ambiguity problem described in section \ref{sec:avoiding-ambiguity}
is also present here.
If the concatenated elements mentioned above contain the separator string,
it needs to be escaped.
The unmangling algorithm will similarly need to look ahead
and replace \emph{doubled} separators with a single occurrence,
and split the string where single separators are encountered.
The same problem discussed earlier arises
when an uneven number of separators occur consecutively.
The same convention will be used:
\emph{User defined names may not begin with the separator string},
but they may end with it.


\subsubsection{Method mapping example}
As an example,
the mapping for the \verb=equals()= method in class \verb=java.lang.Object=
is shown in Listing \ref{prog:signature-mapping-example}.
This method, being part of the \verb=Object= interface ---
the \emph{root} of the Java type system ---
is generated when mapping any Java class.
Because it is an instance method,
the \ac{TTCN-3} signature takes an extra first parameter:
the handle for a particular object instance.
The type of this object corresponds to the Java type for which the mapping
has been generated.
For this example, the \verb=Object= class has been mapped.
The following short analysis will explain
the parts of the signature definition:

\begin{program}
\verbatimtabinput{./ttcn3/signature-equals.ttcn3}
\caption{Method mapping example for \texttt{java.lang.Object.equals()}%
	\label{prog:signature-mapping-example}}
\end{program}

\begin{description}
\item[The signature name] consists of
the \emph{method name} (\verb=equals=)
followed by the \emph{parameter count} (\verb=1=)
and then the mangled name of each parameter type.
The different parts of the method signature are concatenated using
the separator string ``\verb=__=''.
\item[The object handle] is an extra first parameter
of the type for which the module (mapping) has been generated.
In this case it is \verb=Object=.
\item[The signature parameters]
are listed in the same order as for the Java method
and are named p1, p2 etc.
The reflection \ac{API} does not provide information
about method parameter names, only about their types.
\item[The return value] has a \ac{TTCN-3} type
as described in section \ref{sec:type-mapping}.
Primitive values are mapped directly between the two languages,
and record types are created for Java \verb=Object= types.
This method has a \verb=boolean= return value.
\item[Thrown exceptions] listed for each signature
include all exception types declared to be thrown by the Java method.
These may be both \emph{checked} and \emph{unchecked}.
As stated in \citep{website:java-tutorial-exceptions}\footnote{\raggedright%
	\url{http://java.sun.com/docs/books/tutorial/%
	essential/exceptions/catchOrDeclare.html}}
the \emph{Catch or Specify requirement} states that
checked exceptions must be caught or declared to be thrown.
However unchecked exceptions not declared in the method definition
may still be thrown, and test developers may need to catch those as well.
For this reason, the \verb=Throwable= type is always added
to the \emph{set} of exception types thrown by any method.
\end{description}


\subsection{Naming conventions for generated identifiers}
\label{sec:description-naming-conventions}

The previous sections have shown how names are mapped between the two languages
(for types, methods and generated accessor signatures).
The two main tasks are:
\begin{itemize}
\item Deciding which data structures to use for each mapped element
\item Mangling the generated name,
so that it contains only legal \ac{TTCN-3} characters
and the language's lack of support for overloaded signatures is overcome
\end{itemize}

However another important issue is
\emph{avoiding name-clashes between user-defined and generated identifiers},
so the approach of using prefixes (and suffixes) has been used.


\subsubsection{Marking generated identifiers}

A generated module contains signatures
(for methods, constructors and accessors for public fields)
and type definitions.
The methods from a Java type will be mapped to signatures with different names.
But adding records for all the types used
and signatures for constructors and accessors (which have no predefined name)
may lead to name clashes.
For this reason the signatures are normally generated,
but all other names are marked to indicate that
they do not represent Java method mappings.

Because the separator string ``\verb+__+'' used to concatenate
elements of a generated name
is always escaped within those elements,
it can be added at the end of generated names
to mark the fact that they have been internally generated.

Such \emph{problematic} names are those generated for
\emph{constructor signatures}, \emph{accessor signatures}
and \emph{record types}.
By marking them as internally generated
(meaning they are not direct mappings of Java methods)
they will not clash with method manglings.
The mark used is distinctive because it is un-escaped and found at the end.
As an example, the mangled name `\verb=someName='
becomes `\verb=someName__=' by using this mechanism.


\subsubsection{Prefixes for generated types}

In order to make the generated code more readable,
prefixes are used when generating the mapping.
These indicate the type of the element
by simply reading the name (without looking at its structure or definition).
A list of used prefixes follows:

\begin{description}

\item[Types] get the ``\verb=type_='' prefix and the ``\verb+__+'' suffix.
After mangling, the type names become less easy to read
and the prefix may aid in this respect.
As an example, the generated type name for `\verb=java.lang.String=' is:\\
`\verb=type_java_0_lang_0_String__='.

\item[Method signatures] get no prefixes or suffixes.
This will normally be readable as they start with the original name
and include the parameter count and the concatenated type names.
These can not clash with another generated name because they cannot end
with an unescaped separator.
The long generated names are not a problem since the test developer may easily
define a \ac{TTCN-3} template with a short name.

\item[Constructor signatures] get generated
in the same manner as a normal method signature
called ``\verb=construct='' and get a suffix (\verb+__+).
The suffix is necessary to ensure against a name clash
with another method called ``construct'', which may be present
in the type being mapped.
As an example, the no-arg constructor for ``\verb=java.lang.String=''
is mapped to the following signature:\\
\verb=signature construct__0__() return type_java_0_lang_0_String__=\\
\verb=exception (type_java_0_lang_0_Throwable__);=

\item[Accessor signatures] get generated
in the same manner as a normal method signature called
``\verb=get_fieldName='' (for getters)
and ``\verb=set_fieldName='' (for setter signatures)
and get the ``\verb=__='' suffix at the end to mark them as
\emph{internally generated}.
Listing \ref{prog:accessor-example} shows the accessor signatures
generated for the \verb=memory= field of the \verb=Calculator= class in
Listing \ref{prog:example-mapping-java-class.java}.

\begin{program}
\verbatimtabinput{./ttcn3/accessor-example.ttcn3}
\caption{Example accessor signatures for the \texttt{Calculator.memory} field%
	\label{prog:accessor-example}}
\end{program}

The getter returns the field value.
The setter signature has no return value
but an extra parameter for the new field value.
In class \verb=Calculator=, \verb=memory= is an instance (non-static) field,
so both methods take a first parameter for the object handle.

\item[Port types] have the prefix ``\verb+port_+'' followed by
the mapped type name and end with the ``\verb+__+'' suffix.
Port types are not needed but are generated for convenience.
The user may start calling signatures through this provided port type
or may define his/her own, listing all signatures of interest
(from several modules if needed).

The port type defined when mapping the \verb=Calculator= class
from Listing \ref{prog:example-mapping-java-class.java}
is shown in Listing \ref{prog:port-type-example}.
It is a procedure oriented port type and all signatures defined in the module
are listed for usage in its definition.
Only a few signatures have been included
in Listing \ref{prog:port-type-example} to serve as an example.

\begin{program}
\verbatimtabinput{./ttcn3/port-type-example.ttcn3}
\caption{The generated port type for the \texttt{Calculator} class%
	\label{prog:port-type-example}}
\end{program}

\end{description}



\section{Storage Codec}
\label{sec:storage-codec}

As described in section \ref{sec:test-system-arch},
the task of the codec is to encode sent messages
from their \ac{TTCN-3} representation
to the real world bits and bytes which need to be sent.
The codec also decodes received messages from their real world representation
to the data structures used in \ac{TTCN-3}.

In this project the \ac{SUT} is compiled Java code.
The unit tests call signatures and supply parameters for them,
test return values and handle thrown exceptions.
The information sent to the \ac{SUT} in a \emph{procedure oriented} scenario
is made up of signatures (which are called) and their respective parameters.
When testing a remote system (such as a \ac{CORBA} or \ac{RPC} system)
one might expect this information to be encoded in network packets
which get sent as messages to the remote \ac{SUT}.

The particular case of testing local Java bytecode
does not need to send any messages to a remote system
(though the test system itself may be split into several components
which communicate among themselves through messages,
as described in section \ref{sec:future-work}).
The tasks that need to be performed are:
\begin{itemize}
\item identifying the Java types and methods used
(by unmangling the \ac{TTCN-3} names)
\item invoking the requested methods
\item enqueueing the method return values or thrown exceptions
to be processed on the respective port (used for the procedure call)
\end{itemize}

The codec is responsible for encoding parameter types
from the \ac{TTCN-3} representation to a binary one
(as required by the standard interfaces for the codec)
and decode return values and exception types from a binary representation
to the corresponding \ac{TTCN-3} format.
The encoded parameters are communicated between the \emph{codec}
and the \emph{port plugin}
(which for this project resides on the same machine).

If the Java bytecode to be tested would reside on a different machine
(as suggested in section \ref{sec:future-work}).
the system might include a component which would run on that machine.
The binary messages between the components
would then be sent via the network.

Because objects are stateful they need to be \emph{kept alive} between calls.
As mentioned in section \ref{sec:type-safe-object-handles}
type-safe object handles prevent some programmer errors
but type checks also need to be performed at runtime to ensure correct usage
(as type names may be altered by hand or by using the \emph{casting template}
previously mentioned in section \ref{sec:type-casting}).
These checks are made both by the codec and the port plugin.
\begin{itemize}
\item when the codec is asked to
decode a returned value or a thrown exception type
(to make sure the actual value corresponds to the \emph{decoding hypothesis})
\item when the port plugin performs the invocation (to make sure
the handles point to objects of the correct type for parameters)
\end{itemize}

In order to keep the objects alive they will be stored
in an \emph{object registry}.
Since it is the codec's task to encode between the \ac{TTCN-3} representation
and some binary representation of object handles
the codec will also be the component responsible for
maintaining the object registry.

The previous paragraphs have covered \emph{what} tasks need to be performed
and \emph{why} some have been assigned to the codec.
The stage has now been set for presenting the codec's components
and describing the mechanisms used to achieve their functionality.


\subsection{Standard Interfaces}

Since the \ac{TTCN-3} language and test system architecture are standardized,
the codec needs to implement the standard \ac{ETSI} interface\\
\verb=org.etsi.ttcn.tci.TciCDProvided=\\
with its methods of interest presented in
Listing \ref{prog:codec-standard-interf}.
The semantics of the \verb=encode= and \verb=decode= methods
is to convert between \emph{binary} and \ac{TTCN-3} representations of data.
For this project byte representation of keys for the object registry
are returned.

\begin{program}
\verbatimtabinput{./java/codec-standard-interface.java}
\caption{Standard codec interface\label{prog:codec-standard-interf}}
\end{program}

The \ac{TTCN-3} \verb=Value= interface is used to describe
data types and data structures used in the language.
It has methods for retrieving the
\verb=Encoding=, \verb=EncodingVariant= and the \verb=Type=
(which allows inspection of the particular type of the data manipulated).
Several sub-interfaces are defined, both for primitive and structured types,
some of which are listed in Table \ref{tab:value-subinterfaces}.
Most of these types have \emph{accessor} (\verb=get= and \verb=set=) methods
for the data contained;
the \verb=RecordValue= interface has accessor methods which take
a \emph{field name} as argument.
The \verb=RecordValue= interface is used
for manipulating object handles in the codec
(which are represented as record structures
with a field for the object's \emph{key} in the registry
and another field containing a charstring representation
of the Java type name).

\begin{table}
\centering
\begin{tabular}{|l|}
\hline
BooleanValue \\
UniversalCharstringValue \\
IntegerValue \\
FloatValue \\
RecordOfValue \\
\hline
\end{tabular}
\caption{Sub-interfaces of the TTCN-3 \texttt{Value} type%
	\label{tab:value-subinterfaces}}
\end{table}

The \verb=TriMessage= interface allows the representation of a byte array.
This is the encoded message created by the codec for the port plugin.
The port plugin is expected to send it to the \ac{SUT}.
In a traditional scenario this would be sent as a network message.
For unit testing the \verb=TriMessage=s received by the port plugin
are byte representations of keys for registry objects.
These are the parameters for the invocations
and the particular object instance to invoke the method on.


\subsection{Object registry}

The \emph{object registry} is a \emph{Map}
assigning keys to the contained objects.
It is needed because of the way objects are referenced from \ac{TTCN-3} code;
test developers use a record type with two fields:
\begin{itemize}
\item A \emph{handle} (of type \verb=address=) used to identify the object
\item A \emph{field} of type \verb=charstring= used for type safety
\end{itemize}

Since the \emph{handle} is being used to reference an existing object,
the approach chosen is to keep the objects in a Map
and store the key in the handle.
The codec will pass this key to the port plugin when a signature is called.
For object parameters, the key stored in the handle is passed.
Parameters of primitive types are wrapped in an \verb=Object=
and the key for that object is passed.

When a method call has a return value,
the returned object is put in the registry
(after wrapping it if it is of a primitive type)
and the key for it is enqueued as a procedure reply.
When the testcase code tries to store this returned value
in a variable of a proper record type,
the codec is asked to \emph{decode} the returned value.
The codec then checks the object in the registry
against the \emph{encoding variant} of the type used for the return variable.
A successful attempt at decoding is signaled by returning
a proper \ac{TTCN-3} \verb=Value=
while an unsuccessful one is signaled by returning \verb=null=.


\subsubsection{The reversed Map}

Objects are added to the registry
when the signatures called have a return value.
The port plugin asks the storage codec to put a new object in the registry
and then enqueues the key for this object as return value.
It is desirable to only store each object once in the registry.
Since the registry only maps \emph{keys} to \emph{objects},
it is not well suited for checking if an object is already present.

For this reason a second map has been introduced,
called a \emph{reversed registry}.
It maps the objects to their keys in the other registry.
After being added to the first registry,
a new object and its key must also be added to the reversed registry.
When the programmer requests the deleting of an object from the registry,
it must be removed from both.

When adding a new object, the reversed registry is used
to check if the object is already present.
If so, its existing key is returned by the \verb=add()= operation,
otherwise a new key is generated and the object (together with its key)
is added to both registries.


\subsubsection{Implementation}

\paragraph{Encoding and Decoding}
translate between the \ac{TTCN-3} representation of data
(the \verb=Value= interface)
and a binary representation (the \verb=TriMessage= type).
The \verb=TriMessage= sent to the port plugin
(and also enqueued by the port plugin as return value)
is always the representation of a registry key.
The object pointed to is the actual Java object
or a wrapper for a primitive type.

The decoding operation also performs type checking
and if the object type and the handle type are incompatible,
it returns \verb=null=
to signal that decoding cannot be performed with this hypothesis.
When encoding \verb=integer= types, the \verb=java.lang.Integer= type
is always used as wrapper
(because \ac{TTCN-3} has a single \verb=IntegerValue= interface
used for representing unbounded  integers).
The port plugin will retrieve the \verb=Integer= from the registry
and perform the appropriate conversion to a primitive type
before calling a method.
Values returned by the port plugin are automatically wrapped
in the appropriate wrapper type (e.g.\ \verb=Short=, \verb=Byte=)
and the codec will check for this before downcasting
the object obtained from the registry and returning an \verb=IntegerValue=.

\paragraph{Registries}
The first registry maps \verb=String=s to \verb=Object=s
while the reverse registry performs the mapping the other way around,
using the \verb=Object=s as keys.
As previously stated, this requires each object to be stored only once
by the codec (since it must be used as a key in one of the registries).
Obviously, this is also desirable in order to avoid wasting memory.

The keys for the registry have been chosen of type \verb=String=.
They are actually string representations of integer numbers.
An internal counter is used for generating the keys,
and it is incremented whenever a new object is stored.
Keys are assigned consecutively (\verb="0"=, \verb="1"=, \verb="2"= etc.).

When objects are deleted from the registry their keys are not reused.
The only operation changing the internal counter is the adding of new objects.

Some particular challenges need to be overcome
when implementing these two registries.
Keys for a map are considered equal (i.e.\ they are the same key)
based on the \verb=boolean= result of the \verb=equals= method.
This is a crucial problem for storing user objects
and using them as keys.
Storing the distinct value \verb=null= may also be desirable.
When the codec tries to retrieve the object associated with a key,
\verb=java.util.Map= uses \verb=null=
to indicate a missing mapping for that key.
The codec must be able to distinguish this situation from a key mapped to the 
\verb=null= object.

These issues have been addressed by a slight change in design:
The objects are not directly stored in the maps.
They are wrapped inside objects of the type \verb=ObjectHolder=
presented in Listing \ref{prog:ObjectHolder}.
\verb=ObjectHolder= is a private inner type
not exposed outside the \verb=JavaCodec= class.
It is only used internally.
The \verb=encode= and \verb=decode= methods
\emph{always} use the stored \verb=Objects= as parameters or return values.
This holder type overrides the \verb=equals= method such that
it returns \verb=true= when two holders hold the same object
and \verb=false= otherwise.

Another important point to make (regarding implementation) is
the \emph{Contract for the \texttt{hashCode} method} in Java
which states that \emph{equal objects must have equal hashcodes},
as detailed in \citep{website:equals-hashcode}
and \citep{website:java-api}.\footnote{\raggedright%
	\url{http://java.sun.com/javase/6/docs/api/%
	java/lang/Object.html\#equals(java.lang.Object)}}
Thus the \verb=ObjectHolder= class also needs to override
the \verb=hashCode= method, and does so
by returning the hashCode of the encapsulated \verb=Object=.
Fulfilling this contract is especially important
because the \verb=ObjectHolder=s are used as keys in the reversed registry.
All implementation details are found in Listing \ref{prog:ObjectHolder}.

\begin{program}
\verbatimtabinput{./java/ObjectHolder.java}
\caption{ObjectHolder used for the registries\label{prog:ObjectHolder}}
\end{program}


\section{Port Plugin}

The \emph{Port Plugin} handles signature calls
performed through procedure oriented ports used in a testcase.
The \ac{TTCN-3} values get encoded by the Codec as \verb=TriMessage=s,
and these messages are passed as parameters
to a \ac{TRI} \verb=call= operation, which is carried out by the port plugin.

The port plugin obtains the stored objects from the codec
(corresponding to the keys of type \verb=TriMessage=)
and uses the \emph{signature name} and its \emph{variant} attribute
to obtain the method signature and the type it belongs to.
It then carries out the invocation
and enqueues the return value or the thrown exception.
If a class or method is not found, or if incorrect arguments are supplied,
an appropriate error is reported by the port plugin.


\subsection{Performing method, constructor and accessor calls}

As discussed in section \ref{sec:description-naming-conventions}
certain conventions (of adding prefixes and suffixes) have been used
in order to protect against name clashes between
mangled method names and other names generated by the mapping process
(e.g.\ type names, accessor signatures, constructors, port type names).
As explained there, the presence of an unescaped separator (\verb=__=)
at the end of a name guarantees that it has been generated
and is not a mangled signature name.
Such generated identifiers of interest for the port plugin are:
\begin{itemize}
\item constructors
\item accessors
\end{itemize}
These, together with method calls, are handled by the plugin
in the manner described below.
In all three cases the \emph{variant} attribute
is used to determine the type that the signature belongs to.


\subsubsection{Method calls}

The signature name is unmangled in order to find out
the method name and the types of its parameters.
The port plugin then asks the codec for the stored objects
corresponding to the keys received as \verb=TriMessage=s.
If the method is not static,
the first such key indicates the object on which the invocation is performed.

If the class and method are found, and the parameters are of the correct type,
a separate \emph{Thread} is started to perform the invocation
and the signature call is marked as successful.
Otherwise the Java method call would not be performed
and an appropriate error would be returned for the signature call.

The separate thread carries out the invocation
(by using the \emph{Reflection} \ac{API} provided by the Java library)
and enqueues a key for the returned value or for a thrown exception
to the respective queue on the procedure oriented port.
This is performed by storing the particular returned object in the codec
and enqueuing a \verb=TriMessage= representing the registry key for it.


\subsubsection{Constructor calls}

Constructor calls are carried out in a manner similar to method calls.
After unmangling the constructor signature name,
retrieving the parameters stored in the codec and checking their types,
a new thread is launched
and the port plugin reports that the signature invocation has been successful.
This separate thread performs the actual constructor call
and enqueues a key for the newly created object or for the thrown exception.


\subsubsection{Accessor calls}

Accessor methods also get unmangled to find out the name of the public field
which is to be read from or written to.
After retrieving the stored object (for non static fields)
two slightly different courses of action occur depending on
the type of accessor signature called.
\begin{description}
\item[Getter] methods start a new thread to read the field value
and enqueue the result.
\item[Setter] methods retrieve the stored object containing the new value
for the field and check its type.
Then they start a separate thread
to perform the modification of the field value.
This thread enqueues a \emph{dummy} return value
to signal that the operation has been successfully completed
(similar to a method with a \verb=void= return type).
Testcase code must wait for a \verb=getreply= operation to complete
for the invoked signature
(either a method returning \verb=void= or a setter for a public field)
if the respective operation must be completed
before continuing to execute \ac{TTCN-3} code.
\end{description}


\subsection{Provided meta signatures}

Besides the mapped signatures for the types of interest,
a test developer may need some additional functionality.
This is needed because of the way in which objects (i.e.\ non-primitive types)
may be introduced in Java,
and how they can be created from within \ac{TTCN-3} code.

In both languages values of primitive types (e.g.\ numeric and boolean values)
may be introduced directly using literal values in the source code.
From what has been presented so far,
in order to obtain a handle for a Java object in \ac{TTCN-3} code
the programmer needs to call a \emph{signature} which returns an object
(this may represent a \emph{method}, a \emph{constructor}
or a getter for a \emph{public field}).
In order to obtain the first object, the programmer
may only use primitive types as parameters for this call (if it requires any).

The Java language allows several ways
of introducing objects directly as literals in the source code:
\begin{itemize}
\item Using the keyword \verb=null=
\item Using a \verb=String= literal (e.g.\ \verb="Hello World!"=)
\item \emph{Arrays} are true Java \verb=Object=s
and can be created in several ways
(e.g.\ using \emph{aggregate initialization})%
	\footnote{\texttt{int[] array = \{2, 3, 5, 7, 11\};}}
\end{itemize}

To address this (and other functionality)
a few utility signatures have been predefined.
The meta signatures provided to test developers are shown
in Listing \ref{prog:meta-signatures}.

\begin{program}
\verbatimtabinput{./ttcn3/meta-signatures.ttcn3}
\caption{Meta signatures in the JAVAAUX module\label{prog:meta-signatures}}
\end{program}

\paragraph{\texttt{null}}
Programmers may need to obtain a reference to the Java \verb=null=.
For this reason a meta-signature \verb=makeNull= has been provided.
It is a \emph{meta signature}, i.e.\ it is not mapped to a real method
but is intercepted by the port plugin.
The meta-signatures are defined in the built-in JAVAAUX module.
They will not clash with other mapped signatures
because they are in a separate module.
They are distinguished by the port plugin because of their particular
\emph{encoding variant}\\
\verb=com.testingtech.javaplugin.javaaux=\\
and are then intercepted.

\paragraph{\texttt{String}} objects may be created
by writing a string literal in Java.
The meta signature \verb=makeString= takes a \texttt{universal charstring}
argument and creates a new \verb=String=, returning the handle to it.

\paragraph{Arrays} are truly Java \verb=Object=s.
For this reason, whenever a signature takes an Array parameter
or returns an Array,
the programmer must be allowed to treat it as an \verb=Object=.
Since there is no overloading in \ac{TTCN-3},
Java arrays are treated as \verb=Object=s
and one way of manipulating them is by using methods in the
\verb=java.util.Arrays= class.

\paragraph{Java ``\texttt{==}'' operator}
Programmers may need to check for reference equality
using the Java operator ``\verb|==|''.
The meta signature \verb=eq= has been provided for this purpose.
It takes two object handles as parameters
and has a \verb=boolean= return value.

\paragraph{\texttt{deleteObject}}
is a meta-signature used for removing objects from the registry
kept by the storage codec.
Because there is no way of knowing
when the programmer can no longer address objects
using \ac{TTCN-3} variables (handles)
the objects cannot be automatically deleted from the registry.
If memory consumption becomes an issue,
programmers may use this meta-signature.

\paragraph{\texttt{clearRegistry}}
is a meta-signature which may be used by test developers
when no objects in the registry are of interest any more
(for instance at the end of a long testcase).
This allows memory to be recovered and is more efficient
than deleting all objects one by one (the whole map is discarded
instead of searching for and removing objects individually).


\begin{comment}
\begin{verbatimtab}
General overview
	TODO: add workflow diagram
Type mapping
	Primitive types
	Using Java Objects from TTCN-3 code
	Mapping Java Object Types
		What output should look like

Name mangling
	Escaping special characters
	Avoiding name-clashes for overloaded methods
	Avoiding name-clashes between generated and user-defined identifiers
Storage Codec
	Registry architecture
Port Plugin
Limitations
\end{verbatimtab}
\end{comment}
