\chapter{Conclusions}
\label{chap:conclusions}

The previous chapter has evaluated the solution
in light of the requirements stated in chapter \ref{chap:motivation}.
The following pages aim to provide
an overview of inferences made by my work
and some ideas for future research.
During the design and implementation of this project
the topics below have been addressed.
While some are directly related to the topic of this paper,
I feel that others
(such as generating name manglings in order to perform a language mapping)
may be of a more general interest.

\paragraph{Java Unit Testing with \ac{TTCN-3}}
The presented work proves that Java code can successfully be tested
using the \ac{TTCN-3} language.
Object creation, public field reading and writing and
method invocations with exception handling
can be performed from within \ac{TTCN-3} testcases.
Primitive and object types from Java
are automatically mapped to \ac{TTCN-3} types
and some compile time type safety is also provided to the test developer.

\paragraph{Language mapping}
Most of the work on this project
involved finding a \emph{translation scheme} between a Java type
and its representation as an auto-generated \ac{TTCN-3} type
(this correspondence between types needed to be in both directions).
The mapping has been particularly interesting because \ac{TTCN-3}
is not object oriented and lacks support for overloaded signatures.
This has led to a solution which tries to provide support
for the additional features of Java in the generated \ac{TTCN-3} modules.

\subparagraph{Name mangling}
The name mangling scheme is mostly straightforward:
element names are kept unchanged, and some receive a prefix and/or suffix
to protect against clashes with others from the same module.
The more interesting part deals with translating Java overloaded method names.
I feel this is of general interest and is not tied to the particular case
of Java and \ac{TTCN-3} tackled in this project.

\subparagraph{Object creation}
Apart from constructors, objects in Java may be created using
\verb=String= literals, the \verb=null= literal and arrays
(which must be treated as objects
and methods from the \verb=java.util.Arrays= class
may be used by test developers).
Because of these \emph{entry points} for objects in a Java program,
this project needed to provide them as well ---
and did so by using the meta-signatures \verb=makeNull= and \verb=makeString=.
Entry points for objects in a running program are an important subject
for any language mapping.

\subparagraph{Object storage}
For this project, the target language for the mapping (\ac{TTCN-3})
only held \emph{handles} for the real Java objects in use.
The runtime system intercepted signature calls and
decoded the information in the handles to obtain the actual objects.
The solution used for storing the objects and even its implementation
(with the \verb=ObjectHolder= hidden class)
may be of interest for other language mapping situations.


\section{Future work}
\label{sec:future-work}

This project has used and built upon
the architecture of a \ac{TTCN-3} test system.
The purpose has been to bring unit testing into the world of \ac{TTCN-3}
(and not to bring \ac{TTCN-3} to the area of unit testing local code,
although the project is a proof of concept
that this can be successfully done as well).
With this in mind, the following may be developed
by building upon this project:

\paragraph{Unit testing on embedded devices}
Embedded devices have limited resources
and are unable to load an auxiliary \emph{parent} system for
performing unit tests on software.
As a result, software for such devices is tested in a simulator
which may not always reproduce the conditions (and the limitations)
encountered when running on the device itself.

An approach to testing software directly on the device
is to load a resource-consuming test system on a different computer
and only load a small component of the test system on the embedded device.
This small component is responsible for
testing the target software on the device itself.
The complex, resource-consuming parent software
may communicate with this component
by exchanging messages through the network.
An attempt could be made to split the project presented here
in order to have a component (for performing the invocations)
running on a remote device.

\paragraph{Making use of \ac{TTCN-3}'s pattern matching}
One use case for this project involves using existing Java code
for communicating with a device.
Instead of encoding network packets by hand,
users may benefit from from a \verb=.jar= file with code from the vendor.
This code may allow communicating with the device on a higher level.

Such a use case may involve receiving a structure with several fields
as a return value.
This project has designed and implemented a general purpose solution
for Java unit testing,
so all non-primitive return types will be seen as objects
(and a handle will be used for them).

For cases where easier inspection of the received structure is desired
a custom solution could be used which:
\begin{itemize}
\item Maps certain classes to \ac{TTCN-3} records with corresponding fields
\item Falls back to the mapping presented in this paper for everything else
(generating handles for other types)
\end{itemize}
This approach (of customizing the mapping for some classes)
needs to be known when auto-generating the \ac{TTCN-3} modules,
when encoding/decoding and when invoking the Java methods.
It will allow test developers to both access the class fields more easily
and use the powerful pattern matching mechanism built into the language.
