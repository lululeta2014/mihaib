\section{Storage Codec}
\label{sec:storage-codec}

As described in section \ref{sec:test-system-arch},
the task of the codec is to encode sent messages
from their \ac{TTCN-3} representation
to the real world bits and bytes which need to be sent.
The codec also decodes received messages from their real world representation
to the data structures used in \ac{TTCN-3}.

In this project the \ac{SUT} is compiled Java code.
The unit tests call signatures and supply parameters for them,
test return values and handle thrown exceptions.
The information sent to the \ac{SUT} in a \emph{procedure oriented} scenario
is made up of signatures (which are called) and their respective parameters.
When testing a remote system (such as a \ac{CORBA} or \ac{RPC} system)
one might expect this information to be encoded in network packets
which get sent as messages to the remote \ac{SUT}.

The particular case of testing local Java bytecode
does not need to send any messages to a remote system
(though the test system itself may be split into several components
which communicate among themselves through messages,
as described in section \ref{sec:future-work}).
The tasks that need to be performed are:
\begin{itemize}
\item identifying the Java types and methods used
(by unmangling the \ac{TTCN-3} names)
\item invoking the requested methods
\item enqueueing the method return values or thrown exceptions
to be processed on the respective port (used for the procedure call)
\end{itemize}

The codec is responsible for encoding parameter types
from the \ac{TTCN-3} representation to a binary one
(as required by the standard interfaces for the codec)
and decode return values and exception types from a binary representation
to the corresponding \ac{TTCN-3} format.
The encoded parameters are communicated between the \emph{codec}
and the \emph{port plugin}
(which for this project resides on the same machine).

If the Java bytecode to be tested would reside on a different machine
(as suggested in section \ref{sec:future-work}).
the system might include a component which would run on that machine.
The binary messages between the components
would then be sent via the network.

Because objects are stateful they need to be \emph{kept alive} between calls.
As mentioned in section \ref{sec:type-safe-object-handles}
type-safe object handles prevent some programmer errors
but type checks also need to be performed at runtime to ensure correct usage
(as type names may be altered by hand or by using the \emph{casting template}
previously mentioned in section \ref{sec:type-casting}).
These checks are made both by the codec and the port plugin.
\begin{itemize}
\item when the codec is asked to
decode a returned value or a thrown exception type
(to make sure the actual value corresponds to the \emph{decoding hypothesis})
\item when the port plugin performs the invocation (to make sure
the handles point to objects of the correct type for parameters)
\end{itemize}

In order to keep the objects alive they will be stored
in an \emph{object registry}.
Since it is the codec's task to encode between the \ac{TTCN-3} representation
and some binary representation of object handles
the codec will also be the component responsible for
maintaining the object registry.

The previous paragraphs have covered \emph{what} tasks need to be performed
and \emph{why} some have been assigned to the codec.
The stage has now been set for presenting the codec's components
and describing the mechanisms used to achieve their functionality.


\subsection{Standard Interfaces}

Since the \ac{TTCN-3} language and test system architecture are standardized,
the codec needs to implement the standard \ac{ETSI} interface\\
\verb=org.etsi.ttcn.tci.TciCDProvided=\\
with its methods of interest presented in
Listing \ref{prog:codec-standard-interf}.
The semantics of the \verb=encode= and \verb=decode= methods
is to convert between \emph{binary} and \ac{TTCN-3} representations of data.
For this project byte representation of keys for the object registry
are returned.

\begin{program}
\verbatimtabinput{./java/codec-standard-interface.java}
\caption{Standard codec interface\label{prog:codec-standard-interf}}
\end{program}

The \ac{TTCN-3} \verb=Value= interface is used to describe
data types and data structures used in the language.
It has methods for retrieving the
\verb=Encoding=, \verb=EncodingVariant= and the \verb=Type=
(which allows inspection of the particular type of the data manipulated).
Several sub-interfaces are defined, both for primitive and structured types,
some of which are listed in Table \ref{tab:value-subinterfaces}.
Most of these types have \emph{accessor} (\verb=get= and \verb=set=) methods
for the data contained;
the \verb=RecordValue= interface has accessor methods which take
a \emph{field name} as argument.
The \verb=RecordValue= interface is used
for manipulating object handles in the codec
(which are represented as record structures
with a field for the object's \emph{key} in the registry
and another field containing a charstring representation
of the Java type name).

\begin{table}
\centering
\begin{tabular}{|l|}
\hline
BooleanValue \\
UniversalCharstringValue \\
IntegerValue \\
FloatValue \\
RecordOfValue \\
\hline
\end{tabular}
\caption{Sub-interfaces of the TTCN-3 \texttt{Value} type%
	\label{tab:value-subinterfaces}}
\end{table}

The \verb=TriMessage= interface allows the representation of a byte array.
This is the encoded message created by the codec for the port plugin.
The port plugin is expected to send it to the \ac{SUT}.
In a traditional scenario this would be sent as a network message.
For unit testing the \verb=TriMessage=s received by the port plugin
are byte representations of keys for registry objects.
These are the parameters for the invocations
and the particular object instance to invoke the method on.


\subsection{Object registry}

The \emph{object registry} is a \emph{Map}
assigning keys to the contained objects.
It is needed because of the way objects are referenced from \ac{TTCN-3} code;
test developers use a record type with two fields:
\begin{itemize}
\item A \emph{handle} (of type \verb=address=) used to identify the object
\item A \emph{field} of type \verb=charstring= used for type safety
\end{itemize}

Since the \emph{handle} is being used to reference an existing object,
the approach chosen is to keep the objects in a Map
and store the key in the handle.
The codec will pass this key to the port plugin when a signature is called.
For object parameters, the key stored in the handle is passed.
Parameters of primitive types are wrapped in an \verb=Object=
and the key for that object is passed.

When a method call has a return value,
the returned object is put in the registry
(after wrapping it if it is of a primitive type)
and the key for it is enqueued as a procedure reply.
When the testcase code tries to store this returned value
in a variable of a proper record type,
the codec is asked to \emph{decode} the returned value.
The codec then checks the object in the registry
against the \emph{encoding variant} of the type used for the return variable.
A successful attempt at decoding is signaled by returning
a proper \ac{TTCN-3} \verb=Value=
while an unsuccessful one is signaled by returning \verb=null=.


\subsubsection{The reversed Map}

Objects are added to the registry
when the signatures called have a return value.
The port plugin asks the storage codec to put a new object in the registry
and then enqueues the key for this object as return value.
It is desirable to only store each object once in the registry.
Since the registry only maps \emph{keys} to \emph{objects},
it is not well suited for checking if an object is already present.

For this reason a second map has been introduced,
called a \emph{reversed registry}.
It maps the objects to their keys in the other registry.
After being added to the first registry,
a new object and its key must also be added to the reversed registry.
When the programmer requests the deleting of an object from the registry,
it must be removed from both.

When adding a new object, the reversed registry is used
to check if the object is already present.
If so, its existing key is returned by the \verb=add()= operation,
otherwise a new key is generated and the object (together with its key)
is added to both registries.


\subsubsection{Implementation}

\paragraph{Encoding and Decoding}
translate between the \ac{TTCN-3} representation of data
(the \verb=Value= interface)
and a binary representation (the \verb=TriMessage= type).
The \verb=TriMessage= sent to the port plugin
(and also enqueued by the port plugin as return value)
is always the representation of a registry key.
The object pointed to is the actual Java object
or a wrapper for a primitive type.

The decoding operation also performs type checking
and if the object type and the handle type are incompatible,
it returns \verb=null=
to signal that decoding cannot be performed with this hypothesis.
When encoding \verb=integer= types, the \verb=java.lang.Integer= type
is always used as wrapper
(because \ac{TTCN-3} has a single \verb=IntegerValue= interface
used for representing unbounded  integers).
The port plugin will retrieve the \verb=Integer= from the registry
and perform the appropriate conversion to a primitive type
before calling a method.
Values returned by the port plugin are automatically wrapped
in the appropriate wrapper type (e.g.\ \verb=Short=, \verb=Byte=)
and the codec will check for this before downcasting
the object obtained from the registry and returning an \verb=IntegerValue=.

\paragraph{Registries}
The first registry maps \verb=String=s to \verb=Object=s
while the reverse registry performs the mapping the other way around,
using the \verb=Object=s as keys.
As previously stated, this requires each object to be stored only once
by the codec (since it must be used as a key in one of the registries).
Obviously, this is also desirable in order to avoid wasting memory.

The keys for the registry have been chosen of type \verb=String=.
They are actually string representations of integer numbers.
An internal counter is used for generating the keys,
and it is incremented whenever a new object is stored.
Keys are assigned consecutively (\verb="0"=, \verb="1"=, \verb="2"= etc.).

When objects are deleted from the registry their keys are not reused.
The only operation changing the internal counter is the adding of new objects.

Some particular challenges need to be overcome
when implementing these two registries.
Keys for a map are considered equal (i.e.\ they are the same key)
based on the \verb=boolean= result of the \verb=equals= method.
This is a crucial problem for storing user objects
and using them as keys.
Storing the distinct value \verb=null= may also be desirable.
When the codec tries to retrieve the object associated with a key,
\verb=java.util.Map= uses \verb=null=
to indicate a missing mapping for that key.
The codec must be able to distinguish this situation from a key mapped to the 
\verb=null= object.

These issues have been addressed by a slight change in design:
The objects are not directly stored in the maps.
They are wrapped inside objects of the type \verb=ObjectHolder=
presented in Listing \ref{prog:ObjectHolder}.
\verb=ObjectHolder= is a private inner type
not exposed outside the \verb=JavaCodec= class.
It is only used internally.
The \verb=encode= and \verb=decode= methods
\emph{always} use the stored \verb=Objects= as parameters or return values.
This holder type overrides the \verb=equals= method such that
it returns \verb=true= when two holders hold the same object
and \verb=false= otherwise.

Another important point to make (regarding implementation) is
the \emph{Contract for the \texttt{hashCode} method} in Java
which states that \emph{equal objects must have equal hashcodes},
as detailed in \citep{website:equals-hashcode}
and \citep{website:java-api}.\footnote{\raggedright%
	\url{http://java.sun.com/javase/6/docs/api/%
	java/lang/Object.html\#equals(java.lang.Object)}}
Thus the \verb=ObjectHolder= class also needs to override
the \verb=hashCode= method, and does so
by returning the hashCode of the encapsulated \verb=Object=.
Fulfilling this contract is especially important
because the \verb=ObjectHolder=s are used as keys in the reversed registry.
All implementation details are found in Listing \ref{prog:ObjectHolder}.

\begin{program}
\verbatimtabinput{./java/ObjectHolder.java}
\caption{ObjectHolder used for the registries\label{prog:ObjectHolder}}
\end{program}
