\chapter{Code Listings}


\section{The JAVAAUX module}
\label{sec:javaaux-appendix}

This is an auxiliary \ac{TTCN-3} module provided by this solution
and needed by all generated code.

{
\scriptsize
\listinginput[5]{1}{./ttcn3/JAVAAUX.ttcn3}
}


\section{The Storage Codec}

The storage codec extends the \verb=AbstractStorageCodec= class,
used for other (similar) language plugins.
This base class is responsible for encoding \ac{TTCN-3} \verb=Value=s
to binary data (\verb=TriMessage=s).
All implementing classes are only responsible for encoding and decoding
between \ac{TTCN-3} \verb=Values=s and the \verb=Object=s
stored in the registry.
As an example, the encoding and decoding of \verb=record= types is shown below:

{
\scriptsize
\renewcommand\verbatimtabsize{4\relax}
\listinginput[10]{1}{./java/JAVACodec.java}
}

If the \verb=isinstance= check in line 26 fails,
the decoding method returns \verb=null=.
This may happen, for instance, when catching a thrown exception:
\ac{TTCN-3} code may provide several catch clauses
for different types of exceptions (if they need to be handled differently).
Decoding will be tried with each of those types as \emph{decoding hypotheses},
until one matches.
Java inheritance \emph{is} supported:
if the programmer tries to catch a \verb=java.lang.Throwable=,
any thrown exception will be correctly decoded
(because all exceptions \emph{are} \verb=Throwable=s).


\section{The Port Plugin}

The port plugin's task is to carry out the actual invocations on Java bytecode.
It receives encoded registry keys from the codec
and requests the associated objects.
After performing the invocations, the port plugin stores the returned value
or thrown exception in the codec
and enqueues the key for the object on the appropriate port queue.

{
\scriptsize
\renewcommand\verbatimtabsize{4\relax}
\listinginput[10]{1}{./java/JAVAPortPlugin.java}
}

When a signature call is performed, the port plugin's \verb=triCall= method
is called (at line 70) and the signature is examined.
The \emph{encoding variant} specifies the original Java type.
If this indicates that the signature is defined in the \verb=JAVAAUX= module,
the meta-signature semantic is executed.
Lines 76--120 show the implementation of some of the meta-signatures.

Otherwise the signature name is checked for the special prefixes
and an unescaped separator suffix,
indicating that it is a constructor or an accessor.
If this is not the case, then it must represent a method mapping.
Constructor, accessor and method calls are carried out in a similar manner.
The first part of the listing shows how a method call is carried out.
The signature is unmangled and the original method name
and parameter types are obtained.
The supplied parameters are retrieved from the codec
and checked to be of the correct type.
In the end a separate thread is launched to carry out the invocation
and enqueue the returned value or thrown exception.
