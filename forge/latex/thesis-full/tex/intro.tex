\chapter{Introduction}

This paper presents an approach to unit testing Java code
using the \ac{TTCN-3} language.
\ac{TTCN-3} is mostly used for protocol conformance tests
and for testing remote systems by sending messages (black box testing).
Unit testing is usually done in other (programming) languages.
This projects is an attempt to extend the use of \ac{TTCN-3} test systems
to the area of unit testing.
The following paragraphs give an outline of this document.

Chapter \ref{chap:background} (\emph{\nameref{chap:background}})
gives the reader some background information on \ac{TTCN-3} technology
an presents some of its current uses.
Small code examples are included to give an indication of
what can be accomplished with the language.
They are followed by a presentation of
the standardized architecture of a \ac{TTCN-3} test system.
This chapter sets the stage for presenting
the design of the solution, the challenges encountered and the implementation.

Having an overview of the target domain (\ac{TTCN-3} test systems)
a detailed specification of the objectives can now be given.
Chapter \ref{chap:motivation} (\emph{\nameref{chap:motivation}})
re-states the motivation behind this project
and then provides a list of requirements from the solution.
These serve as a starting point for the solution design.

Chapter \ref{chap:description} (\emph{\nameref{chap:description}})
presents the design decisions, challenges faced and some implementation details
of the final work.
The integration with the standardized \ac{TTCN-3} architecture
is also shown here.

The final chapters
(\emph{\nameref{chap:evaluation}} and \emph{\nameref{chap:conclusions}})
examine the implemented work in light of the requirements from
chapter \emph{\nameref{chap:motivation}},
discuss its limitations
and state some directions for further developing this work.
