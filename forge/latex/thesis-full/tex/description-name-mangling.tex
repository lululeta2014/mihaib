\section{Name mangling}
\label{sec:name-mangling}

A \ac{TTCN-3} module must be generated for each Java type used.
Afterwards, the generated elements need to be mapped back to
the Java methods and (exception) types.
In order for this task to be successfully performed,
a name translation scheme needs to be defined.
Java names have several properties:

\begin{itemize}
\item User defined names are formed by letters, numbers
and the underscore character (\verb|_|)
\item Type names should be contained within a package
and may be defined inside other types
so that a fully qualified type name may look like:\\
\texttt{package.name.OuterType.InnerType}\\
in the source code, with the binary name:\\
\texttt{package.name.OuterType\$InnerType}
\item Types may be anonymous
(in which case they receive a compiler-generated name ---
which may look like \verb=package.OuterType$1=)
\item They are case sensitive
\item Binary names for \verb=Array= types consist of
one or more `\verb=[=' characters specifying the depth of the array nesting
followed by the encoding of the element type name which is defined in
Table \ref{tab:array-binary-names}
according to \citep{website:java-api}.\footnote{%
\url{http://java.sun.com/javase/6/docs/api/java/lang/Class.html\#getName()}}
\end{itemize}

\begin{table}
\centering
\begin{tabular}{|l || c|}
	\hline
	\emph{Element Type}	& \emph{Encoding} \\
	\hline
	boolean			& Z \\
	byte			& B \\
	char			& C \\
	class or interface	& L\emph{classname}; \\
	double			& D \\
	float			& F \\
	int			& I \\
	long			& J \\
	short			& S \\
	\hline
\end{tabular}
\caption{Java binary names for Array types\label{tab:array-binary-names}}
\end{table}

But crucial for the design of the naming scheme is the fact that:

\begin{itemize}
\item Java methods may be overloaded
(i.e.\ several methods inside the same Java type may have the same name
and different signatures)
\end{itemize}

Names used in \ac{TTCN-3} are also formed using
letters, numbers and the underscore character (\verb=_=)
and they are case sensitive.
\ac{TTCN-3} does not allow overloaded signatures.
Thus the challenges for the naming scheme are:

\begin{itemize}
\item To account for the special characters in Java type names
(\verb=.= and \verb=$=)
\item To guarantee that overloaded Java methods
map to different \ac{TTCN-3} signatures
and can be used without causing any conflicts
\end{itemize}

As previously mentioned in Section \ref{sec:project-objectives}
another goal is to have a \emph{stable} naming scheme,
i.e.\ one does not change name mappings when the Java types change
(e.g.\ by adding an overloaded method to the type and mapping it again).


\subsection{Treating special characters}
\label{sec:special-characters}

The special characters which need to be changed
are the ones found in Java binary names
but not allowed in \ac{TTCN-3} identifiers.
These are the dot (\verb=.=), the dollar sign (\verb=$=)
and the array nesting indicator (\verb=[=).

These characters cannot simply be discarded because
at runtime the Codec needs to perform a reverse name mapping
and find out the Java type name
(another reason is that several names may clash
by simply discarding the special characters).
This leads to the conclusion that information about the special characters
needs to be kept in the mapped \ac{TTCN-3} name
and the translation scheme needs to represent this information
using legal characters.

The dot (\verb=.=) and the dollar (\verb=$=) characters
will be discussed first.
Since the array depth character (\verb=[=)
may only appear at the start of a binary type name
a different (and simpler) translation scheme will be used for it.


\subsubsection{Replacing with a valid string}

The information about the special characters will be kept in the mapped names
by replacing special characters with strings
which may be part of \ac{TTCN-3} identifiers.
If the replacement strings are called \emph{<rDot>} and \emph{<rDollar>}
then the Java name\\
\verb=package.Outer$Inner=\\
will be mapped to\\
\verb=package=\emph{<rDot>}\verb=Outer=\emph{<rDollar>}\verb=Inner=

The strings chosen for replacement are shown in
Table \ref{tab:replacement-strings}.
They have been chosen because
they are expected to rarely be part of user defined names
(and the one most commonly used, ``\verb=_0_='' for replacing ``\verb=.='',
bears a graphical resemblance to a dot,
so that human examination of automatically generated types is intuitive).

\begin{table}[htb]
\centering
\begin{tabular}{|r || l|}
	\hline
	\emph{Special character} & \emph{Replacement String} \\
	\hline
	\verb=.= & \verb=_0_= \\
	\verb=$= & \verb=_1_= \\
	\hline
\end{tabular}
\caption{Replacement strings for special characters%
	\label{tab:replacement-strings}}
\end{table}


\subsubsection{Escaping the replacement string}

Replacement strings for special characters have been chosen
in an attempt to avoid any strings that may be part of user defined type names.
While this assumption is expected to hold for the majority of use cases,
it is of course legal to define Java type names which include
these particular character strings (``\verb=_0_='' and ``\verb=_1_='').

For example the type name \verb=com.My_0_Type=
will be mapped to \verb=com_0_My_0_Type= in the automatically generated
\ac{TTCN-3} module.
At runtime, the codec will encounter this mapped name
and will perform a reverse mapping to obtain the original Java name.
The reverse mapping will provide \verb=com.My.Type= as result,
which is incorrect.

The previously presented mapping scheme may easily be altered
so that it will be able to also handle this additional case.
The solution is to \emph{escape} the special strings
if they are found in the user defined names:
Every string is \emph{doubled} (written twice)
in the resulting mapped \ac{TTCN-3} name.

This improved algorithm requires a different reverse mapping approach.
The replacement strings introduced earlier (e.g.\ ``\verb=_0_='')
must not be reverse mapped to the original character (e.g.\ ``\verb=.='')
in a \emph{greedy} fashion.
Instead, the reverse mapping operation needs to \emph{look ahead}
when \emph{parsing} the generated \ac{TTCN-3} name.
When a replacement string is found the algorithm will look ahead
to check if this occurrence if immediately followed by another one.
If so, both occurrences are \emph{consumed}
and mapped to a single occurrence of the respective string.
Otherwise the replacement string is reverse mapped to
the original special character.
The pseudocode is shown in Listing \ref{prog:unmangle-escaped-strings}.

\begin{program}
\verbatimtabinput{./pseudocode/unmangle-escaped-strings}
\caption{Pseudocode for unmangling special characters %
	and checking for escaped replacement strings%
	\label{prog:unmangle-escaped-strings}}
\end{program}

In order to avoid any corner-case problems while reverse-mapping names,
the direct mapping escapes first the \emph{dot replacement string} (\verb=_0_=)
then the \emph{dollar replacement string} (\verb=_1_=)
and the reverse mapping un-escapes them in reverse order.


\subsubsection{Requirements for avoiding ambiguity}
\label{sec:avoiding-ambiguity}

The previously presented escaping mechanism introduces a subtle issue
which will be presented by using an example.
Table \ref{tab:mapping-ambiguity} shows two Java names which correspond
to the same \ac{TTCN-3} generated name.
We can easily check that both will determine the same mapped name
to be automatically generated ---
which is not a problem during the direct mapping,
but becomes an important issue when reverse mapping
(in order to find out the original Java name).

\begin{table}[htb]
\centering
\begin{tabular}{|c || c|}
	\hline
	\emph{\ac{TTCN-3} generated name} & \emph{Java name} \\
	\hline
	\multirow{2}{*}{\texttt{package\_0\_\_0\_\_0\_Type}} &
		\verb=package_0_.Type= \\
	& \verb=package._0_Type= \\
	\hline
\end{tabular}
\caption{Reverse mapping ambiguity\label{tab:mapping-ambiguity}}
\end{table}

In order to overcome this issue a restriction must be imposed
on user defined names to be mapped.
Either names aren't allowed to start with a replacement string,
or they are not allowed to end with one.
When reverse mapping a name with an uneven number of
consecutive replacement strings
the algorithm will replace either the last or the first occurrence
with the escaped character.
Since conventions used in the Java Programming language state that:
\begin{itemize}
\item Package names should be reverse domain names owned by the programmer
and should be in lowercase
\item Class names should start with a capital letter
\end{itemize}
the restriction imposed for mapped names is that:
\emph{User defined Java names must not start with one of the strings
replacing special characters (``\texttt{\_0\_}'' and ``\texttt{\_1\_}'')}.
However user defined names may end with these strings.


\subsubsection{Mapping the array depth character}

The array depth specifier (\verb=[=) may only appear at the start
of a binary Java name.
All names in Java and \ac{TTCN-3} must not start with a digit
(this is also true for most other programming languages).
This allows for the following approach to mapping the name of an Array type:
\begin{itemize}
\item Start with a prefix
\item Continue with a number specifying the array depth
\item Use the mapped name of the element type
(not the binary name of the array type
which starts with ``\verb=[='' and ends with
the encoding of the element type name from Table \ref{tab:array-binary-names})
\end{itemize}

Listing \ref{prog:mapping-array-depth} shows an example of how
array types are mapped to \ac{TTCN-3} types.
By starting with a number to specify the array depth
and using the element type name
(not the encoding from Table \ref{tab:array-binary-names})
the generated names are more user friendly.
The double underscore separator string (\verb=__=) ---
appended at the end of the generated type name for a Java array ---
will be explained in section \ref{sec:description-naming-conventions}.

\begin{program}
\verbatimtabinput{./ttcn3/mapping-array-depth.ttcn3}
\caption{Mapping of the \texttt{String[][]} Java Array type%
	\label{prog:mapping-array-depth}}
\end{program}


\subsection{Mangling (overloaded) method names}

The task of mapping Java types to \ac{TTCN-3} types
consists mainly of generating appropriate data structures
and naming them similar to the original elements from Java.
For the most part, the chosen names are the original ones
with slight changes in order to avoid special characters
not allowed in \ac{TTCN-3}
(as detailed in section \ref{sec:special-characters}).

However mapping \emph{method} names from Java to
\ac{TTCN-3} \emph{signature}s is not as trivial.
The \ac{TTCN-3} language does not allow \emph{overloaded signatures},
but Java does.
Overloaded methods belong to the same type (e.g.\ class, interface)
and have the same name but different argument lists.
For these reasons the mangling scheme
needs to generate different names for such methods
and must also be able to perform the reverse operation.

Several options come to mind,
each with advantages and disadvantages of its own.
An overview of such methods is presented below
and the chosen mangling scheme is detailed.


\subsubsection{Avoiding unnecessary manglings}
One option which comes to mind is to simply ignore the problem
for non-overloaded functions, and when overloaded functions are encountered
to use one of the approaches described below.
A solution for overloaded methods would still be needed of course,
but the goal would be to use friendlier (non-mangled) names
for methods which are not overloaded.
An important point to make is that this is not an option,
since the generated mangling would not be \emph{stable}
as required in section \ref{sec:project-objectives}.

The mangling scheme will therefore be needed even for non-overloaded methods,
as this is the only way to guarantee that a mangled name for a method
will not change even if the type is extended
(e.g.\ by overloading that method).


\subsubsection{Counting the overloads}
A first mangling attempt is to simply \emph{count} the signatures
in the following manner:
\begin{itemize}
\item Sort the methods using some criteria
(e.g.\ lexicographically according to the argument list,
by parameter count and then lexicographically by argument types)
\item Name the signatures by appending `\verb=_1=', `\verb=_2=' etc.
at the end of the name.
\end{itemize}
The problem with this approach is that, no matter the sorting criteria chosen,
a Java type may be extended with another overloaded signature
which will end up `in the middle' of the previous sequence,
thus breaking the stability.
What is more, this method does not allow a user to know
the mangled name for a method
without knowing the final version of the type's interface.


\subsubsection{Appending parameter type names to the signature name}
This idea is also used in the Java to \ac{IDL} mapping for \ac{CORBA}
as described in \citep[pg~10]{website:java2idl-mapping}.
That mapping keeps the original names for methods without overloads
and switches to this mangling scheme only for overloaded methods.
As just shown above, a stable mangling scheme cannot leave
any method names unmangled.
With the small change of mangling everything,
this is the naming scheme chosen for this project.

The mangling scheme will concatenate the following elements
to form a signature name:
\begin{itemize}
\item The Java method name
\item The parameter count (optional)
\item The (mapped) type names of the method parameters
(as defined in section \ref{sec:type-mapping}).
\end{itemize}

The above mentioned elements are concatenated using a \emph{separator string}
to form the mangled signature name.
At runtime, when the signature name needs to be unmangled
in order to obtain the original Java name,
the name is `broken' into its elements
(by splitting it around occurrences of the separator string).
The chosen separator string in this case is ``\verb+__+''.

The ambiguity problem described in section \ref{sec:avoiding-ambiguity}
is also present here.
If the concatenated elements mentioned above contain the separator string,
it needs to be escaped.
The unmangling algorithm will similarly need to look ahead
and replace \emph{doubled} separators with a single occurrence,
and split the string where single separators are encountered.
The same problem discussed earlier arises
when an uneven number of separators occur consecutively.
The same convention will be used:
\emph{User defined names may not begin with the separator string},
but they may end with it.


\subsubsection{Method mapping example}
As an example,
the mapping for the \verb=equals()= method in class \verb=java.lang.Object=
is shown in Listing \ref{prog:signature-mapping-example}.
This method, being part of the \verb=Object= interface ---
the \emph{root} of the Java type system ---
is generated when mapping any Java class.
Because it is an instance method,
the \ac{TTCN-3} signature takes an extra first parameter:
the handle for a particular object instance.
The type of this object corresponds to the Java type for which the mapping
has been generated.
For this example, the \verb=Object= class has been mapped.
The following short analysis will explain
the parts of the signature definition:

\begin{program}
\verbatimtabinput{./ttcn3/signature-equals.ttcn3}
\caption{Method mapping example for \texttt{java.lang.Object.equals()}%
	\label{prog:signature-mapping-example}}
\end{program}

\begin{description}
\item[The signature name] consists of
the \emph{method name} (\verb=equals=)
followed by the \emph{parameter count} (\verb=1=)
and then the mangled name of each parameter type.
The different parts of the method signature are concatenated using
the separator string ``\verb=__=''.
\item[The object handle] is an extra first parameter
of the type for which the module (mapping) has been generated.
In this case it is \verb=Object=.
\item[The signature parameters]
are listed in the same order as for the Java method
and are named p1, p2 etc.
The reflection \ac{API} does not provide information
about method parameter names, only about their types.
\item[The return value] has a \ac{TTCN-3} type
as described in section \ref{sec:type-mapping}.
Primitive values are mapped directly between the two languages,
and record types are created for Java \verb=Object= types.
This method has a \verb=boolean= return value.
\item[Thrown exceptions] listed for each signature
include all exception types declared to be thrown by the Java method.
These may be both \emph{checked} and \emph{unchecked}.
As stated in \citep{website:java-tutorial-exceptions}\footnote{\raggedright%
	\url{http://java.sun.com/docs/books/tutorial/%
	essential/exceptions/catchOrDeclare.html}}
the \emph{Catch or Specify requirement} states that
checked exceptions must be caught or declared to be thrown.
However unchecked exceptions not declared in the method definition
may still be thrown, and test developers may need to catch those as well.
For this reason, the \verb=Throwable= type is always added
to the \emph{set} of exception types thrown by any method.
\end{description}


\subsection{Naming conventions for generated identifiers}
\label{sec:description-naming-conventions}

The previous sections have shown how names are mapped between the two languages
(for types, methods and generated accessor signatures).
The two main tasks are:
\begin{itemize}
\item Deciding which data structures to use for each mapped element
\item Mangling the generated name,
so that it contains only legal \ac{TTCN-3} characters
and the language's lack of support for overloaded signatures is overcome
\end{itemize}

However another important issue is
\emph{avoiding name-clashes between user-defined and generated identifiers},
so the approach of using prefixes (and suffixes) has been used.


\subsubsection{Marking generated identifiers}

A generated module contains signatures
(for methods, constructors and accessors for public fields)
and type definitions.
The methods from a Java type will be mapped to signatures with different names.
But adding records for all the types used
and signatures for constructors and accessors (which have no predefined name)
may lead to name clashes.
For this reason the signatures are normally generated,
but all other names are marked to indicate that
they do not represent Java method mappings.

Because the separator string ``\verb+__+'' used to concatenate
elements of a generated name
is always escaped within those elements,
it can be added at the end of generated names
to mark the fact that they have been internally generated.

Such \emph{problematic} names are those generated for
\emph{constructor signatures}, \emph{accessor signatures}
and \emph{record types}.
By marking them as internally generated
(meaning they are not direct mappings of Java methods)
they will not clash with method manglings.
The mark used is distinctive because it is un-escaped and found at the end.
As an example, the mangled name `\verb=someName='
becomes `\verb=someName__=' by using this mechanism.


\subsubsection{Prefixes for generated types}

In order to make the generated code more readable,
prefixes are used when generating the mapping.
These indicate the type of the element
by simply reading the name (without looking at its structure or definition).
A list of used prefixes follows:

\begin{description}

\item[Types] get the ``\verb=type_='' prefix and the ``\verb+__+'' suffix.
After mangling, the type names become less easy to read
and the prefix may aid in this respect.
As an example, the generated type name for `\verb=java.lang.String=' is:\\
`\verb=type_java_0_lang_0_String__='.

\item[Method signatures] get no prefixes or suffixes.
This will normally be readable as they start with the original name
and include the parameter count and the concatenated type names.
These can not clash with another generated name because they cannot end
with an unescaped separator.
The long generated names are not a problem since the test developer may easily
define a \ac{TTCN-3} template with a short name.

\item[Constructor signatures] get generated
in the same manner as a normal method signature
called ``\verb=construct='' and get a suffix (\verb+__+).
The suffix is necessary to ensure against a name clash
with another method called ``construct'', which may be present
in the type being mapped.
As an example, the no-arg constructor for ``\verb=java.lang.String=''
is mapped to the following signature:\\
\verb=signature construct__0__() return type_java_0_lang_0_String__=\\
\verb=exception (type_java_0_lang_0_Throwable__);=

\item[Accessor signatures] get generated
in the same manner as a normal method signature called
``\verb=get_fieldName='' (for getters)
and ``\verb=set_fieldName='' (for setter signatures)
and get the ``\verb=__='' suffix at the end to mark them as
\emph{internally generated}.
Listing \ref{prog:accessor-example} shows the accessor signatures
generated for the \verb=memory= field of the \verb=Calculator= class in
Listing \ref{prog:example-mapping-java-class.java}.

\begin{program}
\verbatimtabinput{./ttcn3/accessor-example.ttcn3}
\caption{Example accessor signatures for the \texttt{Calculator.memory} field%
	\label{prog:accessor-example}}
\end{program}

The getter returns the field value.
The setter signature has no return value
but an extra parameter for the new field value.
In class \verb=Calculator=, \verb=memory= is an instance (non-static) field,
so both methods take a first parameter for the object handle.

\item[Port types] have the prefix ``\verb+port_+'' followed by
the mapped type name and end with the ``\verb+__+'' suffix.
Port types are not needed but are generated for convenience.
The user may start calling signatures through this provided port type
or may define his/her own, listing all signatures of interest
(from several modules if needed).

The port type defined when mapping the \verb=Calculator= class
from Listing \ref{prog:example-mapping-java-class.java}
is shown in Listing \ref{prog:port-type-example}.
It is a procedure oriented port type and all signatures defined in the module
are listed for usage in its definition.
Only a few signatures have been included
in Listing \ref{prog:port-type-example} to serve as an example.

\begin{program}
\verbatimtabinput{./ttcn3/port-type-example.ttcn3}
\caption{The generated port type for the \texttt{Calculator} class%
	\label{prog:port-type-example}}
\end{program}

\end{description}

