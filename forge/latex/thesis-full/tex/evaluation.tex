\chapter{Evaluation}
\label{chap:evaluation}

The solution presented meets the goals stated
in chapter \ref{chap:motivation}.
It allows for unit testing Java code
by using \ac{TTCN-3}'s standardized architecture
and the language's powerful features.
Method calls, constructor calls, reading and writing of public fields
and exception handling can be performed by test developers.
The \emph{stable} mangling scheme devised
allows tests to be written during development,
because extending the interface of the tested code
does not impact existing test cases.

An important challenge was \emph{mapping names} between the languages.
The fact that \ac{TTCN-3} is not object oriented
and its lack of support for overloaded signatures
have led to the type safety considerations
in section \ref{sec:using-java-objects-from-ttcn3}
and to the name mangling scheme presented in section \ref{sec:name-mangling}.

The generated names are long and difficult to use directly,
but must be so in order to compensate for
the lack of overloading support in \ac{TTCN-3}.
However the language's powerful type semantics allow test developers
to easily define \emph{aliases} with short names for the
types and signatures they need.
An example of such template definitions is presented in
Listing \ref{prog:friendly-templates}.

\begin{program}
\verbatimtabinput{./ttcn3/friendly-templates.ttcn3}
\caption{Defining templates with user-friendly names%
	\label{prog:friendly-templates}}
\end{program}


\section{Limitations}

An important limitation of this solution is its
\emph{performance} when running very small unit tests.
The \ac{TTCN-3} architecture is well suited for communication testing.
But it requires unit tests to use
the standard components and ports of a test system,
adding a significant overhead for very small performance tests.

As an example of such a test,
Table \ref{tab:random-generation} presents the results
of generating random numbers using the \verb=java.util.Random= class.
Testing with \ac{TTCN-3} involves using the architecture described
in section \ref{sec:test-system-arch}
and creating a new thread for each Java invocation.

\begin{table}[htb]
\centering
\begin{tabular}{r | c | r | c | r}
 & \verb=nextInt= calls & time & \verb=nextGaussian= calls & time \\
\hline
from Java code & 10 million & 0.5 s & 10 million & 2.6 s \\
from \ac{TTCN-3} & 1000 & 2 s & 1000 & 2.1 s \\
\end{tabular}
\caption{Performance critical testing\label{tab:random-generation}}
\end{table}

As the example shows,
the difference in performance drops
when the invoked methods take longer to complete.
This may be the case in communication testing.
What is more, if such large numbers of very short tests aren't required,
the performance penalty should not pose a problem.

Unit testing for any code can be performed using this solution,
but its strengths are testing code which performs communication
or integrating it with message-based tests
(e.g.\ to set or verify the internal state of a system
while also testing it as a black box, by sending messages to it).

Another limitation of the current solution is that
primitive \verb=long= values larger than a Java \verb=int= can not be used,
because the standardized \ac{ETSI} interfaces (and their implementing classes)
do not allow the use of such values.
This is an external (and probably temporary) limitation imposed on the project.
Its design and implementation handle this case correctly
(by using the \verb=BigInteger= class
expected by the standard interfaces in such cases).
