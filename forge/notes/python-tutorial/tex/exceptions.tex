\section{Exceptions}

An \verb=except= clause may name multiple exceptions as a paranthesized tuple.
The last except clause may omit the exception name to serve as a wildcard.
The optional \verb=else= clause is executed
if the \verb=try= clause does not raise an exception.

An exception may have an associated value,
known as the exception \emph{argument}.
The arguments are stored in \verb=instance.args=.
For convenience, the exception instance defines \verb=__str__()=
so the arguments can be printed directly
(so printing an exception only prints the arguments, not the exception name).
If an exception has arguments, they are printed as the last part (‘detail’)
of the message for unhandled exceptions.

\verb=raise= by itself in an \verb=except= clause re-raises the exception.

When creating a module that can raise several distinct errors,
a common practice is to create a base class
for exceptions defined by that module,
and subclass that to specific exception classes for different error conditions.

\subsection{Predefined Clean-up Actions}

Some objects, like files, provide predefined cleanup actions.
The \verb=with= statement allows objects like files to be used
in a way that ensures they are always cleaned up promptly and correctly.
