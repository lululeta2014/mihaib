\section{Functions}

\verb=def= introduces a function definition.
Its first statement can be a string literal, its \emph{docstring}.

The execution of a function introduces a new symbol table
used for the local variables of the function.
All variable assignments in a function
store the value in the local symbol table.
Variable references first look in the local symbol table,
the in the local symbol tables of enclosing functions,
then in the global symbol table, and finally in the table of built-in names.
Thus, global variables cannot be directly assigned a value within a function
(unless named in a \verb=global= statement), although they may be referenced.

The actual parameters (arguments) to a function call
are introduced in the local symbol table of the called function
when it is called.
When a function calls another function,
a new local symbol table is created for that call.

A function definition introduces the function name in the current symbol table.
The value of the function name has a type
that is recognized by the interpreter as a user-defined function.
This value can be assigned to another name
which can then also be used as a function.

\verb=return= without an expression argument returns \verb=None=.
Falling off the end of a function also returns \verb=None=.

\subsection{Default Argument Values}

The default values are evaluated at the point of function definition
in the defining scope.
Listing \ref{func:default-arg-eval-point} prints \verb=5=.

\begin{program}
\verbatimtabinput{./py/func-default-arg-eval-point.py}
\caption{Default arg values are evaluated at the point of function definition%
\label{func:default-arg-eval-point}}
\end{program}

The default value is evaluated only once!
This matters when the default is a mutable object.
Listing \ref{func:default-val-eval-once} will print
\verb=[1]=, \verb=[1, 2]= then \verb=[1, 2, 3]=.

\begin{program}
\verbatimtabinput{./py/func-default-val-eval-once.py}
\caption{Default arg values are evaluated only once%
\label{func:default-val-eval-once}}
\end{program}

\subsection{Keyword Arguments}

A function call must have any positional arguments
followed by any keyword arguments,
no argument may receive a value more than once.

When a final formal parameter of the form \verb=**name= is present,
it receives a dictionary containing all keyword arguments
except for those corresponding to a formal parameter.\\
This may be combined with a formal parameter of the form \verb=*name=
(described in the next section)
which receives a tuple containing
the positional arguments beyond the formal parameter list.
(\verb=*name= must occur before \verb=**name=).

\subsection{Arbitrary Argument Lists}

Listing \ref{func:arbitrary-arg-list} specifies
that the function can be called with an arbitrary number of arguments.
These will be wrapped up in a tuple.
Before the variable number of arguments,
zero or more normal arguments may occur.
Normally, these variadic arguments
will be last in the list of formal parameters
because they scoop up all remaining input arguments passed to the function.
Any formal parameters which occur after the \verb=*args= parameter
are keyword-only arguments (they can't be used as positional arguments).

\begin{program}
\verbatimtabinput{./py/func-arbitrary-arg-list.py}
\caption{Arbitrary argument list\label{func:arbitrary-arg-list}}
\end{program}

\subsection{Unpacking Argument Lists}
The reverse situation occurs when the arguments are already in a list or tuple
but need to be unpacked
for a function call requiring separate positional arguments.
Write the function call with the \verb=*=-operator
to unpack the arguments out of a list or tuple.
In the same fasion, dictionaries can deliver keyword arguments
with the \verb=**=-operator.

\subsection{Lambda Forms}

With the \verb=lambda= keyword, small anonymous functions can be created.
Lambda forms are syntactically restricted to a single expression
and can be used wherever function objects are required.
Semantically, they're syntactic sugar for a normal function definition.
Like nested function definitions,
they can reference variables from the containing scope.
