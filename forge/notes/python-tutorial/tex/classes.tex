\section{Classes}

The class inheritance mechanism allows multiple base classes,
a derived class can override any methods of its base class or classes,
and a method can call the method of a base class with the same name.
Classes are created at runtime and can be modified further after creation.

In C++ terminology, normally class members are \emph{public}
and all member functions are \emph{virtual}.
There are no shorthands for referencing the object's members from its methods:
the method function is declared with an explicit first argument
representing the object,
which is provided implicitly by the call.
Classes themselves are objects.
This provides semantics for importing and renaming.
Built-in types can be used as base classes for extension by the user.
Most built-in operators with special syntax
(arithmetic operators, subscripting etc.)
can be redefined for class instances.

Objects have individuality,
and multiple names (in multiple scopes) can be bound to the same object.


\subsection{Scopes and Namespaces}

A \emph{namespace} is a mapping from names to objects.
Examples of namespaces are:
the set of built-in names; the global names in a module;
and the local names in a function invocation.
In a sense the set of attributes of an object also form a namespace.

We'll use the word \emph{attribute} for any name following a dot.
Strictly speaking, references to names in modules are attribute references:
in \verb=modname.funcname=,
\verb=modname= is a module object and \verb=funcname= is an attribute of it.
In this case there happens to be a straightforward mapping between
the module's attributes and the global names defined in the module:
they share the same namespace.

Attributes may be read-only or writable.
In the latter case, assignment to attributes is possible.
Module attributes are writable: \verb+modname.the_answer = 42+.
Writable attributes may also be deleted with the \verb=del= statement.

Namespaces are created at different moments and have different lifetimes.
The namespace containing the built-in names
is created when the Python interpreter starts up, and is never deleted.
The global namespace for a module
is created when the module definition is read in;
normally, module namespaces also last until the interpreter quits.
The statements executed by the top-level invocation of the interpreter,
either read from a script file or interactively,
are considered part of a module called \verb=__main__=,
so they have their own global namespace.
(The built-in names actually also live in a module called \verb=builtins=).

The local namespace for a function is created when the function is called,
and deleted when the function returns
or raises an exception that is not handled within the function.
(Actually, forgetting would be a better way to describe what actually happens).

A \emph{scope} is a textual region of a Python program
where a namespace is directly accessible.
\emph{Directly accessible} here means that
an unqualified reference to a name attempts to find the name in the namespace.

Although scopes are determined statically, they are used dynamically.
At any time during execution,
there are at least three nested scopes
whose namespaces are directly accessible:

\begin{itemize}
\item the innermost scope, which is searched first, contains the local names
\item the scopes of any enclosing functions,
which are searched starting with the nearest enclosing scope,
contain non-local, but also non-global names
\item the next-to-last scope contains the current module's global names
\item the outermost scope (searched last)
is the namespace containing built-in names
\end{itemize}

If a name is declared global, then all references and assignments
go directly to the middle scope containing the module's global names.
To rebind variables found outside of the innermost scope,
the \verb=nonlocal= statement can be used;
if not declared nonlocal, those variables are read-only
(an attempt to write to such a variable
will simply create a new local variable in the innermost scope,
leaving the identically named outer variable unchanged).

Usually, the local scope references the local names
of the (textually) current function.
Outside functions, the local scope
references the same namespace as the global scope: the module's namespace.
Class definitions place yet another namespace in the local scope.

Scopes are determined textually:
the global scope of a function defined in a module is that module's namespace,
no matter from where or by what alias the function is called.
On the other hand, the actual search for names is done dynamically,
at run time – however, the language definition is evolving
towards static name resolution, at "compile" time,
so don't rely on dynamic name resolution!
(In fact, local variables are already determined statically.)

If no \verb=global= statement is in effect,
assignments to names always go into the innermost scope.
Assignments to not copy data – they just bind names to objects.
The same is true for deletions:
the statement \verb=del x= removes the binding of \verb=x=
from the namespace referenced by the local scope.
In fact, all operations that introduce new names use the local scope:
in particular, \verb=import= statements and function definitions
bind the module or function name in the local scope.

The \verb=global= statement can be used to indicate
that particular variables live in the global scope and should be rebound there;
the \verb=nonlocal= statement indicates that particular variables
live in an enclosing scope and should be rebound there.


\subsection{A First Look at Classes}

Class definitions, like function definitions,
must be executed before they have any effect.
(You could conceivably place a class definition
in a branch of an \verb=if= statement, or inside a function).

The statements inside a class definition are usually function definitions,
but other statements are allowed and sometimes useful.
The function definitions inside a class
normally have a peculiar form of argument list,
dictated by the calling conventions for methods.

When a class definition is entered, a new namespace is created,
and used as the local scope – thus, all assignments to local variables
go into this new namespace.
In particular, function definitions bind the name of the new function here.

When a class definition is left normally (via the end),
a \emph{class object} is created.
This is basically a wrapper
around the contents of the namespace created by the class definition.
The original local scope
(the one in effect just before the class definition was entered) is reinstated,
and the class object is bound here
to the class name given in the class definition header.


\subsubsection{Class Objects}

Class objects support two kinds of operations:
attribute references and instantiation.

\emph{Attribute references} use the standard syntax
used for all attribute references in Python: \verb=obj.name=.
Valid attribute names are all the names that were in the class's namespace
when the class object was created.
In Listing \ref{classes:simple} \verb=MyClass.i= and \verb=MyClass.f=
are valid attribute references, returning an integer and a function object,
respectively.
Class attributes can also be assigned to.

\begin{program}
\verbatimtabinput{./py/classes-simple.py}
\caption{Simple class definition\label{classes:simple}}
\end{program}

Class \emph{instantiation} uses function notation.
The instantiation operation --- "calling" a class object ---
creates an empty object.
Many classes like to create objects
with instances customized to a specific initial state.
Therefore a class may define a special method named \verb=__init__()=
like in Listing \ref{classes:init}.
When a class defines an \verb=__init__()= method,
class instantiation automatically invokes \verb=__init__()=
for the newly-created class instance.
Of course, the \verb=__init__()= method may have arguments
for greater flexibility.
In that case, arguments given to the class instantiation operator
are passed on to \verb=__init__()=.

\begin{program}
\verbatimtabinput{./py/classes-init.py}
\caption{init method\label{classes:init}}
\end{program}


\subsubsection{Instance Objects}

The only operations understood by instance objects are attribute references.
There are two kinds of valid attribute names: data attributes and methods.

\emph{Data attributes} need not be declared.
Like local variables,
they spring into existence when they are first assigned to.

A \emph{method} is a function that "belongs to" an object.
(In Python, the term method is no unique to class instances:
other object types can have methods as well.
For example, list objects.
However, in the following discussion, we'll use the term method
exclusively to mean methods of class instance objects,
unless explicitly stated otherwise.)

Valid method names of an instance object depend on its class.
By definition, all attributes of a class that are function objects
define corresponding methods of its instances.
But \verb=x.f= is not the same as \verb=MyClass.f= ---
it is a method object, not a function object.


\subsubsection{Method Objects}

Usually a method is called right after it is \emph{bound}: \verb=x.f()=.
However, it is not necessary to call a method right away:
\verb=x.f= is a method object, and can be stored away and called later.

When a method is called,
the object is passed as the first argument of the function.
\verb=x.f()= is exactly equivalent to \verb=MyClass.f(x)=.

A look at the implementation:
when an instance attribute is referenced that isn't a data attribute,
its calss is searched.
If the name denotes a valid class attribute that is a function object,
a method object is created by packing (pointers to)
the instance object and the function object just found together
in an abstract object: this is the method object.
When the method object is called with an argument list,
a new argument list is constructed
from the instance object and the argument list,
and the function is called with this new argument list.


\subsubsection{Random Remarks}

Data attributes override method attributes with the same name.

Data attributes may be referenced by methods
as well as by ordinary users of an object.
(Data hiding can't be enforced.)
Clients should use data attributes with care,
and may add data attributes to an instance object
without affecting the validity of the methods,
as long as name conflicts are avoided.

There is no shorthand for referencing data attributes (or other methods!)
from within methods.
This way local variables can't be confused with instance variables.

Often the first argument of a method is called \verb=self=.
This is only a convention (\verb=self= has no special meaning to Python).

Any function object that is a class attribute
defines a method for instances of that class.
It is not necessary that the function definition
is textually enclosed in the class definition:
assigning a function object to a local variable in the class is also ok.

Methods may reference global names in the same way as ordinary functions.
The global scope associated with a method
is the module containing the class definition.
(The class itself is never used as a global scope.)
There are many legitimate uses of the global scope:
functions and modules imported into the global scope can be used by methods,
as well as functions and classes defined in it.
Usually the class containing the method is itself defined in this global scope.

Each value is an object, and therefore has a class (also called its type).
It's stored as \verb=object.__class__=.


\subsection{Inheritance}

The syntax for a derived class definition is\\
\verb=class DerivedClassName(BaseClassName):=\\
Execution of a derived class definition proceeds the same as for a base class.
When the class object is constructed, the base class is remembered.
This is used for resolving attribute references:
if a requested attribute is not found in the class,
the search proceeds to look in the base class.
This rule is applied recursively
if the base class itself is derived from some other class.

There's nothing special about instantiation of derived classes.\\
\verb=DerivedClassName()= creates a new instance of the class.
Method references are resolved as follows:
the corresponding class attribute is searched,
descending down the chain of base classes if necessary,
and the method reference is valid if this yields a function object.

Derived classes may override methods of their base classes.
Because methods have no special privileges
when calling other methods of the same object,
a metod of a base class
that calls another method defined in the same base class
may end up calling a method of a derived class that overrides it
(in C++ terminology, all methods in Python are effectively \emph{virtual}).

An overriding method in a derived class
may want to call the base class method directly:
\verb=BaseClassName.methodname(self, arguments)=.
This may be useful to clients as well.
Note that this only works
if the base class is accessible as \verb=BaseClassName= in the global scope.

Python has two built-in functions that work with inheritance:

\begin{itemize}
\item
\verb=isinstance()= checks an instance's type:
\verb=isinstance(obj, int)= will be \verb=True=
only if \verb=obj.__class__=
is \verb=int= or some class derived from \verb=int=.

\item
\verb=issubclass()= checks class inheritance:
\verb=issubclass(bool, int)= is \verb=True=
since \verb=bool= is a subclass of \verb=int=.
However \verb=issubclass(float, int)= is \verb=False=
since \verb=float= is not a subclass of \verb=int=.
\end{itemize}


\subsubsection{Multiple Inheritance}

A class definition with multiple base classes looks like this:\\
\verb=class DerivedClassName(Base1, Base2, Base3):=\\
For most purposes, in the simplest cases,
you can think of the search for attributes inherited from a parent class
as depth-firts, left-to-right, not searching twice in the same class
where there is an overlap in the hierarchy.

In reality it's more complex and requires dynamic ordering.
A class can be subclassed without affecting the precedece order of its parents.

All classes inherit from \verb=object=,
so any case of multiple inheritance exhibits diamond relationships.


\subsection{Private Variables}

\emph{Private} instance variables that cannot be accessed
except from inside an object don't exist in Python.
There's a convention that a name prefixed with an underscore
should be treated as a non-public part of the API
(whether it's a function, a method or a data member).

Since there's a valid use-case for class-private members
(namely to avoid name clashes of names with names defined by subclasses),
there is limited support for such a mechanism, called \emph{name mangling}.
Any identifier of the form \verb=__spam=
(at least two leading underscores, at most one trailing underscore)
is textually replaced with \verb=_classname__spam=,
where \verb=classname= is the current class name
with leading underscore(s) stripped.
This mangling is done
without regard to the syntactic positon of the identifier,
as long as it occurs within the defintion of a class.


\subsection{Odds and Ends}

A data type similar to a C \verb=struct=
can be obtained with an empty class definition,
as in Listing \ref{classes:c-struct}.

\begin{program}
\verbatimtabinput{./py/classes-c-struct.py}
\caption{An empty class can be used as a C struct\label{classes:c-struct}}
\end{program}

Instance method objects have attributes, too:
\verb=m.__self__= is the instance object with the method \verb=m()=,
and \verb=m.__func__= is the function object corresponding to the method.


\subsection{Extensions Are Classes Too}

The \verb=raise= statement has two valid (semantic) forms:\\
\verb=raise Class=\\
\verb=raise Instance=\\
In the first form, \verb=Class= must be an instance of \verb=type=
or of a class derived from it.
The first form is a shorthand for \verb=raise Class()=.

When an error message is printed for an unhandled exception,
the exception's class name is printed, then a colon and a space,
and finally the instance converted to a string
using the built-in function \verb=str()=.


\subsection{Iterators}

Most container objects can be looped over using a \verb=for= statement:\\
\verb=for element in [1, 2, 3]:=\\
\verb=for char in "123":=\\
\verb=for line in open('myfile.txt'):=\\
Behind the scenes,
the \verb=for= statement calls \verb=iter()= on the container object.
The function returns an iterator object
that defines the method \verb=__next__()=
which accesses elements in the container one at a time.
When there are no more elements,
\verb=__next__()= raises a \verb=StopIteration= exception
which tells the \verb=for= loop to terminate.
You can call the \verb=__next__()= method
using the \verb=next()= built-in function.

Adding iterator behavior to your classes is easy.
Define an \verb=__iter__()= method
which returns an object with a \verb=__next__()= method.
If the class defines \verb=__next__()=,
then \verb=__iter__()= can just return \verb=self=.


\subsection{Generators}

Generators are a simple and powerful tool for creating iterators.
They are written like regular functions but use the \verb=yield= statement
whenever they want to return data.
Each time \verb=next()= is called on it,
the generator resumes where it left off
(it remembers all the data values and which statement was last executed).
Listing \ref{classes:generators} has an example.

\begin{program}
\verbatimtabinput{./py/classes-generators.py}
\caption{Generator example\label{classes:generators}}
\end{program}

Anything that can be done with generators
can also be done with class based iterators.
What makes generators so compact
is that the \verb=__iter__()= and \verb=__next__()= methods
are created automatically.
Another key feature
is that the local variables and execution state
are automatically saved between calls.
In addition to automatic method creation and saving program state,
when generators terminate, they automatically raise \verb=StopIteration=.
These features make it easy to create iterators
with no more effort than writing a regular function.


\subsection{Generator Expressions}

Some simple generators can be coded succinctly as expressions
using a syntax similar to list comprehensions
but with parantheses instead of brackets.
Generator expressions are more compact but less versatile
than full generator definitions
and tend to be more memory friendly than equivalent list comprehensions.
See Listing \ref{classes:generator-expressions}.

\begin{program}
\verbatimtabinput{./py/classes-generator-expressions.py}
\caption{Generator expressions\label{classes:generator-expressions}}
\end{program}
