\section{Basics}


\subsection{Strings}

\begin{itemize}
\item
Enclosed in \verb='single'= or \verb="double"= quotes.
Can span multiple lines by ending the line with a backslash (\verb=\=).

\item
Surrounded by a pair of triple quotes
(\verb='''single'''= or \verb="""double"""=).
Can span multiple lines, newlines are included in the string.
Ending a line with a backslash (\verb=\=)
will not include that newline in the string.

\item
\verb=r'Raw'= strings.
No escape sequences are interpreted.
They contain exactly what you type.
\end{itemize}

Two string literals next to each other are automatically concatenated.
Concatenate with \verb='hello '+'world'= and repeat with \verb='hi'*5=.

Strings are immutable.
Strings can be indexed and sliced.
There is no separate character type (just strings of size 1).
Degenerate slice indices (e.g.\ upper too large, upper < lower)
are handled gracefully.
Negative indices count from the right.\\
See: sequence types, string methods, string formatting
linked from the tutorial.


\subsection{Lists}

Concatenate with \verb=[2,3]+[7]=, repeat with \verb=3*[5,4]=.
All slice operations return a new list.
Assignment to slices is possible: can change the list's size or even clear it.


\subsection{Control Flow}

\verb=if ... elif ... elif ... else ...=\\
substitutes \verb=switch=, \verb=case= from other languages.

\verb=for= iterates over any sequence.
It's not safe to modify the sequence being iterated over.
\verb=range()= returns an object that is \emph{iterable}, but not a list.
The \verb=for= statement and the \verb=list()= function are \emph{iterators}.

\verb=break= and \verb=continue= work like in C.\\
\verb=else= clauses on loops
are executed when the loop wasn't terminated by a \verb=break=.

\subsection{Documentation Strings}

The first line should be a short, concise summary of the object's purpose,
not explicitly stating the object's name or type
(unless the name is a verb describing a function's operation).
This line should begin with a capital letter and end with a period.
If there are more lines, the second should be blank.

\subsection{Coding Style}
\verb=CamelCase= for classes
and \verb=lower_case_with_underscores= for functions and methods.
