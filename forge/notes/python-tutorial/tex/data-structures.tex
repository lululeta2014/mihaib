\section{Data Structures}

\verb=collections.deque=\\
List comprehensions provide a concise way to create lists from sequences.

\subsection{The del statement}

\verb=del= can be used to remove an item from a list given its index
(\verb=del a[0]=),
to remove slices from a list (\verb=del a[2:4]=)
or to clear the entire list (\verb=del a[:]=).\\
It can also be used to delete entire variables: \verb=del a=
(referencing the name \verb=a= after this is an error).

\subsection{Tuples and sequences}

Sequence types: str, bytes, bytearray, list, tuple, range.\\
Tuples, like strings, are immutable
(but they may contain mutable objects, such as lists).
Empty tuple is \verb=()=, single-item tuple has trailing comma \verb='hello',=.
Multiple assignment is a combination of tuple packing and sequence unpacking.

\subsection{Dictionaries}

Dictionary keys can be any immutable type
(strings and numbers can always be keys).
Tuples can be used as keys if they contain only strings, numbers, or tuples;
if a tuple contains any mutable object either directly or indirectly,
it can not be used as a key.

The \verb=dict()= constructor builds dictionaries
directly from sequences of key-value pairs (e.g.\ a list of tuples);
when the keys are simple strings,
you can specify pairs using keyword arguments.

\subsection{Looping Techniques}

When looping through dictionaries,
the key and corresponding value can be retrieved at the same time
using the \verb=items()= method.\\
When looping through a sequence,
the position index and corresponding value can be retrieved at the same time
using the \verb=enumerate()= function.\\
To loop over two or more sequences at the same time,
the entries can be paired with the \verb=zip()= function.\\
To loop over a sequence in reverse, use the \verb=reversed()= function.\\
To loop over a sequence in sorted order, use the \verb=sorted()= function
which returns a new sorted list while leaving the source unaltered.

\subsection{Conditions}

Operators \verb=in= and \verb=not in=
check whether a value occurs in a sequence.
Operators \verb=is= and \verb=is not=
compare whether two objects are really the same object.

Operators \verb=and= and \verb=or= are short-circuit.
They return the value of the last evaluated argument.

\subsection{Comparing Sequences and Other Types}

Sequence objects may be compared to other objects with the same sequence type.
The comparison uses lexicographical ordering.\\
Comparing objects of different types with \verb=<= or \verb=>= is legal
provided that the objects have the appropriate comparison methods.
