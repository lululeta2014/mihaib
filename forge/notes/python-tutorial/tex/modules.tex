\section{Modules}

A module is a file containing Python definitions and statements.
The file name is the module name with the suffix \verb=.py= appended.
Within a module, the module's name (as a string)
is available as the value of the global variable \verb=__name__=.
A script executed at the top level, and calculator mode,
run in the \emph{main} module.

Create the file \verb=fibo.py= with some functions.
The command \verb=import fibo= only enters the module name \verb=fibo=
in the current symbol table.
The functions can be accessed as: \verb=fibo.fib(10)=.
They can also be assigned a local name: \verb|fib = fibo.fib|.

A module can contain executable statements as well as function definitions
(function definitions are also statements that are executed:
the execution of a module-level function definition
enters the function name in the module's global symbol table).
These statements are intended to initialize the module.
They are executed only the first time the module is imported somewhere.

Each module has its own private symbol table,
which is used as the global symbol table
by all functions defined in the module.
Thus, the author of a module can use global variables in the module
without worrying about accidental clashes with a user's global variables.
You can touch a module's global variables
with the same notation used to refer to its functions, \verb=modname.itemname=.

Modules can import other modules.
It is customary but not required to place all \verb=import= statements
at the beginning of a module (or script).
The imported module names are placed
in the importing module's global symbol table.

A variant of the \verb=import= statement imports names from a module
directly into the importing module's symbol table:
\verb=from fibo import fib, fib2=.
This does not introduce the module name from which the imports are taken
in the local symbol table (i.e.\ \verb=fibo= is not defined).
The variant \verb=from fibo import *= imports all names
(except those beginning with an underscore \verb=_=).

When you run a Python module with \verb=python fibo.py <args>=
the code in the module will be executed, just as if you imported it,
but with the \verb=__name__= set to \verb="__main__"=.

\subsection{Module Search Path}

When a module named \verb=spam= is imported,
the interpreter searches for a file named \verb=spam.py=
in the current directory, then in a list of directories
specified by the environment variable \verb=PYTHONPATH=,
then in an installation-dependent default path.

Actually modules are searched in the list of directories given by the variable
\verb=sys.path= which is initialized from the above sources.

The built-in \verb=dir()= function is used to find out
which names a module defines: \verb=dir(fibo)=.
It returns a sorted list of strings.
Without arguments, it lists the names you have defined currently.
It does not list the names of built-in functions and variables.
They're defined in the standard module \verb=builtins=.
To get a list of those: \verb=import builtins; dir(builtins)=.

\subsection{Packages}

\verb=A.B= designates a submodule named \verb=B= in a package named \verb=A=.

When importing the package,
Python searches through the directories on \verb=sys.path=
looking for the package subdirectory.
The \verb=__init__.py= files are required
to make Python treat the directories as containing packages.
It can be an empty file,
but it can also execute initialization code for the package
or set the \verb=__all__= variable, described later.

Users of the package can import individual modules from the package:\\
\verb=import sound.effects.echo=\\
loads the submodule \verb=sound.effects.echo=
which must be referenced with its full name.\\
\verb=from sound.effects import echo=\\
also loads the submodule \verb=echo=,
but makes it available without its package prefix.\\
You can also import the desired function or variable directly:\\
\verb=from sound.effects.echo import echofilter=\\
This loads the submodule \verb=echo=
but makes its function \verb=echofilter()= directly available.

When using \verb=from package import item=
the item can be either a submodule (or subpackage) of the package,
or some other name defined in the package,
like a function, class or variable.
The \verb=import= statement
first tests whether the item is defined in the package;
if not, it assumes it is a module and attempts to load it.
If it fails to find it, an \verb=ImportError= exception is raised.

When using \verb=import item.subitem.subsubitem=,
each item except for the last must be a package;
the last item can be a module or a package
but can't be a class or function or variable defined in the previous item.

\subsection{Importing * from a Package}

\verb=from package import *=

If a package's \verb=__init__.py= defines a list named \verb=__all__=,
it is taken to be the list of module names that should be imported
when \verb=from package import *= is encountered.

If \verb=__all__= is not defined, the statement \verb=from package import *=
only ensures that the package has been imported
(possibly running any initialization code in \verb=__init__.py=)
and then imports whatever names are defined in the package.
This includes any names defined (and submodules explicitly loaded)
by \verb=__init__.py=.
It also includes any submodules of the package that were explicitly loaded
by previous \verb=import= statements
(e.g.\ \verb=import package.module=).

\subsection{Intra-package References}

You can use absolute imports to refer to submodules of siblings packages.

You can also use relative imports
with the \verb=from module import name= form of the statement.
Leading dots are used to indicate the current and parent packages.
See \url{http://docs.python.org/py3k/tutorial/modules.html}.

Relative imports are based on the name of the current module.
Since the name of the main module is always \verb="__main__"=,
the main module must always use absolute imports.
