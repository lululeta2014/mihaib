\section{Getting Started}

byte: 8 bit, signed, [-128, 127]
short: 16 bit, int: 32 bit [-2 billion, 2 billion], long: 64 bit
float: 32 bit, double: 64 bit, char: 16 bit unicode character
\verb='\u0000'..'\uffff'=,
boolean: undefined size.\\
String – immutable; special language support via literals.\\
Underscores in numeric literals: \verb=123_456_789L;=
any number of underscores, anywhere between digits.

Arrays:\\
\verb=type[] my array; //discouraged: type array[];=\\
The array size is not part of the type.
Array literal: \verb={1, 2, 3, 4, 5};=.
Multi-dimensional jagged arrays; built-in \verb=.length=.

Null:\\
\verb=null= is not an \verb=instanceof= anything.\\
\verb=null= in \verb=switch= statement condition
throws \verb=NullPointerException=.

\begin{description}
\item[Parameters] list of variables in a method declaration.
\item[Arguments] the values passed in at invocation.
\end{description}

Parameters \emph{shadow} fields with the same name.\\
Primitives are passed by value.\\
Reference parameters are also passed by value; after the method returns,
the passed in reference still references the same object.

\begin{description}
\item[Covariant return type]
overriding a method and making it return a subclass
of the original method's return type.
\end{description}

\verb=this(args)= invokes another constructor from within a constructor.

A constant (\verb|static final int x = 3;|) of primitive or \verb=String= type
might be replaced by the compiler with its value everywhere it's used.
If you change its value, you must recompile all classes that use it.

A class can have any number of \emph{static initialization blocks}.
They are all called in order.
You can have \emph{non-static initializer blocks}.
The compiler copies them into every constructor.
You can share a block of code between multiple constructors this way.

Use asserts.

Nested classes:\\
\emph{static}: static nested classes\\
\emph{non-static}: inner classes

Non-static nested classes (inner classes) have access to other members
of the enclosing class, even if they're declared private.
Static nested classes don't have access to other members
of the enclosing class.

An inner class is associated with an instance of its enclosing class
and has direct access to that object's methods and fields.
Because it's associated with an instance,
it can't define any static members itself.

Static nested class: \verb=new OuterClass.StaticNestedClass();=\\
Inner class: \verb=outerObject.new InnerClass();=\\
local inner class: declared within the body of a method\\
anonymous inner class: local inner class without name

ENUM: a type whose fields consist of a fixed set of constants.
\begin{verbatimtab}
public enum CompassDir {
	NORTH, SOUTH, WEST, EAST
}
\end{verbatimtab}
Enums can contain methods and other fields.
Compiler-added \verb=values()= method returns an array with all enum values.

Annotations

Interfaces: can contain only constants, method signatures and nested types.
All methods are implicitly public.
All constant values are implicitly public, static and final.

\paragraph{Generics}
\verb=Box<T>=;
during compilation, all generic information (\verb=T=) is removed entirely.\\
The diamond \verb=<>=: you can replace the type args for the constructor
with an empty set of type params as long as the compiler can infer them
from the context.

Generic Methods and Constructors:
type params in method/constructor signatures.
\begin{verbatim}
public <U> void f(U u) { ... }
MyClass.f("Hello"); MyClass.<String>f("Hello");
\end{verbatim}

\paragraph{Type Inference}
\begin{verbatimtab}
class MyClass <X> {
	<T> MyClass(T t) {
		// ...
	}
}
\end{verbatimtab}
\verb=new MyClass<Integer>("");= ---
\verb=X=--\verb=Integer=, specified; \verb=T=--\verb=String=, inferred.\\
\verb|MyClass<Integer> myObject = new MyClass<>("");| ---
\verb=X=--\verb=Integer=, inferred; \verb=T=--\verb=String=, inferred.\\
\verb|MyClass<Integer> myObject = new <String>MyClass<>("");| ---
\verb=X=--\verb=Integer=, inferred; \verb=T=--\verb=String=, explicit.

\paragraph{Bounded Type Params}
\verb=<U extends Number>= --- use \verb=&= for additional interfaces.\\
\verb=<U extends Number & MyInterface>=

\paragraph{Subtyping}
\begin{verbatimtab}
public void boxText(Box<Number> n) {
	...
}
\end{verbatimtab}
Cannot pass in \verb=Box<Integer>= or \verb=Box<Double>= to that method.
\verb=Box<Integer>= is not a subtype of \verb=Box<Number>=.

\paragraph{Wildcards}
\begin{description}
\item[upper bound]
\verb=<? extends Animal>=\\
\verb=<?>= is the same as \verb=<? extends Object>=
\item[lower bound]
\verb=<? super Animal>=
\end{description}

\verb=Cage<Lion>= is not a subtype of \verb=Cage<Animal>=,
but it is a subtype of\\
\verb=Cage<? extends Animal>=.

\paragraph{Type Erasure}
The compiler removes all info related to type params and type args.
This enables binary compatibility with pre-generic Java code.\\
\verb=Box<String>= is translated to \verb=Box= --- called the raw type.

Most parameterized types, such as \verb=ArrayList<Number>= and
\verb=List<String>= are non-reifiable types:
they're not completely available at runtime.

\paragraph{Packages}
The compiler auto-imports:
\begin{itemize}
\item the package with no name
\item the \verb=java.lang= package
\item the current package (the package for the current file)
\end{itemize}

\begin{verbatim}
import static java.lang.Math.PI;
import static java.lang.Math.*;
\end{verbatim}
