\section{Attributes and Actions}

Token and rule references have predefined attributes.
Use labels if there is more than one reference to a rule element.
Operator \verb-+=- adds elements to a list.
See Listings \ref{aa:label-attr-simple} and \ref{aa:dynam-scope-simple}.

\begin{program}
\verbatimtabinput{./g/aa/label-attr-simple.g}
\caption{Labels and attributes\label{aa:label-attr-simple}}
\end{program}

\begin{program}
\verbatimtabinput{./g/aa/dynam-scope-simple.g}
\caption{Dynamic attribute scope example\label{aa:dynam-scope-simple}}
\end{program}


\subsection{Token Attributes}

We can also label literals;
that is, we can access the tokens matched for literals:\\
\verb%stat: r='return' expr ';' {System.out.println($r.line);};%\\
Note: in lexer rules, labels on elements are sometimes characters (if length=1)
not Tokens, and evaluate to type int:\\
\verb%R: a='c' b='hi' c=. {$a, $b.text, $c};%

\subsubsection{Predefined Token and Lexer Rule Attributes}
\label{subsub:predef-token-lexer-attr}

\begin{description}
\item[text] \verb=String=.
The text matched for the token; translates to a call to \verb=getText()=.

\item[type] \verb=int=.
The token type (\verb=int > 0=) --- \verb=getType()=

\item[line] \verb=int=.
The line number, counting from 1 --- \verb=getLine()=

\item[pos] \verb=int=.
Character position within the line, counting from 0,
of the Token's first character --- \verb=getCharPositionInLine()=

\item[index] \verb=int=.
Overall index in the Token stream, counting from 0.\\\verb=getTokenIndex()=.

\item[channel] \verb=int=.
The token's channel number

\item[tree] \verb=Object=.
When building trees, the tree node created for the token.
Local variable reference that points to the tree node.
\end{description}


\subsection{Rule Attributes}

The predefined attributes of the currently executing rule are also available:\\
\verb=$attribute= or \verb=$enclosingRuleName.attribute=\\
Predefined attributes are automatically computed
and you should not try to change them.
Exception: the \verb=after= action can set attributes \verb=tree= and \verb=st=
when generating ASTs or templates.

In Lexer rules, a rule that refers to other lexer rules
can access the portion of the overall token that is matched by other rules:\\
\verb|PREPROC_CMD: '#' ID {System.out.println("cmd=" + $ID.text);} ;|\\
The attributes of a lexer rule reference
are the same as a Token reference in a parser grammar,
with the exception that \verb=index= is undefined.

\subsubsection{Predefined Parser Rule Attributes}
\label{subsub:predef-parser-rule-attr}

\begin{description}
\item[text] \verb=String=.
The text matched from the start of the rule
up until the point of the \verb=$text= expression evaluation.

\item[start] \verb=Token=.
The first token to be potentially matched by the rule,
that is on the main token channel.

\item[stop] \verb=Token=.
The last nonhidden channel token to be matched by the rule.
When referring to the current rule,
this attribute is available only to the \verb=after= action.

\item[tree] \verb=Object=.
The AST computed for this rule.
When referring to the current rule, available only to the \verb=after= action.

\item[st] \verb=StringTemplate=.
The template computed for this rule;
for the current rule, only available to the \verb=after= action.
\end{description}

\subsubsection{Lexer Rule Attributes available to expressions}

These attributes are available to attribute expressions
referring to the surrounding rule.

\begin{description}
\item[text] \verb=String=.
The text matched thus far from the start of the token
at the outermost rule nesting level; translated to \verb=getText()=.

\item[type] \verb=int=.
The token type of the surrounding rule,
even if it does not emit a token (because it is invoked from another rule).

\item[line] \verb=int=.
The line number, counting from 1, of this rule's first character.

\item[pos] \verb=int=.
The character position in the line, counting from 0,
of this rule's first character.

\item[channel] \verb=int=.
The default channel number, 0, unless you set it in an action in this rule.
\end{description}

\subsubsection{Predefined Tree Grammar Rule Attributes}

The same as for parser grammar rules
except that input symbols are tree nodes instead of tokens
and \verb=stop= is not defined.
\verb=$ID= refers to the tree node matched for token \verb=ID=,
not to a Token object.

\begin{description}
\item[text] \verb=String=.
The text derived from the first node matched by this rule.
When referring to the current rule, available in any action.
Note: not well defined for rules like \verb=slist: stat+;=
because \verb=stat= is not a single node or rooted with a single node.
\verb=$slist.text= gets only the first \verb=stat= tree.

\item[start] \verb=Object=.
The first tree node to be potentially matched by the rule.
When referring to the crt rule, available to any action.

\item[st] \verb=StringTemplate=.
The template computed for this rule.
When referring to the current rule,
this attribute is available only to the \verb=after= action.
\end{description}

\subsubsection{Rule Parameters}
Visible to the entire rule.
In lexers, only fragment rules can have parameters
bacause they are the only rules you can explicitly invoke in the lexer
(non-fragment rules are implicitly invoked from \verb=nextToken()=).


\subsection{Dynamic Attribute Scopes for Interrule Communication}
\label{subsec:dynam-attrib-scopes}

\subsubsection{Rule Scopes}

\begin{program}
\verbatimtabinput{./g/aa/rule-scopes.g}
\caption{Rule scopes\label{aa:rule-scopes}}
\end{program}

Consider nested blocks of code.
Rule \verb=block= has a dynamic scope with attribute \verb=List symbols;=.
If \verb=block= is invoked recursively (perhaps indirectly),
each invocation pushes a new copy of the attributes on a stack (of scopes).
To access elements other than the top
of the dynamically scoped attribute stack, use syntax \verb=$x[i]::y=
where \verb=x= --- scope name, \verb=y= --- attribute name and
\verb=i= --- absolute index into the stack (0 is bottom).
Expression \verb=$x.size()-1= is the index of the top of the stack.

\subsubsection{Global Scopes}

Each rule defining a dynamic attribute scope has a separate attribute stack.
The previous example had only one place where variables could be defined:
(nested) code blocks.
So the grammar had a single rule, \verb=block=, with a dynamic attribute scope.
If we think of a C-like language with variable declarations
at the topmost level, in functions, and in nested code blocks in functions,
then we have several rules (\verb=prog=, \verb=func=, \verb=block=)
where variables can be defined.

If we'd define a dynamic scope in each rule we'd end up with 3 stacks.
But we need a single stack to properly resolve variable references.

\begin{program}
\verbatimtabinput{./g/aa/global-scopes.g}
\caption{Global scopes\label{aa:global-scopes}}
\end{program}

In general, though, a more sophisticated symbol table is required
(which must persist beyond parser completion).


\subsection{References to Attributes within Actions}

\begin{description}
\item[\$tokenRef]
An expression of type Token,
also useful to test whether a token was matched in an optional subrule:\\
\verb|ID {$ID} (ELSE stat)? {if ($ELSE != null) ..}|

\item[\$tokenRef.attr]
See section \ref{subsub:predef-token-lexer-attr}
for the list of valid attributes.

\item[\$listLabel]
Expression that evaluates to type \verb=List=.
Is a flat list of elements collected thus far by the listLabel.
Only within a parser or tree grammar:\\
\verb|ids+=ID (',' ids+=ID)* {$ids}|

\item[\$ruleRef]
Isolated \verb|$rulename| not allowed in parser or tree grammar unless
the rule has a dynamic scope and there is no reference to \verb=rulename=
in the enclosing alternative, whoch would be amgibuous.
The expression is of type \verb=Stack=. Example:
\verb|$block.size()| checks imbrication level.

\item[\$ruleRef.attr]
Predefined or user-defined attribute of the referenced rule.
See section \ref{subsub:predef-parser-rule-attr}
for the list of available rule attributes.

\item[\$lexerRuleRef]
Within a lexer, expression of type \verb=Token=
containing all predefined properties except \verb=index=.
Can be label or reference to a rule mentioned within this rule.

\item[\$attr]
Return value, parameter or predefined rule property of enclosing rule.

\item[\$enclosingRule.attr]
Fully qualified name of return value, parameter or predefined property.

\item[\$globalScopeName]
Isolated global dynamic scope reference. Example:\\
\verb=$symbols.size()= checks stack size.

\item[\$x::y]
Refer to attribute \verb=y= within the dynamic scope identified by \verb=x=,
which can be a rule scope or a global scope.
The scope prefix is always required.

\item[\$x{[-1]}::y]
Attribute \verb=y= (just under top of stack) of the previous \verb=x= scope.

\item[\$x{[-i]}::y]
Attribute \verb=y= of a previous scope of \verb=x=;
\verb=i= down from the top of stack.
Minus sign \verb=must= be present (not just have a negative \verb=i=).

\item[\$x{[i]}::y]
Attribute \verb=y= of scope \verb=x=, \verb=i= up from bottom of stack;
\verb=i= in \verb=0..size-i=.

\item[\$x{[0]}::y]
Attribute \verb=y= of bottommost scope of \verb=x=.
\end{description}
