\section{Grammars}

ANTLR generates recognizers which apply structure to a stream of input symbols
which can be characters, tokens or tree nodes.
There are four kinds of grammars (with the same syntax):
lexer, parser, tree and combined lexer and parser (no modifier).

The following files are generated from grammar file \verb=T.g=:
\verb=TLexer.java=, \verb=TParser.java=, \verb=T.java= (for tree grammars).
For combined grammars \verb=T__.g= is also generated,
an intermediate temporary file containing the lexer specification
extracted from the combined grammar.
The vocabulary file \verb=T.tokens= is always generated.
It is used by other grammars to keep their token types in sync with \verb=T=.


\subsection{Comments}
Single-line, multi-line and javadoc-style.


\subsection{Literals}
Encosed in single quotes, never contain regular expressions.
For non-ascii characters use unicode escape sequences \verb='\u00E8'=.


\subsection{Action syntax}
Arbitrary text in the target language surrounded by \verb={}=.
To get the \verb=}= character escape it with a \verb=\=:
\verb={System.out.println("\}");}=.


\subsection{Labels}

\verb|e=expr|, \verb|n=INT|, \verb|l='const'|, \verb|l+=ID|,
\verb|(',' l+=ID)* ';'|\\
We may collect Tokens in a List using \verb|l+=ID|.
To do the same for parser rules,
we must use the grammar \verb=output= option (\verb|{output=AST;}|).
In this case \verb|l+=expr| will collect parser rule ASTs
and will be a List of tree nodes (CommonTree) instead of a List of Tokens.
However, we may collect rule return values
without using the \verb=output= option like this:
\verb|(l=expr {$myList.add($l.val);})+|,
where \verb=mylist= can be a local variable or the current rule's return value.


\subsection{Dot wildcard}

\verb=.= Matches a single token in a parser, any single character in a lexer.
In a tree parser, matches \emph{an entire subtree}.
Example: matching a function without parsing the body:
\verb|func: ^(FUNC ID ARGS .);|.
The wildcard skips the entire last child (the body subtree).


\subsection{Subrules}

Allow all the rule-level options and have the syntax:\\
\verb|(options{«option-assignments»}: «subrule-alternatives»)|\\
The only addition is the \verb=greedy= option, useful for nondeterminisms
between subrule alternatives and the subrule exit branch.
They occur when the parser can't decide which path to take
because an input symbol predicts taking multiple paths.
\emph{Greedy} means: match as much input as possible.
\emph{Nongreedy} means the parser exits
as soon as it sees input consistent with what follows the subrule.

The default is to always be greedy.
The only exceptions are \verb|.*| and \verb|.+|
which are non-greedy by default.


\subsection{Actions Embedded within Rules}

To execute actions before anything else in a rule and define local variables,
use an \verb=@init= action.
To execute something after any alternative has been executed
and right before the rule returns,
use an \verb=@after= action.

\begin{program}
\verbatimtabinput{./g/action-init-after.g}
\caption{Actions @init and @after\label{gram:action-init-after}}
\end{program}


\subsection{Rule Arguments and Return Values}

ANTLR parser and tree parser rules can have arguments and return values.
Lexer rules can't have return values,
and only fragment lexer rules can have parameters.
Rules may return multiple values.

\begin{program}
\verbatimtabinput{./g/rule-args-retvals.g}
\caption{Rule arguments and return values\label{gram:rule-args-retvals}}
\end{program}


\subsection{Dynamic Rule Attribute Scopes}

Besides predefined rule attributes, rules can define scopes of attributes
that are visible to all rules invoked by a rule.
Down in a deeply nested \verb=expr= rule,
we can directly access the method's name
without having to pass it all the way down to that rule from \verb=method=.
If the language has nested method definitions,
each method definition gets its own name.
Upon entry to \verb=method=, the old name is pushed onto a stack;
upon exit, it's popped off.

\begin{program}
\verbatimtabinput{./g/dynamic-rule-attr-scope.g}
\caption{Dynamic rule attribute scopes\label{gram:dynamic-rule-attr-scope}}
\end{program}


\subsection{Rewrite Rules}

ANTLR parsers can generate ASTs or StringTemplate templates.
Specify the \verb=output= option \verb=AST= or \verb=template=.
If so, all rules have an implied return value
that is set manually in an action or in a rewrite rule.
Every alternative, whether in a subrule or at the outermost rule level,
can have a rewrite rule.
Regardless of location,
the rewrite rule \emph{always} sets the return object for the entire rule.\\
\verb_unaryID: '-' ID -> ^('-' ID);_


\subsection{Rule exception handling}
ANTLR catches the exception, reports the error, attempts to recover,
then returns from the rule.
To disable ANTLR's catch clause and write your own,
see Listing \ref{gram:rule-exception-handling}.

\begin{program}
\verbatimtabinput{./g/rule-exception-handling.g}
\caption{Rule exception handling\label{gram:rule-exception-handling}}
\end{program}


\subsection{Syntactic Predicates}

Indicate the syntactic context that must be satisfied
if an alternative is to match.
Amounts to specifying the lookahead language for an alternative.
In general a parser needs to backtrack
over the elements within a syntactic predicate
to properly test the predicate against the input stream.
Alternatives predicated with syntactic predicates
are attempted in the order specified.
The first alternative that matches wins.

\begin{program}
\verbatimtabinput{./g/syn-pred-simple.g}
\caption{Simple syntactic predicate\label{gram:syn-pred-simple}}
\end{program}

The last alternative in a series of predicated alternatives
does not need a predicate,
because it's assumed to be the default if nothing else before it matches.
Alternatives which are not mutually ambiguous do not need syntactic predicates.
We can add some more alternatives to \verb=stat=, and both ways of writing it
shown in Listing \ref{gram:syn-pred-non-ambig-any-order} are good.

\begin{program}
\verbatimtabinput{./g/syn-pred-non-ambig-any-order.g}
\caption{Alternatives not mutually ambiguous %
do not need syntactic predicates and may be given in any order%
\label{gram:syn-pred-non-ambig-any-order}}
\end{program}

As a convenience and to promote clean looking grammars,
there is the \verb=backtrack= option.
It tells ANTLR to automatically insert syntactic predicates where necessary
to disambiguate decisions that are not LL(*).
An equivalent version of stat is in Listing \ref{gram:rule-backtrack-option}.

\begin{program}
\verbatimtabinput{./g/rule-backtrack-option.g}
\caption{An equivalent version of `stat' using the rule's backtrack option%
\label{gram:rule-backtrack-option}}
\end{program}


\subsection{Lexical Rules}

To distinguish between token type definitions
and methods generated from lexer rules,
generated methods are prefixed with `m'.
Lexer rule \verb-ID- generates method \verb-mID()-.
Unlike a parser grammar, there is no start symbol.

\subsubsection{Fragment Lexer Rules}

ANTLR assumes all lexer rules are valid tokens.
A \verb=fragment= rule can only be called by other rules;
it will not yield a token to the parser.

\begin{description}
\item[Token definition rules]
may have no parameters or return values.
They aren't invoked explicitly in general,
just implicitly by \verb=nextToken()=.

\item[fragment rules]
are never implicitly invoked.
Other lexer rules must explicitly invoke them.
They can define parameters, but not return values
(lexer rules always return \verb=Token= objects).
\end{description}

\begin{program}
\verbatimtabinput{./g/fragment-lexer-param.g}
\caption{Fragment lexer rule with parameter\label{gram:fragment-lexer-param}}
\end{program}

\subsubsection{Recursive Lexer Rules}

ANTLR supports recursive lexer rules.
Lexer rules can call other lexer rules,
but that doesn't change the token type of the invoking rule.
When lexer rule \verb=T= calls another lexer rule or fragment rule,
the return type for \verb.T. is still \verb.T.
(not the token type of the other lexer rule).

\subsubsection{Ignoring Whitespace and Comments}

See Listing \ref{gram:ignore-whitespace}.

\begin{program}
\verbatimtabinput{./g/ignore-whitespace.g}
\caption{Ignoring whitespace\label{gram:ignore-whitespace}}
\end{program}

\subsubsection{Emitting More Than One Token per Lexer Rule}
In the Lexer, the current char position within the current line
is always available via \verb=getCharPositionInLine()=.
When parsing Python code,
we need to emit additional imaginary \verb=INDENT= and \verb=DEDENT= tokens.
See pg.\ 109 in the book for a short discussion.


\subsection{Tree Matching Rules}

Syntax: \verb_^(root child1 child2 .. childn)_\\
Tree grammars look like parser grammars
with the addition of a few tree expressions.
A tree grammar reduces to a parser grammar when the input is a flat tree
(linked list).
Tree grammar rules can have parameter and return values
just like parser grammar rules.
Actions can contain attribute references.
The difference is that \verb=$T= yields pointers to the tree node matched.

\begin{program}
\verbatimtabinput{./g/tree-rule-simple.g}
\caption{Tree matching rule example\label{gram:tree-rule-simple}}
\end{program}


\subsection{Rule Options}

\verb=backtrack=, \verb=memoize=, \verb=k=\\
These can be specified for each rule.
The default is to use the grammar's
\verb=backtrack=, \verb=memoize=, \verb=k= options.


\subsection{Tokens Specification}

Introduce imaginary tokens or give better names to token literals.
See Listings \ref{gram:tokens-specification} and \ref{gram:tokens-simple}.

\begin{program}
\verbatimtabinput{./g/tokens-specification.g}
\caption{Tokens specification\label{gram:tokens-specification}}
\end{program}

\begin{program}
\verbatimtabinput{./g/tokens-simple.g}
\caption{Imaginary token example\label{gram:tokens-simple}}
\end{program}


\subsection{Global Dynamic Attribute Scopes}

ANTLR lets you define global attribute scopes.
These are visible to all actions in all rules.
Their syntax is shown in Listing \ref{gram:glob-scope-syntax}.
The recognizer pushes a new entry onto a stack of scopes
upon entry to each method that declares scope \verb=SymbolScope=.
See Subsection \ref{subsec:dynam-attrib-scopes}
for more about dynamic attribute scopes.

\begin{program}
\verbatimtabinput{./g/glob-scope-syntax.g}
\caption{Global dynamic attribute scopes\label{gram:glob-scope-syntax}}
\end{program}


\subsection{Grammar Actions}

ANTLR generates a method for each rule;
all methods are wrapped in a class definition (for OO target languages).
ANTLR provides named actions so you can insert fields and instance methods
in the generated class definition.
The syntax is shown in Listing \ref{gram:grammar-actions-syntax}.
The \emph{action-name} can be \verb=header= or \verb=members=.
The \emph{action-scope-name} can be
\verb=lexer=, \verb=parser= or \verb=treeparser=.
In a combined grammar,
\verb=@header= is a shorthand for \verb=@parser::header=.

\begin{program}
\verbatimtabinput{./g/grammar-actions-syntax.g}
\caption{Grammar actions\label{gram:grammar-actions-syntax}}
\end{program}


\subsection{Grammar-Level Options}

These options affect all the elements in the grammar,
unless you override them in a rule.
The syntax is shown in Listing \ref{gram:grammar-options-syntax}.
Specify lexer rules in order of priority if two rules can match the same input.

\begin{program}
\verbatimtabinput{./g/grammar-options-syntax.g}
\caption{Grammar options\label{gram:grammar-options-syntax}}
\end{program}

\begin{description}
\item[language]
Target language in which recognizer should be generated
\item[output]
\verb=template= or \verb=AST=; for combined, parser and tree grammars.
The default is to generate nothing.
\item[backtrack]
false by default
\item[memoize]
false by default
\item[tokenVocab]
needed when one grammar uses the token types of another
\item[rewrite]
when output is very similar to input, modify the input buffer in place;
works in conjunction with \verb|output=template|.
When \verb_rewrite=true_, the recognizer replaces the input matched by the rule
with the template. The default is \verb-false-.
\item[superClass]
default \verb=Lexer=, \verb=Parser= or \verb=TreeParser=.
\item[filter]
lexer only, useful for \emph{fuzzy parsing}
(matching only those constructs of interest).
All lexer rules are tried in the order specified, looking for a match.
Upon finding a match, \verb=nextToken()= returns that rule's Token object.
If no rule matches,
the lexer consumes a single character and again looks for a matching rule.
The default is \verb=false= (do not filter).
\item[ASTLabelType]
default is \verb=Object=.
\item[TokenLabelType]
Set type for all token labels and all token valued expressions.
Default is \verb=Token=.
\item[k]
turn off LL(*) in favor of LL(k);
default value is \verb=*= to engage LL(*).
\end{description}
