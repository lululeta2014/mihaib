\section{Quick Tour}

\subsection{Arithmetic Expressions Pattern}

A series of rules, one for each operator precedence level,
and one for the lowest level describing atoms.
Start with rule \verb=expr= representing a complete expression
which matches operators with the weakest precedence
and refers to a rule that matches subexpressions
for operators with the next highest precedence.

\begin{program}
\verbatimtabinput{./g/arithm-expr-pattern.g}
\caption{Arithmetic expressions pattern\label{tour:arithm-expr-pattern}}
\end{program}


\subsection{Using Syntax to Drive Action Execution}
Syntax drives the ealuation of actions in the parser.
To execute actions for a particular construct,
place them in the appropriate grammar alternative.


\subsection{Evaluating Expressions using an AST}

Build a tree, then walk it with a tree parser and execute actions.
AST $\neq$ Parse Tree
(a parse tree is the sequence of rule invocations to match an input stream).
To build an AST with a grammar, add AST construction rules to the grammar:\\
\verb!options { output = AST; ASTLabelType = CommonTree; }!\\
Now all rules return a node or a subtree;
the starting rule gives us the complete AST.


\subsection{Tree Grammars}

ANTLR encodes 2D tree structure in a 1D stream of tree nodes
using additional imaginary nodes: UP and DOWN.
A depth first traversal of the AST for the expression \verb=3 + 4 * 5= is:
\verb=+ DOWN 3 * DOWN 4 5 UP UP=.
Copy-paste the rewrite rules from the parser grammar
and use them to recognize the AST structure in the tree grammar.

\begin{program}
\verbatimtabinput{./g/tree-grammar-header.g}
\caption{Tree grammar header\label{tour:tree-grammar-header}}
\end{program}


\subsection{Note about \$ in actions}

Label assignments from a grammar alternative, e.g.\ \verb+e=multExpr+,
are replaced with \verb+e=multExpr();+ in the generated (e.g.\ parser) code.
Grammar actions involving labels, e.g.\ \verb=$e.value=,
are replaced in the generated code
with \verb=e= (for rules with a single return value)
or \verb=e.value= (for rules with multiple return values).

The \verb=$= sign tells ANTLR to interpret what follows it
and replace that appropriately when generating (e.g.\ parser) code.
Parts of a grammar action with no \verb=$= sign
are inserted unchanged in the generated code.
This is why we sometimes \emph{get away with}
writing \verb=e= instead of \verb=$e.value= in a grammar action
(for rules with a single return value).
It just happens to be the destination code that ANTLR would have generated.
The explicit \verb=$= form should always be used.
