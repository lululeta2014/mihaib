\section{Trees}

Many problems cannot be solved with a single pass over the input.
Rather than repeatedly rescanning the characters
and reparsing the token stream,
construct and walk a condensed version of the input.
This is called an \emph{intermediate form}
and is usually a tree data structure.
ASTs are not parse trees
(ASTs contain only the nodes associated with input symbols).


\subsection{Proper AST Structure}

\begin{itemize}
\item
Record the meaningful input tokens (and only the meaningful ones).

\item
Encode, in the 2D tree, the grammatical structure used by the parser
(but not the rule names themselves).

\item
Be easy for the computer to recognize and navigate.
\end{itemize}

Computers deal with programs most naturally as simple instruction streams.
Trees are much more convenient to examine and manipulate
than low-level machine code.

Encoding arithmetic expressions as a tree is straightforward.
But for more abstract language constructs
we can invent some high-level pseudomachine instructions.
Each subtree should be like an imperative command in English
with a verb and object:\\
int i; $\rightarrow$ \verb|^(VARDEF int i)|\\
array indexing $\rightarrow$ \verb=^(INDEX a i)=\\
function call $\rightarrow$ \verb=^(CALL a i)=\\
Fortran uses \verb=a(i)= for both array indexing and function calls,
so the parser grammar will need semantic predicates
before building the AST nodes.
An \verb=EXPR= node can be added on top of all expressions:
computation results and other information can be stored there.


\subsection{Default AST construction}

To create ASTs, set option \verb=output= to \verb=AST=.
By default ANTLR builds a flat tree (a linked list)
with pointers to all the input tokens.


\subsection{Constructing ASTs using Operators}

\verb=expr: INT ('+'^ INT)*;=

\begin{description}
\item[!]
Do not include node or subtree (if referencing a rule) in rule's tree.
\item[\textasciicircum]
Make node root of subtree created for entire enclosing rule.
If the suffixed element is a rule reference,
that rule must return a single node, not a subtree.
The result must become a root node.
\end{description}

AST construction operators work great for left-associative operators
such as plus and multiply.
To handle right-associative arithmetic operators
use tail recursion to the enclosing rule
(see pg.\ 275 in the book for Arithmetic Expression Grammars):\\
\verb=pow: INT ('^'^ pow)?;=


\subsection{Constructing ASTs with Rewrite Rules}

\begin{verbatimtab}
rule	: «alt1» -> «build-this-from-alt1»
	| «alt2» -> «build-this-from-alt2»
	...
	| «altN» -> «build-this-from-altN»
	;
\end{verbatimtab}

Each rule returns a single AST (available as a predefined attribute)
that you can set zero or more times using the \verb=->= operator;
generally you will use the rewrite operator once per rule invocation.\\
\verb=r: e {System.out.println($e.tree);};=

Parser grammars specify how to recognize input tokens.
Rewrite rules are generational grammars that specify how to generate trees.
Rewrite rules may omit input elements, reorder them
or make some elements the root of others:\\
\verb=decl: 'var' ID ':' type -> ^('var' type ID);=

To create an imaginary node in the AST, refer to its token type.
ANTLR will create a Token object with that token type
and make it the payload of the new tree node.
An imaginary token reference is a token reference
with no corresponding token reference
on the left side of the \verb=->= operator:\\
\verb=decl: type ID ';' -> ^(VARDEF type ID);=

\paragraph{Collecting input elements and emitting together}
\begin{verbatimtab}
list: ID (',' ID)* -> ID+;
decl: 'int' ID (',' ID)* -> ^('int' ID+);
\end{verbatimtab}

\paragraph{Duplicating nodes and trees}
ANTLR will make duplicates
(otherwise, \verb=^(ID ID)= would create a cycle in the tree):\\
\verb=// duplicate tree returned from type=\\
\verb=decl: type ID (',' ID)* -> ^(type ID)+;=

\paragraph{Choosing between tree structures at runtime}
List the multiple rewrite rules with a semantic predicate in front.
The predicates are tested in the order specified.
The rewrite rule of the first true predicate generates the rule's return tree.
You may specify a default rewrite as the last unpredicated rewrite.
\begin{verbatimtab}
varDef	: modifiers type ID ('=' expr)? ';'
	-> {inMethod}?	^(VARIABLE ID modifier* type expr?)
	->		^(FIELD ID modifier* type expr?)
	;
\end{verbatimtab}

\paragraph{Referring to labels in rewrite rules}
Using \verb=ID+= in the rewrite rule yields a list of all \verb=ID= tokens
matched on the left side.
To mark the first method as the main one:\\
\verb|prog: main=method others+=method* -> ^(MAIN $main) $others*;|

\paragraph{Creating nodes with arbitrary actions}
Cannot be suffixed with \verb=+= and \verb=*=,
can access trees created elsewhere in a grammar.\\
\verb=a: INT->{new CommonTree(new CommonToken(FLOAT, $INT.text+".0"));};=\\
\verb=typeDefinition: modifiers! classDefinition[$modifiers.tree];=\\
\verb=classDefinition [CommonTree mod]=\\
\verb=    : ... -> ^('class' ID {$mod} ...);=

\paragraph{Rewrite rule element cardinality}
If an operator (like \verb=+=) suffixes a tree,
and at least one element has cardinality \verb=n= greater than 1,
then \verb=n= trees are generated.
All elements with cardinality greater than 1
must have exactly the same cardinality.
The elements with cardinality 1 are duplicated as the parser creates the tree.
Imaginary nodes do not force tree construction.
At least one real element within the tree must have cardinality greater than 0.
\\
\verb=initValue: expr? -> ^(EXPR expr)?;=\\
\verb=^(EXPR expr)?= yields no tree when \verb=expr= on the left side
returns no tree,
while \verb=^(EXPR expr?)= always yields at least an \verb=EXPR= node.

\paragraph{Rewrite Rules in Subrules}
Even when a rewrite rule is not at the outermost level in a rule,
it still sets the rule's result AST:
\begin{verbatimtab}
decl	: type
	(ID '=' INT	-> ^(DECL_WITH_INIT type ID INT)
	|ID		-> ^(DECL type ID)
	);
\end{verbatimtab}

\paragraph{Referencing previous rule ASTs in rewrite rules}
Sometimes, executing a single rewrite after the parser has matched everything
is insufficient.
We may need to iteratively build up the AST.
To reference the previous value of the current rule's AST,
use \verb=$r= within a rewrite rule where \verb=r= is the enclosing rule:
\begin{verbatimtab}
expr: INT ('+'^ INT)*;
// is equivalent to
expr: (INT -> INT) ('+' i=INT -> ^('+' $expr $i))*;
\end{verbatimtab}

\paragraph{Deriving imaginary nodes from real Tokens}
Imaginary nodes, by default, have no line and column information
or token index.
ANTLR allows creating imaginary nodes with a constructor-like syntax,
to derive them from existing tokens.\\
\verb|compoundStat: lc='{' stat* '}' -> ^(SLIST[$lc] stat*);|\\
The \verb=SLIST= node gets line and column information
from the left curly's information.
We can also set the text to something more appropriate:\\
\verb=SLIST[$lc, "statements"]=

\begin{verbatimtab}
Imaginary Node Constr.	Tree Adapter Invocation
T			adaptor.create(T, "T")
T[]			adaptor.create(T, "T")
T[token-ref]		adaptor.create(T, token-ref)
T[token-ref, "text"]	adaptor.create(T, token-ref, "text")
\end{verbatimtab}

\paragraph{Automatic AST construction}
By default, \verb-output=AST- causes each parser rule to build up a list
of the nodes and subtrees created by its elements.
Rewrite rules turn this off.
In may cases, we might want to combine the approaches:
\begin{verbatimtab}
primary	: INT
	| FLOAT
	| '(' expression ')' -> expression
	;
\end{verbatimtab}


\subsection{Tree Grammars}

Build a tree grammar by copy-pasting the parser grammar,
leaving only its rewrite rules.
Parser rules that have AST rewrite rules within subrules also translate easily,
because rewrite rules always set the rule's result tree.
If the parser rule decides at runtime which structure to build
using a semantic predicate,
the tree grammar lists all alternatives.
The AST construction operators are trickier.

\begin{verbatimtab}
tree grammar treeGrammarName;

options {
	tokenVocab = parserGrammarName;
	ASTLabelType = CommonTree;
}

CommonTreeNodeStream nodes = new CommonTreeNodeStream(tree);
// nodes.setTokenStream(tokens);
treeGrammarName walker = new treeGrammarName(nodes);
walker.start_symbol();
\end{verbatimtab}

All expression rules from the parser grammar
collapse into a single recursive rule in the tree grammar.
\begin{verbatimtab}
expr	: ^('+' expr expr)
	| ^('*' expr expr)
	| ID
	| INT
	;
\end{verbatimtab}
