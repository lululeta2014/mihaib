\section{Building Interpreters}


\subsection{Building High-Level Interpreters}

High-level interpreters directly execute source code instructions
or the AST equivalent.
Low-level interpreters execute bytecodes close to CPU machine instructions.
These high-level patterns are better suited to DSLs
rather than general-purpose programming languages.
Usually DSLs need simplicity and low-cost implementation
more than execution efficiency.
We'll focus on simple implementations and use dynamically typed languages.

\subsubsection{Designing High-Level Interpreter Memory Systems}
High-level interpreters store values according to variable names
not memory addresses (like low-level interpreters and CPUs do).
Three kinds of memory spaces for most programming languages:
\begin{itemize}
\item global memory
\item function spaces (params and locals)
\item data aggregate instances (structs or objects)
\end{itemize}

For simplicity we can normalize all of them by treating them as dictionaries.
Fields are variables stored within the instance's space.
A memory space is the run-time analog of a scope from static analysis.

Interpreters have one global memory space but multiple function spaces.
Function spaces are kept on a stack.
We store references to aggregate instances just like any other variable.
To handle class instances instead of structs,
the easiest thing to do is pack all fields (inherited and direct)
into a single instance space.

Function spaces can point at their function definition symbols,
and instances at their class definition symbols.

\subsubsection{Tracking Symbols in High-Level interpreters}

\subsubsection{Processing instructions}
The basic idea is the \emph{fetch-decode-execute} cycle.


\subsection{P24 Syntax-Directed Interpreter}

Directly executes source code without building an IR.
Only two key components:
\begin{itemize}
\item the source code parser recognizes input constructs
and immediately triggers actions.
\item the interpreter maintains state
and houses instruction implementation methods.
\end{itemize}

Only suitable to a narrow range of languages.
Interpreting \verb=if= statements, loops, functions and classes
is extremely awkward.


\subsection{P25 Tree-Based Interpreter}
Executes programs by constructing an AST from the source code
and walking the tree.
Builds a complete scope tree before executing a program.
Can support both statically typed and dynamically typed languages.
It can resolve all symbols statically before execution.

The biggest difference from P24
is that we don't drive the interpreter with the parser.
We build an AST with the parser
and then drive the interpreter with a tree visitor.
Thus:
\begin{itemize}
\item we can separate symbol definition from symbol resolution;
this allows forward references
\item tree based interpreters are more flexible ---
we can do substitutions in the AST, rewrite it for optimizations and so on.
\end{itemize}
