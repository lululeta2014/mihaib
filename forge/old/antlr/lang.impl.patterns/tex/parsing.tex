\section{Parsing}


\subsection{Recursive-Descent Parsers}

Don't need to actually build parse trees ---
can trace them out implicitly via a function call sequence (a call tree).
Each function executes code to match its children.
To match a substructure (subtree) it calls the function for that subtree.
To match token children, it calls \verb=match("+")=, \verb=match("return")=.
Four classic parsing patterns follow.

LL(1) --- Top-Down Parser with 1 lookahead token.
The first `L' stands for \emph{read input from left to right}.
The second for \emph{descend into parse tree children from left to right}.

Substructures in the parse tree $\approx$ functions in parser
$\approx$ rules in grammar.


\subsection{P1 Mapping grammars to recursive descent recognizers}

Problems: left recursion and nondeterminisms (doesn't know which path to take).
From a grammar $\rightarrow$ make a class with a method for each rule.
Rule references become method calls: rule \verb=r= $\rightarrow$ \verb=r()=.
Token references for token \verb=T= $\rightarrow$ \verb=match(T)=
(throws Exception for mismatch).\\
Subrules: \verb=(alt1 | alt2 | .. | altN)= see Listing \ref{parse:subrules-if}.

\begin{program}
\verbatimtabinput{./code/parsing/subrules-if}
\caption{Subrules translated to if statements\label{parse:subrules-if}}
\end{program}

We may use switch statements (each alternative is predicted by some tokens).
See Listing \ref{parse:subrules-switch}.
If \verb=alt1= has something like \verb=(A|B|C)=
we may put them in a Set instead of many \verb=case= statements.

\begin{program}
\verbatimtabinput{./code/parsing/subrules-switch}
\caption{Implementing subrules with switch\label{parse:subrules-switch}}
\end{program}

The nature of the lookahead prediction expressions
dictates the strength of the strategy.
P2 and P3 are LL(1) $\rightarrow$ they test 1 symbol of lookahead.
P4 is LL(k) $\rightarrow$ tests k symbols of lookahead.
P5 (Backtracking) and P7 (Predicated) augment LL(k) decisions
with arbitrary amount of lookahead and arbitrary user defined runtime tests,
respectively.

Converting subrule operators: \verb=?=, \verb=+=, \verb=*=.
\begin{description}
\item[?]
Remove the error clause
(\verb=else= or \verb=default= from the previous Listings).

\item[+]
Put the code in a do-while block:
\verbatimtabinput{./code/parsing/plus-do-while}

\item[*]
Put the code in a while loop:
\verbatimtabinput{./code/parsing/star-while}

\end{description}


\subsection{P2 LL(1) Recursive Descent Lexer}

Emits a sequence of tokens. Each token has 2 attributes:
token type and associated text.
Follow pattern 1 to write code for lexer rules.
To make the lexer look like a sequence of tokens, define \verb=nextToken()=.
\verb=nextToken()= uses the lookahead character to route control flow
to the appropriate recognition method.
Very similar to P3 (both are generated using P1).


\subsection{P3 LL(1) Recursive Descent Parser}

Analyze syntactic structure of the token sequence of a phrase
using a single lookahead token.
Each alternative has a set of tokens that can begin it $\rightarrow$
lookahead set.
The parser tests the current lookahead token
against the alternatives' lookahead sets.

\paragraph{Computing Lookahead Sets}
Formally, FIRST and FOLLOW are used.
If an alternative begins with a token reference,
that token is the lookahead set.
If an alternative begins with a rule reference,
the lookahead set is whatever begins any alternative of that rule.

Things get complicated with empty alternatives.
For empty alternatives, the lookahead set is FOLLOW.
We need to see where the current rule is used in other rules,
and figure out what tokens can follow it.

We need lookahead sets predicting alternatives to be disjoint.
See Listing \ref{parse:left-factor-LL1}.

\begin{program}
\verbatimtabinput{./code/parsing/left-factor-LL1}
\caption{Left factoring to obtain LL(1) grammar\label{parse:left-factor-LL1}}
\end{program}


\subsection{P4 LL(k) Recursive Descent Parser}


\subsection{P5 Backtracking Parser}

For arbitrary lookahead, we need infrastructure to support backtracking.
Backtracking also gives us a way to specify
precedence of ambiguous alternatives.
Backtracking parsers, by definition, try the alternatives in order.

\verbatimtabinput{./code/parsing/backtracking-code}

Exceptions are used to guide the parse.
Next we'll see how to avoid unnecessary reparsing
by recording partial parsing results.


\subsection{P6 Memoizing Parser}

Records partial parsing results during backtracking
to guarantee linear parsing performance
at the cost of a small amount of memory.
Memoization only helps us
if we invoke the same rule at the same input position more than once.

A rule can avoid parsing if it succeeded the last time at that position.
Just pretend to parse by skipping ahead and returning.
It can also avoid parsing if it failed the last time.


\subsection{P7 Predicated Parser}
Augments any top-down parser with arbitrary boolean expressions
that help make parsing decisions.
Semantic predicates.
Predicates that evaluate to false turn off a parser decision path.
Just like converting subrules in P1,
with the boolean test after the lookahead test:

\verbatimtabinput{./code/parsing/predicated-code}

Predicated loop decisions for \verb=(..)+= and \verb=(..)*= look like:
\verbatimtabinput{./code/parsing/predicated-plus-star}
