\section{Symbols}


\subsection{Tracking and Identifying Program Symbols}

\paragraph{Collecting Information about Program Entities}
\begin{itemize}
\item Name
\item Category (e.g.\ class, method, variable, label)
\item Type
\end{itemize}

A symbol table implements each symbol category with a separate class,
holding \verb=name= and \verb=type= properties.
Common \verb=Symbol= superclass and the simplest program entity category,
\verb=VariableSymbol=:

\begin{verbatimtab}
public class Symbol {
	public String name;
	public String type;
}

public class VariableSymbol extends Symbol {
	public VariableSymbol(String name, Type type) {
		super(name, type);
	}
}
\end{verbatimtab}

User-defined types like classes and structs can be represented,
for consistency, like any other program symbol.
We can derive \verb=BuiltinTypeSymbol= and \verb=ClassSymbol=
from \verb=Symbol= (though they don't need the \verb=type= field).
To distinguish them from other program symbols,
it's a good idea to tag types with a \verb=Type= interface.
Here's a class representing built-in types like \verb=int= and \verb=float=:

\begin{verbatimtab}
public class BuiltinTypeSymbol extends Symbol implements Type {
	public BuiltinTypeSymbol(String name) { super(name); }
}
\end{verbatimtab}

The \verb=Type= interface is used as a tag (a role that classes can play).
All categories would subclass \verb=Symbol=,
and some would implement \verb=Type=.\\
\verb=public interface Type { public String getName(); }=

\paragraph{Grouping Symbols into Scopes}
To represent scopes we'll use an interface
so we can tag entities like functions and classes as scopes.
Scopes have pointers to their enclosing scopes.
Scopes don't need to track their code region;
instead the AST for the code region points to the scope:
because we'll look up symbols in scopes based on what we find in the AST nodes.

\begin{verbatimtab}
public interface Scope {
	public String getScopeName();
	public String getEnclosingScope();
	public void define(Symbol sym);
	public Symbol resolve(String name);
}
\end{verbatimtab}

\paragraph{Monolithic Scopes}
If we have a single global scope,
we can use a dictionary that maps names to Symbol objects.

\paragraph{Multiple and Nested Scopes}
To track nested scopes, we push and pop them onto a scope stack.
To keep everything around, we'll use a \emph{scope tree} (a tree of scopes)
that can function like a collection of stacks.
Instead of pushing a scope onto a stack,
we're going to add a child to a scope tree.
Building a scope tree means executing a sequence of
\verb=push=, \verb=pop= and \verb=def= operations:

\begin{description}
\item[push]
At the start of a scope, push a new scope on the scope stack.
More like an \emph{addChild()} tree operation.\\
\verb+currentScope = new LocalScope(currentScope);+

\item[pop]
At the end of a scope, pop it off the stack.
Moves the current scope pointer up one level in the tree.\\
\verb+currentScope = currentScope.getEnclosingScope();+

\item[def]
Define a symbol in the current scope:\\
\verb+Symbol s = «some-new-symbol»;+\\
\verb+currentScope.define(s);+
\end{description}

\paragraph{Resolving Symbols}

\begin{description}
\item[Semantic context] means a reference's scope stack
(from its enclosing scope to the root of the scope tree).
\end{description}

\begin{verbatimtab}
public Symbol resolve(String name) {
	Symbol s = members.get(name);
	if (s != null) return s;
	if (enclosingScope != null)
		return enclosingScope.resolve(name);
	return null;
}
\end{verbatimtab}

Rather than making a clever algorithm that implements all of the lookup rules,
we assemble a handy data structure that encodes them.
Four key abstract operations:
\verb=push=, \verb=pop= and \verb=def= to construct scope trees
and \verb=ref= to resolve references in the scope tree.


\subsection{P16 Symbol Table for Monolithic Scope}

Suitable for simple programming languages (without functions),
configuration files, other small DSLs.
We don't need to push and pop scopes,
but we'll do so for consistency with other patterns.

\begin{description}
\item[Start of file]
\verb=push= a GlobalScope.
\verb=def= BuiltinType objects for any built-in types.

\item[Declaration x]
\verb=ref= x's type (if any). \verb=def= x in the current scope.

\item[Reference x]
\verb=ref= x starting in the current scope.

\item[End of file]
\verb=pop= the GlobalScope.
\end{description}

Example:
\begin{verbatimtab}
		push global scope, def int, def float
int i = 9;	ref int, def i
float j;	ref float, def j
int k = i + 2;	ref int, def k, ref i
		pop global scope
\end{verbatimtab}


\subsection{P17 Symbol Table for Nested Scopes}

MethodSymbol is both a Symbol and a Scope.
Methods have two scopes: one for parameters (the MethodSymbol)
and one for local variables (a LocalScope whose parent is the MethodSymbol).
A LocalScope is also used for nested code blocks.

\begin{description}
\item[Start of file]
\verb=push= a GlobalScope. \verb=def= BuiltinType objects for int, float, void.

\item[Variable declaration x]
\verb=ref= x's type. \verb=def= x in the current scope.

\item[Method declaration f]
\verb=ref= f's return type.
\verb=def= f in the current scope and \verb=push= it as the current scope.

\item[\{]
\verb=push= a LocalScope as the new current scope.

\item[\}]
\verb=pop=, revealing previous scope as the current scope

\item[End of method]
\verb=pop= the MethodSymbol scope (the parameters).

\item[Reference x]
\verb=ref= x starting in the current scope.
If not found, look in the immediately enclosing scope, if any.

\item[End of file]
\verb=pop= the GlobalScope.
\end{description}

Build an AST and do pattern matching:
push scopes on the way down and pop them on the way up.

\begin{verbatimtab}
enterBlock:	BLOCK {crtScope = new LocalScope(crtScope);};
exitBlock:	BLOCK {crtScope = crtScope.getEnclosingScope();};
\end{verbatimtab}

The rules match the same \verb=BLOCK= but we execute one on the way down and the other on the way up:

\begin{verbatimtab}
topdown:	enterBlock | enterMethod | .. ;
bottomup:	exitBlock | exitMethod | .. ;
\end{verbatimtab}

Resolving references (a preview of what we'll do later):
\begin{verbatimtab}
idref:	{$start.hasAncestor(EXPR)}? ID
	{Symbol s = currentScope.resolve($ID.text);}
;
\end{verbatimtab}


\subsection{Managing Symbol Tables for Data Aggregates}

A \emph{data aggregate scope}, like any other scope,
contains symbols and has a place within the scope tree.
Code outside this scope can access its members
with an expression like \verb=user.name=.

\verb=struct=s and calsses are similar:
they are simultaneously symbols, user-defined types and scopes.
Class scopes have a superclass scope as well as the usual enclosing scope:
forward references require an extra AST pass.


\subsubsection{Building Scope Trees for Structs}

\verb=struct= scopes are just another node in the scope tree.
From within it we resolve symbols by scanning upward in the scope tree,
like any other nested scope.

\verb=struct=s have no executable statements
so the only symbols to look up are type names.

We must also resolve symbols within struct scopes from the outside.
The general (recursive) rule for resolving \verb=«expr».x=
is to determine the type of \verb=«expr»=
and then look up \verb=x= in that scope.
For \verb=a.b.y= in Listing \ref{symbols:struct-example}
we look up \verb=a= and find its type to be \verb=A=.
Looking up \verb=b= in \verb=A= yields scope \verb=B=.
Finally we look up field \verb=y= in \verb=B=.

\begin{program}
\verbatimtabinput{./code/symbols/struct-example.c}
\caption{Struct example\label{symbols:struct-example}}
\end{program}

Member access expressions are like scope overrides:
they say exactly in which scope to look for a field.
We must not look in the struct's enclosing scope (e.g. for \verb=a.b.x=).
We need two different resolve methods:
one for looking up isolated symbols (like \verb=a=)
and one for resolving member access expressions (like \verb=a.x=).


\subsubsection{Building Scope Trees for Classes}

Classes can inherit members from the superclass.
They have two parent scopes:
the usual enclosing (lexical) scope and a superclass scope.
To resolve symbols, we sometimes chase the enclosing scope pointer
and sometimes the superclass pointer.
The language definition dictates which stack to use:
usually check the superclass before the enclosing scope.

If we don't find an isolated symbol (\verb=x=)
in the class or its superclasses,
look it up in the global scope.
When referring to members from outside that class (\verb=a.x=)
we shouldn't see global variables.


\subsubsection{Forward References}

We want to allow forward references for class members
in Listing \ref{symbols:class-fwd-ref}.
We can make two passes over the input (AST):
one to define symbols and one to resolve them.

\begin{program}
\verbatimtabinput{./code/symbols/class-fwd-ref.cpp}
\caption{Forward references for class members\label{symbols:class-fwd-ref}}
\end{program}

But we don't want to allow this outside of classes
(as our language Cymbol is a C++ subset).
We can use a trick involving token indexes:
if a symbol reference resolves to a local or global symbol,
the reference's token index must come after the definition's token index.


\subsection{P18 Symbol Table for Data Aggregates}

Like P17 but with different symbol resolution. We need a StructSymbol class.
Rules for building the scope tree:

\begin{description}
\item[Start of file]
\verb=push= a GlobalScope, \verb=def= BuiltinType objects for int, float, void.

\item[Var declaration x]
\verb=ref= x's type.
\verb=def= x as a VariableSymbol object in the current scope.
This works for globals, struct fields, parameters and locals.

\item[struct declaration S]
\verb=def= S as a StructSymbol object in the current scope
and \verb=push= it as the current scope.

\item[Method declaration f]
\verb=ref= f's return type.
\verb=def= f as a MethodSymbol object in the current scope
and \verb=push= it as the current scope.

\item[\{]
\verb=push= a LocalScope as the new current scope.

\item[\}]
\verb=pop=, revealing the previous scope as the current scope.
This works for struct, methods and local scopes.

\item[Variable reference x]
\verb=ref= x starting in the current scope.
If not found, look in the immediately enclosing scope if any.

\item[Member access «expr».x]
Compute the type of «expr» using the previous rule and this one recursively.
\verb=ref= x only in that type's scope, not in any enclosing scopes.

\item[End of file]
\verb=pop= the GlobalScope.
\end{description}


\subsection{P19 Symbol Table for Classes}

Tracks symbols and builds a scope tree
for non-nested classes with single inheritance.
We'll replace StructSymbol with ClassSymbol.
They'll point at their superclasses as well as their enclosing scopes.
To support forward references, we'll make two passes:
first define them then resolve references.

We need to pass information between the phases.
We'll stash it in the AST.
A \verb='{'= node records the associated LocalScope.
As we define symbols, record them in the AST.
As we resolve symbol references, store that in the AST as well
(for future phases).

During the definition phase,
when defining a variable we make its ID node point to a new VariableSymbol
(which also has a back pointer \verb=def= to the ID node).

This phase also sets a \verb=scope= field for some AST nodes
(e.g.\ a variable definition's type node (\verb=int=, \verb=A=),
variable references (\verb=x=)).
The resolution phase needs the scope in order to resolve those symbols.
The resolution phase updates e.g.\ the \verb=symbol= field of AST nodes,
the \verb=type= field of VariableSymbols.

To check for illegal forward references,
we use the \verb=def= field of VariableSymbol.
Rules for the definition phase:

\begin{description}
\item[Start of file]
\verb=push= a GlobalScope, \verb=def= BuiltinType objects for int, float, void.

\item[Identifier reference x]
Set x's \verb=scope= field to the current scope.

\item[Variable declaration x]
\verb=def= x as a VariableSymbol, \verb=sym=, in the current scope.
Works for globals, class fields, parameters, locals.
Set \verb=sym.def= to x's ID AST node.
Set \verb=ID.symbol= to \verb=sym=.
Set the \verb=scope= field of x's type AST node (e.g.\ \verb=int=, \verb=A=)
to the current scope.

\item[Class declaration C]
\verb=def= C as a ClassSymbol object \verb=sym= in the current scope.
Set \verb=sym.def= to the class name's ID AST node.
Set that ID node's \verb=symbol= to \verb=sym=.
Set the \verb=scope= field of C's superclass AST node to the current scope.
\verb=push= \verb=sym= as the current scope.

\item[Method declaration f]
\verb=def= f as a MethodSymbol object, \verb=sym=, in the current scope.
Set \verb=sym.def= to the function name's ID AST node.
Set that ID node's \verb=symbol= to \verb=sym=.
Set the \verb=scope= field of f's return type AST node to the current scope.
\verb=push= \verb=sym= as the current scope.

\item[\{]
\verb=push= a LocalScope as the new current scope.

\item[\}]
\verb=pop=, revealing previous scope as current scope.

\item[End of file]
\verb=pop= the GlobalScope.
\end{description}

Resolution phase rules:

\begin{description}
\item[Variable declaration x]
Let \verb=t= be the ID node for x's type (e.g.\ \verb=int=, \verb=myclass=).
\verb=ref= \verb=t=, yielding \verb=sym=.
Set \verb-t.symbol=sym-. \verb=x.symbol= is a VariableSymbol.
Set \verb+x.symbol.type=sym+.

\item[Class declaration C]
Let \verb=t= be the ID node for C's superclass.
\verb=ref t=, yielding \verb=sym=.
Set \verb=t.symbol= to \verb=sym=.
Set C's \verb=superclass= field to \verb=sym=.

\item[Method declaration f]
Let \verb=t= be the ID node for f's return type.
\verb=ref t=, yielding \verb=sym=.
Set \verb=t.symbol= to \verb=sym=.
Set the \verb=type= field of the MethodSymbol for f to \verb=sym=.

\item[Variable reference x]
\verb=ref x=, yielding \verb=sym=. Set \verb=x.symbol= to \verb=sym=.

\item[this]
Resolve to the surrounding class scope.
Set the symbol field of \verb=this='s ID node to the surrounding class scope.

\item[Member access «expr».x]
Resolve «expr» to a particular type symbol, \verb=esym=, using these rules.
\verb=ref x= within \verb=esym='s scope, yielding \verb=sym=.
Set \verb=x.symbol= (\verb=x='s ID node) to \verb=sym=.
\end{description}

Resolving members (\verb=«expr».x=) only looks in the class hierarchy
(doesn't continue to global scope).
Algorithm for resolving symbol references:

\begin{description}
\item[x in method]
Look in enclosing local scope, then method scope, then enclosing class scope.
If not found look up in the class hierarchy.
If not found look in global scope.

\item[x in field definition init expression]
Look in surrounding class scope.
If not found, look up the class hierarchy.
If not found, look in global scope.

\item[x in global scope]
Look in surrounding global scope.
\end{description}


\subsection{P20 Computing Static Expression Types}

Make an AST and walk it twice:
first to define symbols, then to resolve symbols and compute expression types.
We can't just match pattern `\verb=ID=', we need some context information;
but we don't need a full tree grammar.
We can use a pattern matcher to look for \verb=EXPR= root nodes,
then invoke a type computation rule to traverse the expression subtree.
Listing \ref{symbols:expr-pattern-matching}.
See page 202 for all the details.

\begin{program}
\verbatimtabinput{./code/symbols/expr-pattern-match.g}
\caption{\label{symbols:expr-pattern-matching}}
\end{program}


\subsection{P21 Automatic Type Promotion}

Describes how to automatically and safely promote arithmetic operand types.
We can automatically convert between types
as long as we don't lose information.
We call this \emph{promotion}
(we can safely \emph{widen} types but not \emph{narrow} them in general).

There is a simple formula to express valid type-to-type promotions.
First order and number the arithmetic types from narrowest to widest.
Then we can automatically promote
\emph{type\textsubscript{i}} to \emph{type\textsubscript{j}}
as long as $i < j$.
For Cymbol, the ordered arithmetic type list is: char, int, float.

Compiler semantic analyzers typically modify an IR tree
to incorporate value promotion nodes.
Translators can usually get away with just annotating the tree
and checking for promotions later during code generation.

Besides expressions, static type analyzers have to promote values
in assignments, return statements, function call arguments,
and array index expressions.\\
E.g.\ \verb-float f=1;-, \verb=a['z']=.

To implement arithmetic type promotion, we need two functions.
The first returns the result type given an operator and two operand types.
See Listing \ref{symbols:result-type}.
A result type of \verb=void= indicates the operation is illegal,
which we'll exploit in P22, Enforcing Static Type Safety.

\begin{program}
\verbatimtabinput{./code/symbols/result-type}
\caption{resultType() function\label{symbols:result-type}}
\end{program}

The second tells us whether we need to promote an operand
for a particular operator and destination type.
See Listing \ref{symbols:promote-from-to}.
A promotion result of \verb=null= means \emph{no promotion necessary}
(not \emph{invalid promotion}).
To store the results of these functions, annotate AST nodes using two fields:
\verb=evalType= and \verb=promoteToType=.
See page 210 for a diagram.

\begin{program}
\verbatimtabinput{./code/symbols/promote-from-to}
\caption{promoteFromTo() function\label{symbols:promote-from-to}}
\end{program}

Computing the result type can be done with a table
(using operand types as indexes).
For each operator, make a table
showing how to map any two operand types to a result type.
The promotion table is similar.
(for Cymbol semantics we can get away with a single promotion table).


\subsection{P22 Enforcing Static Type Safety}

Adds type compatibility checks to P21.
An operation must be defined for the operand types it's applied to:\\
\verb-resultType(opType1, op, opType2) != void-\\
If looking for a value of type \verb=t=,
the value's type must be \verb=t= or promotable to \verb=t=:\\
\verb+value-type == dest-type || value-promoted-type == dest-type+\\
This computation is called \verb=canAssignTo=.

Symbol category rules:
\begin{itemize}
\item in \verb=x.y=, \verb=x= must be a \verb=struct=
\item in \verb=f()=, \verb=f= must be a function symbol
\item in \verb=a[]=, \verb=a= must be an array symbol
\end{itemize}

Type compatibility rules for Cymbol:
\begin{itemize}
\item \verb=if= conditionals must evaluate to a \verb=boolean= value
\item array reference indexes must be integers
\item the left and right sides of an assignment must have compatible types
\item function call arguments and formal function declarations
must have compatible types
\item \verb=return= expressions and function return types must be compatible
\item the two operands of a binary arithmetic operation
must have compatible types
\item the operand of unary operators must have an appropriate type
\end{itemize}
