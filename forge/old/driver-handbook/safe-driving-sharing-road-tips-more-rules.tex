\section{Safe Driving Practices}

Always signal, whatever you want to do.
For left\&right turns: during the last 100 feet.
Signal before every lane change; also look over your shoulder.
At least 5 seconds before changing lanes at freeway speeds.
Even when you don't see other vehicles around.

Keep a "space cushion" on all sides of your vehicle.
Look 10 to 15 seconds down the road to see hazards early.
Avoid tailgating by using the "three-second rule".
Allow a cushion of 4 seconds or more when being followed by a tailgater,
driving on a wet or gravel road,
following large vehicles that block your view ahead.

Do not stay in another driver's blind spot or drive directly alongside them.
Drive either ahead of or behind them.
Keep a space between you and parked cars.
Leave plenty of room between your vehicle and motorcyclists or bicyclists.

In heavy rain or snow, if you're unable to see more than 100 feet ahead
don't drive faster than 30 mph.

Only pass a vehicle if a hill or curve is at least one-third of a mile away.

Use headlights:
when cloudy, raining, snowing or foggy;
whenever you use your windshield wipers.

When exiting the freeway signal for about 5 seconds.

Avoid passing any vehicle (including motorcycles) on two-lane roads:
it's dangerous.
Return to your lane after passing a vehicle
when you can see both its headlights in your inside rear view mirror.


\section{Sharing the Road With Other Vehicles}

Do not drive through a safety zone (space set aside for pedestrians and
marked by raised buttons or markers on roadway).
When people are boarding or leaving a streetcar or trolley
and there's no safety zone,
stop behind the vehicle's nearest door or vehicle platform.

Yield the right-of-way to emergency vehicles using a siren and red lights:
drive to the right edge of the road and stop; never stop in an intersection.

Neighborhood Electric Vehicles (NEV) and Low-Speed Vehicles (LSV):
reach a maximum speed of 25 mph
and are restricted from roadways with a speed limit greater than 35 mph.
You may see lanes marked as "NEV Use Only".

Move over and slow down when approaching a roadside emergency
along a highway or freeway.


\section{Important Tips}

It's illegal to drive and read/write emails, text messages or instant messages
or talk on the phone.
Adults may use a phone with a hands-free device.

In heavy rain at 50 mph you will hydroplane.

If you are involved in a collision:
\begin{itemize}
\item You must stop
\item Call 911 if anyone is hurt
\item Move your vehicle out of the traffic lane if no one is injured or killed
\item Show your driver license, registration card,
evidence of financial responsibility and current address to the other driver
\item You (or your insurance agent) must make a written report to the police
or California Highway Patrol within 24 hours if someone is killed or injured
\item You (or your insurance agent)
must make a written report to the DMV within 10 days
\item If you hit a parked vehicle or other property,
leave a note with your name, phone number and address
in or securely attached to the vehicle or property.
Report the collision to the city police or the CHP.
\item The same if your parked car rolls away and hits another vehicle
\item If you kill or injure an animal, call the police or CHP;
don't try to move it or leave it to die
\end{itemize}

Report any collision you have to the DMV, whether you caused it or not.
The police won't do this for you.


\section{Additional Driving Laws/Rules}

Don't smoke if a minor is in the vehicle.\\
Don't wear a headset or earplugs in \emph{both} ears.\\
Don't put signs on the windows, don't hang objects on the mirror.

Use your headlights 30 minutes after sunset and leave them on
until 30 minutes before sunrise.\\
Dim your lights to low beams within 500 feet of a vehicle coming towards you
or within 300 feet of a vehicle you are following.\\
Move your vehicle out of the traffic lane, if safe to do so,
if you're involved in a collision.

\paragraph{What to do during an Enforcement Stop}
Acknowledge the officer's presence by turning on your right turn signal.\\
Move your vehicle to the right shoulder of the road.\\
Turn off your radio.\\
Remain inside your vehicle.
Never step out of your vehicle unless an officer directs you to do so.\\
Place your hands in clear view, including all passengers' hands.

\paragraph{}
Don't drink and drive. Don't drink in a vehicle.
If you carry any alcohol inside a vehicle, the container must be full,
sealed and unopened.
Otherwise it must be in the trunk.
