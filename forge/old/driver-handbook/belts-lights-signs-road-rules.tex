\section{Seat Belts}

Driver and all passengers ($\geq$ 6 years old or $\geq$ 60 lbs.)
must wear seatbelts.
Children under 6 years old and weighing less than 60 lbs.
must be secured in a child passenger restraint system and ride in the back.
It's illegal to leave a child $\leq$ 6 years old unattended in a vehicle.


\section{Traffic lights and signs}

\subsection{Traffic signal lights}

\begin{description}
\item[Solid Red]
Means "STOP".
You can make a right turn against a red light after you stop then yield to
pedestrians, bicyclists, and vehicles close enough to be a hazard.
Make the right turn only when it is safe.
Do not turn if a "NO TURN ON RED" sign is posted.

\item[Red Arrow]
Means "STOP".
Remain stopped until the green signal or green arrow appears.
Do not turn against a red arrow.

\item[Flashing Red]
Means "STOP".
After stopping, you may proceed when it is safe.
Observe the right-of-way rules.

\item[Solid Yellow]
Means "CAUTION".
The red signal is about to appear.
Stop if you can do so safely, otherwise cross the intersection cautiously.

\item[Yellow Arrow]
Means the "protected" turning time period is ending.
Be prepared to obey the next signal,
which could be the green or red light or the red arrow.

\item[Flashing Yellow]
Warns you to "PROCEED WITH CAUTION".
You do not need to stop for a flashing yellow light,
but you must slow down and be especially alert
before entering the intersection.

\item[Flashing Yellow Arrow]
Means turns are permitted (unprotected)
but you must first yield to oncoming traffic and pedestrians
and then proceed with caution.

\item[Solid Green]
Give the right of way to any
vehicle, bicyclist or pedestrian in the intersection.
Means "GO".
If you are turning left, make the turn
only if you have enough space to complete the turn
before creating a hazard for anyone else.
Do not enter the intersection if you cannot get completely across
before the light turns red.

\item[Green Arrow]
Means "GO".
You must turn in the direction the arrow is pointing after yielding to any
vehicle, bicyclist or pedestrian still in the intersection.
The green arrow allows you to make a "protected" turn.
Oncoming vehicles, bicycles and pedestrians are stopped by a red light
as long as the green arrow is lighted.

\item[Traffic Signal Blackout]
If a traffic signal light is not working, proceed cautiously
as if the intersection is controlled by a stop sign in all directions.
\end{description}

\subsection{Traffic Signs}

\begin{description}
\item[Eight-sided red STOP] sign indicated you must make a full "STOP"
at the crosswalk or white limit line, otherwise at the corner.

\item[Three-sided red YIELD] sign indicates you must slow down
and be ready to stop.

\item[Square red and white regulatory] sign
indicates that you must follow its instruction.

\item[Yellow and black circular] sign
indicates that you are approaching a railroad crossing.

\item[X-shaped signs] with a white background stating RAILROAD CROSSING
indicate that you must slow down and be ready to stop.

\item[Five-sided] sign indicates you are near a school.
Stop for children in the crosswalk.

\item[Four-sided diamond-shaped] sign
warns of specific road conditions and dangers ahead.

\item[White rectangular] sign indicates that you must obey important rules.
\end{description}


\section{Laws and Rules of the Road}

\subsection{Right of Way Rules}

\subsubsection{Pedestrians}
Drive cautiously when pedestrians are near
because they may suddenly cross your path.
Always stop for any pedestrian crossing at corners or other crosswalks,
even if the crosswalk is in the middle of the block,
at corners with or without traffic lights,
whether or not the crosswalks are marked by painted lines.

Do not pass a vehicle that has stopped at a crosswalk.
A pedestrian you cannot see may be crossing the street.
Do not stop in a crosswalk.
Pedestrians have the right-of-way in crosswalks.

\subsubsection{Intersections}
At intersections without \emph{STOP} or \emph{YIELD} signs,
slow down and be ready to stop.
Yield to traffic and pedestrians already in the intersection
or just entering the intersection.
Also, yield to the vehicle or bicycle which arrives first,
\emph{or} to the vehicle or bicycle on your right
if it reaches the intersection at the same time as you.

At \verb=T= intersections without \emph{STOP} or \emph{YIELD} signs,
yield to traffic and pedestrians on the through road.
They have the right of way.

When you turn left, give the right of way to all vehicles approaching
that are close enough to be dangerous.
When you turn right, be sure to check for pedestrians
and bicyclists coming up behind you on the right.

When there are \emph{STOP} signs at all corners,
stop first then follow the above rules.

If you have parked off the road, are leaving a parking lot, etc.
yield to all traffic before reentering the road.

Safety suggestion: while waiting to turn left,
keep your wheels pointed straight ahead.
If your wheels are pointed to the left, and a vehicle hits you from behind,
you could be pushed into oncoming traffic.

\subsubsection{Roundabouts}
Vehicles entering or exiting the roundabout must yield to all traffic
including pedestrians.
Slow down as you approach the roundabout.
Yield to pedestrians and bicyclists crossing the roadway.
Drive counter-clockwise, do not stop or pass other vehicles.
Use your turn signals when changing lanes or exiting.

\subsubsection{Mountain Roads}
When two vehicles meet on a steep road where neither vehicle can pass,
the vehicle facing downhill must yield the right-of-way,
by backing up until the vehicle going uphill can pass.

\subsection{Speed Limits}

\emph{Basic Speed Law}: never drive faster than is safe for current conditions.

\subsection{Maximum Speed Limit}

On most highways 65 mph. You may drive 70 mph where posted.
Unless otherwise posted, 55 mph on two-lane undivided highways
and for vehicles towing trailers.

\subsection{Reduced Speeds}

\subsubsection{Heavy Traffic or Bad Weather}
Drive slower when there's heavy traffic or bad weather.
If you block the reasonable movement of traffic by driving too slowly
you may get a ticket.
If you choose to drive slower than other traffic,
do not drive in the "Number 1" (fast) lane.

\subsubsection{Towing Vehicles, Buses, or Large Trucks}
Drive in the right hand lane or in a lane specially marked for slower vehicles.
If no lanes are marked and there are four or more lanes in your direction,
drive in either of the two on the right.

\subsubsection{Around Children}
When within 500–1000 feet of a school
while children are outside or crossing the street,
the speed limit is 25 mph unless otherwise posted.
Also 25 mph if the school grounds have no fence and children are outside.
Always drive more carefully near
schools, playgrounds, parks and residential areas.

Near schools look for
bicyclists, pedestrians, school safety patrols and school busses.
School busses flashing yellow lights
are preparing to stop to let children off the bus.
This warns you to slow down and be prepared to stop.
When the bus flashes red lights (top front and back)
you must stop from either direction
(and remain stopped while the lights are flashing).
If the school bus is on the other side of a divided or multilane highway,
you do not need to stop.

\subsubsection{Blind Intersections}
Speed limit is 15 mph.
An intersection is considered blind if there are no stop signs at any corner
and you can't see for 100 feet in either direction
during the last 100 feet before crossing.
Trees, bushes, buildings or parked cars at intersections can block your view.

\subsubsection{Alleys}
Speed limit is 15 mph.

\subsubsection{Near Railroad Tracks}
The speed limit is 15 mph within 100 feet of a railroad crossing
where you cannot see the tracks for 400 feet in both directions.
You may drive faster than 15 mph if the crossing is controlled by gates,
a warning signal, or a flag man.

Look in both directions and listen for trains, be ready to stop if necessary.
Never stop on the railroad tracks.
Watch for vehicles that must stop before they cross train tracks: these include
buses, school buses, and trucks with hazardous loads.
Stop between 15 and 50 feet if flashing red lights are active
or you see or hear a train.
Do not go under lowering gates or around lowered gates.
Do not proceed until the red lights stop flashing, even if the gate rises.

\subsubsection{Near Streetcars, Trolleys or Buses}
The passing speed limit is no more than 10 mph:
this applies at a safety zone or an intersection where the
streetcar, trolley or bus is stopped.
Usually where streetcars or trolleys and vehicles share the roadway.

\subsubsection{Business or Residential Districts}
The speed limit is 25 mph, unless otherwise posted.


\section{Traffic Lanes}

\subsection{Line Colors}

\begin{description}
\item[Solid yellow lines]
mark the center of a road used for two-way traffic.
\item[Broken yellow lines]
indicate that you may pass if the broken line is next to your driving lane.
\item[Two solid yellow lines]
indicate no passing. Never drive to the left of these lines unless you're:
turning left at an intersection,
turning into or out of a private road or driveway,
in a carpool lane that has a designated entrace on the left
or instructed by construction signs to do so.
\item[Two sets of solid double yellow lines]
spaced two or more feet apart are considered a barrier.
Do not drive on or over this barrier or make a left turn or a U-turn across it
except at designated openings.
\item[Solid white lines] mark traffic lanes going in the same direction,
such as one-way streets.
\item[Broken white lines] separate traffic lanes on roads
with two or more lanes in the same direction.
\end{description}

\subsection{Choosing a Lane}

The left or \emph{fast} lane is called the \emph{Number 1 Lane}.
The lane to the right of it is called the \emph{Number 2 Lane},
then \emph{Number 3} etc.
Drive in the lane with the smoothest flow of traffic
(middle if road has 3 lanes, right if it has two).

\subsection{Changing Lanes}

Changing lanes includes moving from one lane to another,
entering the freeway from an on-ramp,
entering the road from a curb or the shoulder.
Signal, look in all mirrors, check traffic behind and beside you,
glance over your shoulder,
look for all vehicles and bycicles in your blind spot.

\subsection{Passing Lanes}

Never drive off the road (or on the shoulder) to pass.
Don't pass other vehicles at crossroads or railroad crossings.
Pass traffic on the left.
You may pass on the right only when:
on an open highway with two or more lanes in your direction or
the driver ahead of you is turning left.

\subsection{Carpool/High Occupancy Vehicles Lanes}

A carpool lane is a special freeway lane used only for carpools,
buses or decaled low-emission vehicles.
Signs tell the minimum number of people per vehicle required
for the carpool lanes and the days\&hours of the week
when the requirement applies.

\subsection{Center Left Turn Lanes}

Located in the middle of a two-way street
and marked on both sides by two painted lines.
The inner line is broken and the outer line is solid.
If a street has it, you must use it when turning left
or starting a permitted U-turn.

\subsection{Turnout Areas and Lanes}

Sometimes marked on two-lane roads:
drive into them to allow cars behind you to pass.
Some two-lane roads have passing lanes:
if driving slowly where passing is unsafe
and five or more vehicles are following you
drive into the turnout areas to let them pass.


\section{Turns}

\paragraph{Left Turns}
Drive close to the center divider lane or into the left turn lane.
Begin signaling about 100 feet before the turn.
Look over left shoulder, reduce speed, stop.
Left turn against a red light can only be made from a one way street
onto a one way street.

\paragraph{Right Turns}
Drive close to the right edge of the road.
If there's a bike lane, drive into it no more than 200 feet before the turn.
Begin signaling about 100 feet before the turn.
Look over right shoulder, reduce speed, stop.
Right turn against red light if no sign prohibits.

No turn (left or right) against a red arrow.

\subsection{U-Turns}

Signal and use the far left lane or the center left turn lane.
Legal:
\begin{itemize}
\item Across a double yellow line when safe and legal
\item In a residential district:
	\begin{itemize}
	\item If no vehicles approaching within 200 feet
	\item Whenever a traffic sign, light or signal
	protects you from approaching vehicles.
	\end{itemize}
\item At an intersection on a green light or green arrow,
unless a "No U-Turn" sign is posted.

\item On a divided highway,
only if an opening is provided in the center divider.
\end{itemize}

Illegal:
\begin{itemize}
\item On railroad crossing
\item On a divided highway by crossing a dividing section, strip of land,
two sets of double yellow lines.
\item When you can't see clearly 200 feet in each direction
\item Where a "No U-Turn" sign is posted
\item On a one-way street
\item In front of a fire station
\item In business districts.
Areas with churches, apartments, clubs, public buildings (except schools)
are also considered business districts.
Turn only at an intersection or where openings are provided for turns.
\end{itemize}


\section{Parking}

On a hill with a curb present:
If headed downhill turn front wheels towards curb,
if headed uphill turn front wheels away from curb
and roll back to gently touch the curb.
If no curb, probably turn towards edge of the road all the time.

Colored curbs:
\begin{description}
\item[White] stop only long enough to pick up or drop off passengers or mail.
\item[Green] park for a limited time;
look for time limit on a sign or painted on the curb.
\item[Yellow] stop no longer than the time posted
to load or unload passengers or freight.
\item[Red] no stopping, standing, or parking.
\item[Blue] parking allowed only for disabled persons displaying placard.
\end{description}

Illegal parking:
\begin{itemize}
\item on marked on unmarked crosswalk,
on (or partially blocking) a sidewalk, in front of a driveway
\item within 3 feet of a sidewalk ramp for disabled persons
\item in a space for parking or fueling zero-emission vehicles
\item in a tunnel or on a bridge
\item within 15 feet of a fire hidrant or a fire station driveway
\item within 8 feet of a railroad track
\item between a safety zone and the curb, double parking
\item on the wrong side of the street
\item at a red curb
\item on a freeway
\end{itemize}

When parking alongside a curb on a level street,
the front and back wheels must be parallel and within 18 inches of the curb.
