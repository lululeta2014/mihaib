\documentclass[a4paper,12pt]{article}

\usepackage{fontspec}
\usepackage{xunicode}
\usepackage{xltxtra}

\usepackage{amsmath}
\usepackage{hyperref}
\usepackage[all]{hypcap}

%\hypersetup{pdfborder = {0 0 0}}
\hypersetup{colorlinks=true,%
%	citecolor=black,%
%	filecolor=black,%
%	linkcolor=black,%
	urlcolor=blue,%
%
	pdftitle={DiceLottery},%
	pdfauthor={Mihai Borobocea}%
}

\begin{document}
\title{DiceLottery math}
\date{2009}
\author{}
\maketitle

We have 2 event universes,
and in each universe all events have equal probability.
We want to perform a bidirectional mapping between these 2 universes
\(U_{Src}\) and \(U_{Dest}\).
Let \(N_S\) and \(N_D\) stand for
the event counts of the respective event universes:
\[N_S = |U_S|, N_D = |U_D|\]

The mapping has an additional parameter --- \(K\) ---
the number of outcomes from \(U_S\) which are used
for a single mapping operation to an event in \(U_D\).
\(K\) outcomes from \(U_{Src}\) represent an event in \({U_{Src}}^K\).
All the \(N_S\) events in \(U_{Src}\) have equal probability,
so all the \({N_S}^K\) events in \({U_{Src}}^K\)
have equal probability as well.
The mapping is thus performed between
\({N_S}^K\) and \(N_D\) elements.
The following relation must hold (it establishes a minimum value for \(K\)):
\begin{equation}
\label{eq:K}
{N_S}^K \geq N_D
\end{equation}

Having the values of \(N_S\), \(N_D\) and \(K\),
the values of \(Q\) and \(R\) can be computed by integer division
(with \(R < N_D\)):
\begin{equation}
\label{eq:QR}
{N_S}^K = Q * N_D + R
\end{equation}

In order for all possible \(N_D\) results of the mapping
to have equal probability,
only \(Q * N_D\) source elements can be mapped (out of the total \({N_S}^K\)).
The remaining \(R\) cannot be mapped and the user is asked to try again.
The fraction of mappings which cannot be performed is:
\begin{equation}
\label{eq:invalid-mappings}
\text{invalid fraction of mappings} =
\frac{R}{{N_S}^K} < \frac{N_D}{{N_S}^K} \leq 1
\end{equation}


\section{Treating the K source outcomes}

\subsection{Treating K outcomes as a single larger outcome}

We wish to map
\(e_{K-1}, e_{K-2}, \dots, e_1, e_0 \in U_{Src}\)
to a single event \(e' \in {U_{Src}}^K\).
\begin{equation}
e' = e_{K-1} * {N_S}^{K-1} + e_{K-2} * {N_S}^{K-2} + \dots + e_1 * N_S + e_0
\end{equation}

\subsection{Treating a larger outcome as K source outcomes}

We wish to map \(e' \in {U_{Src}}^K\) to
\(e_{K-1}, e_{K-2}, \dots, e_1, e_0 \in U_{Src}\)
\begin{equation}
\begin{array}{l|l}
e' = q_{-1} & \\
q_{-1} = q_0 * N_S + r_0 & e_0 = r_0\\
q_0 = q_1 * N_S + r_1 & e_1 = r_1\\
\vdots & \vdots \\
q_{K-2} = q_{K-1} * N_S + r_{K-1} & e_{K-1} = r_{K-1}
\end{array}
\end{equation}


\section{The mapping operation}

The mapping is performed between \({U_{Src}}^K\) and \(U_{Dest}\).

\subsection{Forward mapping}
We wish to map \(e \in {U_{Src}}^K\) to \(e' \in U_{Dest}\).
According to equations \ref{eq:QR} and \ref{eq:invalid-mappings}:
\begin{equation}
\begin{array}{r l}
e' = e \bmod N_D, & \text{if } e < q * N_D \\
\text{invalid}, & \text{if } e \geq q * N_D
\end{array}
\end{equation}

\subsection{Reverse mapping}
We wish to map \(e' \in U_{Dest}\) to \(e \in {U_{Src}}^K\).
There are multiple solutions. All \(Q\) are listed below:
\begin{equation}
e \in \{ e', e' + N_D, e' + 2*N_D, \dots, e' + (Q-1) * N_D \}
\end{equation}

\end{document}
